\section{Задачи и развитие систем машинного зрения} \label{chapter1.1}

%Машинное зрение — это применение компьютерного зрения для промышленности и производства. Для решения задачи автоматизации извлечения антропометрических признаков применяется МЗ в области пошива одежды и фитнес-тестирования.
%\mrk{МЗ и его краткая история} 
\emph{Машинное зрение} (МЗ) - это междисциплинария область, получившая в настоящее время широкое развитие. Концепция обработки изображений и МЗ является важным разделом компьютерного зрения и основана на комбинации многих дисциплин. Прогресс в области машинного зрения определяется развитием математической теории (статистики, теории вероятности, методов решения линейных и нелинейных уравнений и т.д.), численных методов, а также развитием вычислительных ресурсов, в частности, смартфонов. Долгое время теоретические исследования в области МЗ опережали вычислительные возможности ЭВМ, что затрудняло их использование для решения практических задач. В \cite{Paragios2008} условно выделен ряд этапов развития средств зрения МЗ и отмечено, что только с 1970-х годов XX века появилась возможность обрабатывать большие наборы данных. С тех пор концепции и методы МЗ привлекают все большее внимание.
\begin{figure}[htb]
\centering
\includegraphics [scale=0.8] {images/h47.png}
\begin{center}
%\captionsetup{justification=justified, labelsep=period}
\caption{Описание структуры МЗ.} \label{img47}
\end{center}
\end{figure}
%\mrk{Структура МЗ.}
Системы МЗ включают в себя следующие основные компоненты \cite{Rosenfeld2000}:
\begin{itemize}
	\item Подсистему формирования изображений, которая сама, как правило, включает разные компоненты, например, оптическую систему, осветщение и ПЗС- или КМОП-матрицу;
	\item Вычислитель;
	\item Алгоритмы анализа изображений, которые могут реализовываться программно на процессорах общего назначения, аппаратно в структуре вычислителя и даже аппаратно в рамках подсистемы формирования изображений.
\end{itemize}

%\mrk{МЗ - важнейший источник информации для ИИ. Приложения.}
МЗ предоставляет важнейшую информацию для создания систем искусственного интеллекта. Такие системы могут получать информацию как из полученных  изображений так и из наборов многомерных данных различной природы \cite{Fan2013}. Сочетание машинного зрения с другими областями, такими как: информационные технологии, связь, электроника, автоматическое управление и т.д. дает нам множество применений в области науки, неразрушающего контроля качества, промышленных роботов и т.д.

%\mrk{Применение МЗ -- повторение}
В последние годы проводятся научно-исследовательские работы в различных областях обработки и распознавания изображений. МЗ стало самостоятельной дисциплиной \cite{Huang1991, Kevin1991}. В настоящее время методы МЗ реализованы во множестве устройств, применяются технологии обработки и управления, на основе изображения. Приведем краткий обзор приложений систем МЗ в робототехнике.


\subsection{Машинное зрение в робототехнике}

%\mrk{Структура МЗ роботов и решаемые задачи.}
Система МЗ роботов является интегрированной системой, включающей одну или несколько камер \cite{Abebe2016}. В зависимости от назначения, система должна обнаруживать положение объектов в поле зрения камеры (ROI, Region of Interest). На основе этой системы, робот решает задачи определения местоположения и направления движения объектов. Для достижения высокой эффективности (скорости и точности обработки в режиме видеопотока) работы необходимо обеспечивать эффективную совместную работу системы, включая аппаратные средства и программное обеспечение \cite{Zhu2004, Vayda1991}.

%\mrk{Применение МЗ в промышленности.}
В современной промышленности системы МЗ роботов применяется в различных областях, таких как:

\begin{itemize}
	\item автомобильная промышленность с автоматизированными системами сборки и обработки двигателей, автомобильных кузовов;
	\item пищевая промышленность, например, система контроля автоматического закрывания пакета, закрывания контейнера  обнаружения посторонних предметов в пище при упаковке;
	\item фармацевтическая промышленность: контроль упаковки и партии, обнаружение дефектов;
	\item военно-силовые структуры: обнаружение целей беспилотными летательными аппаратами, анализ багажа, анализ поведения групп людей в общественных местах.
	\item обеспечение безопасности, предупреждение преступности на основе распознавания лиц, отпечатков пальцев, формы человека;
	\item индустрии развлечений и спорта: системы автоматического управления камерой слежения за объектами в футбольном матче, гонках и т.д.
\end{itemize}

%\mrk{Классы задач МЗ в робототехнике.}
Таким образом, современная робототехника требует решения многих сложных задач МЗ, в том числе следующих:

\begin{itemize}
	\item ориентация во пространстве и определение расстояний до объектов;
	\item распознавание и классификации различных объектов, интерпретации сцен, навигация;
	\item обнаружение людей, распознавание их лиц и анализ эмоций;
	\item восстановление изображений и подавление шумов, повышение разрешнения изображений, в том числе получение изображений со сверхразрешением.
\end{itemize}

\subsection{Машинное обучение и информационный поиск}
%\todo[inline]{Связь МЗ и машинного обучения}
Методы обработки изображений широко применяются в МЗ. Кроме того, машинное обучение (МО) содержит алгоритмы и методы, которые позволяют эффективно работать с данными изображений и видеопоследовательности. Сочетание МЗ и МО является перспективным для создания полезных и эффективных приложений в различных областях науки и техники.

%\mrk{Машинное обучение и интернет-проекты}
На сегодняшний день МО используется в бесчисленном множестве интернет-проектов. У многих известных IT-компаний (к примеру, Яндекс, Google, Facebook, Microsoft) на нём базируются многие ключевые технологии \cite{Cormier2016}.

%\mrk{Поиск, сравнение и классификация изображений в Интернет}
Задачи поиска изображений по содержанию также разнообразны \cite{Meer2000}. Здесь важными являются задачи понижения размерности, извлечения признаков, сравнение содержимого изображения для обнаружения и распознавания объектов на изображениях. Такие алгоритмы очень полезны для создания приложений, таких как: классификация данных (изображения, видео и т.д.), поиск товаров на основе изображений для интернет-магазинов, для извлечения изображений в геоинформационных системах, для систем биометрической идентификации, для специализированного поиска изображений в социальных сетях (например, для поиска лиц людей, привлекательных для пользователя) \cite{Findface} и т.д., вплоть до поиска изображений в интернете.

%\mrk{МЗ-задачи решаются методами машинного обучения. -- хороший кандидат на первый абзац раздела.}
Методами МО при разработке систем КЗ решаются проблемы распознавания объектов. Разработка алгоритмов классификации является одной из важнейших областей МО \cite{Murino2000}. Методы глубокого обучения (deep learning) \cite{Bengio2013} требуют огромных вычислительных ресурсов, и даже для обучения распознаванию ограниченного класса объектов могут требоваться несколько дней работы на вычислительном кластере. При этом в будущем могут быть разработаны еще более мощные, но возможно требующие еще больших вычислительных ресурсов методы.  Отметим, что использование специфики решаемой задачи позволяет существенно сократить вычислительную сложность, однако требует более глубокого понимания сути решаемой задачи.  

%\todo[inline,color=cyan]{Ну и какая СОДЕРЖАТЕЛЬНО ПРЕДСТАВЛЕННАЯ связь М-обучения с МЗ и приложениями на производстве и робототехнике, кроме специфики задачи???}

\subsection{Мобильные приложения компьютерного зрения}

%\mrk{МЗ на моб.устройствах. Фичи мобил}
Задачи КЗ все шире используются в приложениях для персональных мобильных устройств, таких как смартфоны, планшеты и прочее. В частности, число смартфонов неуклонно растет и уже практически превысило по численности население земли \cite{Battiato2012}. Часть задач по обработке изображений для мобильных устройств с камерами совпадает с задачами для цифровых фотоаппаратов. Основное отличие заключается в качестве объективов и в условиях съемки. В спектре аппаратного обеспечения, доступного для решения задач отметим модули с доступом к интернет, наличия интернета, GPS и конечно мощного процессора и большой памяти. Именно с этим связано появление такого термина, как <<smart phone>> \cite{Hannuksela2007}.

%\mrk{???? + МЗ+Дополненная реальность....???}
В этой связи решение, казалось бы, идентичных задач для разных устройств может различаться, что делает эти решения высоко востребованными на рынке. Приложения для смартфонов выходят на рынок, наблюдается огромный интерес к компьютерному зрению и приложений дополненной реальности в мобильных устройствах \cite{Shubina2010}.

%\mrk{Примеры задач МЗ на мобилах}
В настоящее время существует много приложений, использующих обработку изображений, для мобильных устройств. Например, программа автоматической коррекции лицевых дефектов (таких, как морщины, угри, веснушки) в соответствии с заданным стилем (натуральный, классический, контраст и т.д.). Это такие приложения, как camera360, Perfect Selfie. Приложения по обработке изображений на видео позволяют пользователям создавать тематические видео из фотографий, анимации и т.д. -- MiniMovie \cite{Mini}, Imovie-editing \cite{Videoedit} и Маскарад \cite{2016}, которые преобразует видео в стиль картин художника Ван Гога и мультипликации.

%\mrk{Задачи ориентации, распознавание кепоинтов. ДОПИСАТЬ ПРИМЕРЫ.}
Более сложные задачи, связанные с сопоставлением (отождествлением сопряженных точек) на изображениях, оценкой трехмерной структуры сцены, определением изменения ориентации камеры, распознавание объектов, а также анализа лиц людей, также находят свои приложения в отслеживании объектов, безопасности для смартфонов. Повышение качества решения перечисленных задач требует совершенствования и разработки нового математического и программного обеспечения, и адаптации его к вычислительным возможностям мобильных устройств, а также к их периферии.

%\mrk{Резюме - МЗ на мобилах - это круто.}
Перечисленные примеры и задачи показывают, что класс приложений КЗ и МО на мобильных устройствах крайне широк и любое продвижение в развитии и исследовании методов обработки изображений является актуальным.
%\todo[inline,color=cyan]{Какая связь СОДЕРЖАНИЯ РАЗДЕЛА с АТРОПОМЕТРИЕЙ?}

