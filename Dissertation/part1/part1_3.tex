\section{Оценивание точности численного метода извлечения антропометрических признаков}
В этой части представлен результат извлечения антропометрических признаков на основе алгоритма компьютерного зрения из $50$ изображений. Извлечение ключевых точек (меток - опорных точек) состоит из $2$ главных этапов:

\begin{itemize}
	\item \textbf{Шаг 1:} предварительная обработка изображения, преобразование исходного цветного изображения в изображение в оттенках серого, фильтрация шумов, разряжение пикселей, эквализация гистограммы для получения лучшего качества изображения;
	\item \textbf{Шаг 2:} извлечение точек признаков с использованием алгоритма вычитания фона, обнаружения человеческого лица, сегментации изображения, поиска опорных точек для вычисления антророметрических признаков.
\end{itemize}

Для оценки эффективности процесса извлечения признаков применяются средняя абсолютная ошибка - $MAPE$:
\begin{equation}\label{eq26}
M=\frac{1}{n}\sum^n_{i=1}\left|\frac{A_t-F_t}{A_t}\right|,
\end{equation}
Где:

\begin{itemize}
	\item $A_t$: результат измерений, рассчитанных вручную;
	\item $F_t$: результат извлечения антропометрических признаков.
\end{itemize}
Эксперимент был использован для проверки погрешности размеров человеческих частей на основе извлечения количества признаков. На основании результатов для калибровки точности алгоритма.

Проведенный анализ показал, что выбор оптимального числа опорных точек не является тривиальной задачей. Опорные точки описывают расположение точек, лежащих на контуре объекта, возможно принадлежат части человеческого тела. В диссертационной работе проведены эксперименты с различными количествами опорных точек. Особенно трудные определенные области,  между грудью и талией, бедрами и талией.

\textbf{Случай 1}: Результат извлечения 24 опорных точек на контуре человеческого тела. Иллюстрация расположения опорных точек (рис. \ref{img15}): 
\begin{figure}[ht!]
\centering
\includegraphics [scale=0.8] {images/h15.png}
\begin{center}
%\captionsetup{justification=justified, labelsep=period}
\caption{Результат расположения 24 опорных точек \cite{long1,long2}.} \label{img15}
\end{center}
\end{figure}

Количество опорных точек в этом случае включает в себя 24 точки: головы, шея, рук, плеча, груди, талии, бедер, бицепса. 

\begin{table}[b!]%
\begin{center}
\caption{Результат извлечения антропометрических признаков и погрешности системы компьютерного зрения \cite{long1,long2}.}\label{tab1}
  \begin{tabular}{|c|c|c|c|c|c|c|}
    \hline
    \multirow{2}{*}{Размеры} & {Ручной} & \multicolumn{3}{c}{Результаты системы (см)} & {$\sum^n_{i=1}\left|\frac{A_t-F_t}{A_t}\right|$} &{$M$} \\
      & метод (см) &1 &2 &3& & \\
    \hline
Обхват груди &95	&93.65	&94.11	&93.32	&0.04126	&0.01375\\
\hline
Обхват талии               &79	&76.06	&78.80	&80.15	&0.05430	&0.01810\\
\hline
Обхват бедер              &96.5	&93.07	&95.19	&95.78	&0.05658	&0.01886\\
\hline
Длина руки           &53	  &52.45	&54.32	&54.72	&0.06774	&0.02258\\
\hline
Обхват бицепса      &28	 &25.68	  &27.09	&27.60	&0.12964	&0.04321\\
\hline
Обхват шеи          &37	 &35.71	  &36.67	&36.79	&0.04946	&0.01649\\
\hline
Длина спины         &40	 &38.50	  &40.11	&40.01	&0.04050	&0.01350\\
\hline
Длина плеча       &36	 &35.86	  &36.19	&35.50	&0.02306	&0.00769\\
\hline
Ширина плеча         &14	 &13.91	  &13.56	&14.74	&0.09071	&0.03024\\
\hline
Высота груди        &18	 &17.78	  &17.60	&17.86	&0.04222	&0.01407\\
\hline
  \end{tabular}
\end{center}
\end{table}%\vspace{10mm}

\textbf{Случай 2:} Применение метода калибровки для улучшения точности извлечения признаков (рис. \ref{img152}).
\begin{figure}[ht!]
\centering
\includegraphics [scale=0.8] {images/h152.png}
\begin{center}
%\captionsetup{justification=justified, labelsep=period}
\caption{Результат расположения 28 опорных точек  \cite{long1,long2}.} \label{img152}
\end{center}
\end{figure}
\begin{table}[b!]%
\begin{center}
\caption{Результат извлечения антропометрических признаков и погрешности системы компьютерного зрения после применения калибровки извлеченных точек \cite{long1,long2}.}.\label{tab3}

Количество опорных точек в этом случае включает в себя 28 точки: головы, шея, рук, плеча, груди (верхней и нижней), талии, бедер, бицепса.
 
  \begin{tabular}{|c|c|c|c|c|c|c|}
    \hline
    \multirow{2}{*}{Размеры} & {Ручной} & \multicolumn{3}{c}{Результаты системы (см)} & {$\sum^n_{i=1}\left|\frac{A_t-F_t}{A_t}\right|$} &{$M$} \\
      & метод (см)  &1 &2 &3 & & \\
    \hline
Обхват груди &95	&95.03	&95.00	&95.02	&0.00053	&0.00018 \\
\hline 
Обхват талии             &79	&78.46	&79.05	&78.89	&0.00557	&0.00186\\

\hline
Обхват бедер               &96.5	&97.29	&96.11	&96.4	&0.00010	&0.00003\\

\hline
Длина руки           &53	&52.93		&52.1	&52.93	&0.03531	&0.01177\\

\hline
Обхват бицепса      &28	&28.24	&28.03	&28.07	&0.01071	&0.00357\\

\hline
Обхват шеи         &37	&38.35	&38.13	&38.05	&0.09672	&0.03224\\

\hline
Длина спины         &40	&38.35	&39.96	&39.81	&0.04325	&0.01442\\

\hline
Длина плеча       &36	&36.01	&36.17	&36.1	&0.00971	&0.00324\\

\hline
Ширина плеча        &14	&13.97	&14.02	&14.02	&0.00071	&0.00024\\

\hline
Высота груди       &18	&18.05	&18.11	&18.01	&0.01500	&0.00500\\

\hline
  \end{tabular}
\end{center}
\end{table}%\vspace{10mm}

\begin{figure}[ht!]
\centering
\includegraphics [scale=0.7] {images/h16.png}
\begin{center}
%\captionsetup{justification=justified, labelsep=period}
\caption{Погрешность в обоих случаях в извлечении антропометрических признаков. Случай 1: результат извлечения 24 опорных точек на контуре человеческого тела. Случай 2: результат извлечения 28 опорных точек на контуре человеческого тела.} \label{img16}
\end{center}
\end{figure}

С антропометрическими признаками как длина руки, длина плеча и т.д. (не сложная структура геометрия) можно применить формулу Евклид. Тогда с количеством 24 опорных точек будет удовлетворено. Но если антропометрические признаки со сложной структурой геометрии в виде эллипса как талия, грудь, бедро необходимо использовать больше опорных точек, чтобы точно определить и  избежать путаться между точками.

\begin{figure}[ht!]
\centering
\includegraphics [scale=0.7] {images/h18.png}
\begin{center}
%\captionsetup{justification=justified, labelsep=period}
\caption{Процент правильной классификации (\%)  в обоих случаях в извлечении антропометрических признаков. Случай 1: результат извлечения 24 опорных точек на контуре человеческого тела. Случай 2: результат извлечения 28 опорных точек на контуре человеческого тела} \label{img18}
\end{center}
\end{figure}

Для вычисления технической погрешности измерений при извлечении антропометрических признаков используется средняя абсолютная ошибка $MAPE$. В данной работе выполняется извлечение антропометрических признаков 100 раз для одного человека. Получена гистограмма технической погрешности измерений (рис.\ref{img45}).

\begin{figure}[ht!]
\centering
\includegraphics [scale=1] {images/h45.png}
\begin{center}
%\captionsetup{justification=justified, labelsep=period}
\caption{Результат анализа погрешности.} \label{img45}
\end{center}
\end{figure}

В (рис.\ref{img45}) показано, что среднее арифметичесное $mean = 0.19131$, медиана $median = 0.18512$ и скошенность $skewness= 0.460$. В этом распределении среднее арифметичесное и медиана приблизительны, скошенность $\in \left[-1;1\right]$. Исходя из этого, можно заключить, что погрешность измерений имеет нормальное распределение (распределение Гаусса) (рис. \ref{img44}).

\begin{figure}[ht!]
\centering
\includegraphics [scale=1] {images/h44.png}
\begin{center}
%\captionsetup{justification=justified, labelsep=period}
\caption{Гистограмма технической погрешности измерений.} \label{img44}
\end{center}
\end{figure}



  