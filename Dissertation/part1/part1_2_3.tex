\subsection{Математическая модель на основе разрезов на графах для сегментации изображений}
\textbf{Математическая модель сегментации изображений на основе разреза на графах}

Предположим, что множество $S=\left\{s_i|i=1, ..., n\right\}$ представляет собой изображение. Сегментация определяется набором случайной величины $A=\left\{A_i|i=1, ..., n\right\}$, $A_i\in L $ которого указывает маркировка $s_i$ и $L=\left\{L_j|j=1, ..., m\right\}$ является набор меток. Сегментация решается путем минимизации функции энергии:
\begin{equation}\label{eq25}
E\left(f\right) = \sum_{p \in P}D_p\left(f_p\right)+\sum_{p,q \in N}V_{p,q}\left(f_p,f_q\right).
\end{equation}
Найти маркировки $f:P\rightarrow L$ , что сводит к минимуму $E\left(f\right)$ из множеств пикселей $P$, набор этикеток $L,N \in P$ является система соседства по пикселям. $D_p\left(f_p\right)$ является функцией, которая получена из данных наблюдений, и которая измеряет стоимость присвоения метки $f_p$ на пиксель $p$. $V_{\left(p,q\right)} \left(f_p,f_q\right)$ , и измеряет стоимость присвоения метки $f_p$,$f_q$ на соседние пиксели $p$, $q$. Используется для улучшения пространственного сглаживания \cite{Boykovv2001, Boykov2004}.

На каждом шаге алгоритм выполняет каждый поток, который соединяет от $s$ к $t$. Сложность алгоритма зависит от числа ребер, осуществляется по индукции всех потоков. Движение потока увеличивается, по меньшей мере один раз на каждом шаге алгоритма. Поэтому, сложность алгоритма не превосходит $O\left(f\right)$, где  $f$ - максимальный поток в графе.  $E$ - это число ребер графа, так что сложность будет $О\left(E\ast f\right)$.

После этого в полученном графе находится минимальный разрез, который делит граф на $2$ части. Вес ребер между метками обеспечивает выполнение заданных пользователем ограничений: маркировки объекта будут отнесены к объекту, маркировки фона - к фону.

\textbf{Численный метод для сегментации изображений на основе разреза на графах}

\textbf{Проблема о максимальном потоке} (maximum flow problem): найти поток $f$ такой, что величина потока максимальна. 
\[
f: \sum_u f\left(u\rightarrow v\right)=\sum_wf\left(v\rightarrow w\right). 
\]

Величина потока (value of flow) — сумма потоков из источника.
\[
\left|f\right|:= \sum_w f \left(s\rightarrow w\right) - \sum_u f\left(u\rightarrow s\right).
\]

\textbf{Минимальный разрез} — разрез с минимальной пропускной способностью. 
$C_{min} \left(A,B\right)=\sum_{u\in S, v \in T} W_{uv}$.

Остаточная сеть $G_f\left(V,E_f\right)$ - сеть с  пропускной способностью $c_f\left(u,v\right)=c\left(u,v\right)-f\left(u,v\right)$ и без потока Ford – Fulkerson (1956) \cite{Ford1956}. Максимальный поток на основе алгоритма (\ref{img12}).


\begin{figure}[ht!]
\centering
\includegraphics [scale=1] {images/h12.png}
\begin{center}
%\captionsetup{justification=justified, labelsep=period}
\caption{Блок-схема алгоритма разреза на графах.} \label{img12}
\end{center}
\end{figure}

Метод разреза графов находит сильные локальные минимумы энергетической функции. Метод достаточно мощный, чтобы решать множество полезных задач, и он может быть применен к решению проблемы графика min-среза. 
Разрез (s-t cut) — разбиение множества всех вершин V на два подмножества $A$ и $B$ таких, что $s\in A $, $t\in B$, причем пересечение $A$ и $B$ равно пустому множеству. Если рассматривать весовые коэффициенты, связанные с каждым узлом в качестве емкости потока, можно показать, что максимальное количество энергии потока от источника равно емкости минимального разреза. Поэтому проблема минимального разреза также известна как проблема максимального потока. Нужно выбрать необходимый поток и необходимый разрез $\left(S, T\right)$, а затем следить неравенством \cite{Boykov12001}:
\begin{equation}\label{eq21}
\left|f\right| = \sum_w f\left(s\rightarrow w\right) - \sum_u f \left(u\rightarrow s\right),
\end{equation}
\[
=\sum_{v\in S}\left(\sum_w f \left(v \rightarrow w \right) - \sum_u f\left( u \rightarrow v \right) \right),
\]
\[
= \sum_{v \in S} \left(\sum_{w \in T} f \left( v \rightarrow w\right) - \sum_{u \in T} f \left( u \rightarrow v \right)\right),
\]
\[
\leq \sum_{v \in S}\sum_{w \in T} f \left( v \rightarrow w \right) после  f \left(u \rightarrow v\right) \geq 0,
\]
\[
\leq \sum_{v \in S}\sum_{w \in T} c \left(v\rightarrow w \right)  после f \left(u \rightarrow v \right) \leq c \left( v \rightarrow w\right),
\]
\[
=\left\|S, T\right\|.
\]

Грань между пиксельной $i$ и $j$ будем обозначать $W^I_{ij}$ и терминал весов между пиксельной $i$ и источником $\left(s\right)$ и мойкой $\left(t\right)$. $W^s_i$ и $W^t_i$ соответственно задаются \cite{Parvathy}:
\begin{equation}\label{eq22}
W^i_{ij} = e^{\left(-\frac{r\left(i,j\right)}{\sigma R}\right)}e^{\left(-\frac{\left\|w\left(i\right)-w\left(j\right)\right\|^2}{\sigma W}\right)},
\end{equation}
\begin{equation}\label{eq23}
W^s_i = \frac{p\left(w\left(i\right)| i \in s\right)}{p \left(w \left(i\right)\right)|i \in s + p \left(w\left(i\right)|i \in t\right)},
\end{equation}
\begin{equation}\label{eq24}
W^t_i = \frac{p\left(w\left(i\right) | i \in t\right)}{p \left(w \left(i\right)\right)|i \in s + p \left(w\left(i\right)|i \in t\right)}.
\end{equation}
Где:

\begin{itemize}
	\item $r\left(i,j\right)$ расстояние между пиксельной $i$ и $j$;
	\item $\left\|.\right\|$ обозначает евклидову норму;
	\item $\sigma R$ и $\sigma W$ настраивают параметры, взвешивая различные значения;
	\item $W^I_{ij}$ содержит сходство между пикселями;
	\item $W^s_i$, $W^t_i$ описывает пиксель фона и переднего плана соответственно.
\end{itemize}


