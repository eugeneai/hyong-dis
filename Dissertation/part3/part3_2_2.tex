\subsection{Применение системы компьютерного зрения в антропометрии для фитнеса (E-Fitness)}
Программа <<E-Fitness>> предназначена для фитнеса.
При запуске программы появляется главное окно формы (рис.\ref{img38}).

\begin{itemize}
\item <<Measure Me>> В рабочей области отображается видеопоследовательность, область обнаружения объектов и извлечение антропометрических признаков;
\item <<Credit>> Контакты с командой разработчиков;
\item <<Exit>> Позволяет осуществить выход из программы.
\end{itemize}

\begin{figure}[ht!]
\centering
\includegraphics [scale=0.5] {images/h38.png}
\begin{center}
%\captionsetup{justification=justified, labelsep=period}
\caption{Основный интерфейс приложения E-Fitness} \label{img38}
\end{center}
\end{figure}

\textbf{Основные функции приложения E-Fitness}\ref{img39}

\begin{itemize}
	\item Позволять пользователям добавлять информацию;
	\item Показать 3D-модель на основе антропометрических признаков и классификации данных (рис.\ref{img40});
	\item Анализ антропометрических признаков на основе стандартов Fitness (рис.\ref{img41}).
\end{itemize}

\begin{figure}[ht!]
\centering
\includegraphics [scale=0.5] {images/h39.png}
\begin{center}
%\captionsetup{justification=justified, labelsep=period}
\caption{Интерфейс работы программы } \label{img39}
\end{center}
\end{figure}

\begin{figure}[ht!]
\centering
\includegraphics [scale=0.5] {images/h40.png}
\begin{center}
%\captionsetup{justification=justified, labelsep=period}
\caption{Результат 3D-моделей} \label{img40}
\end{center}
\end{figure}

Приложение E-Fitness даёт 2 полезные функции в области фитнеса. \textit{Первая функция:} Учёт индекса массы тела (ИМТ - body mass index BMI) (рис.\ref{img40} - величина, позволяющая оценить степень соответствия массы человека и его роста и тем самым косвенно оценить, является ли масса недостаточной, нормальной или избыточной. ИМТ рассчитывается по формуле:
$I=\frac{m}{h^2}$
где: $m$ - масса тела в килограммах, $h$ - рост в метрах.

\begin{figure}[ht!]
\centering
\includegraphics [scale=0.5] {images/h41.png}
\begin{center}
%\captionsetup{justification=justified, labelsep=period}
\caption{Результат анализа антропометрических признаков на основе стандартам Fitness} \label{img41}
\end{center}
\end{figure}

\textit{Вторая функция:} Анализ антропометрических признаков по фитнес-стандартам, чтобы позволять пользователи отслеживать здоровье каждый день. Где красная линия - реальные размеры, зеленная линия - размеры по фитнес-стандартам. Приложение позволяет пользователям сравнить свои параметры с размерами по фитнес-стандартам (рис.ref{img41}). 