% Формат А4, 14pt (ГОСТ Р 7.0.11-2011, 5.3.6)
\documentclass[a4paper,14pt]{extreport}

\input{../common/packages}  % Пакеты общие для диссертации и автореферата
\input{configurations/dispackages}         % Пакеты для диссертации
\input{configurations/userpackages}        % Пакеты для специфических пользовательских задач

\input{configurations/setup}               % Упрощённые настройки шаблона 
\usepackage{array}
\input{configurations/preamblenames}       % Переопределение именований, чтобы можно было и в преамбуле использовать
\input{../common/data}      % Основные сведения
\input{../common/styles}    % Стили общие для диссертации и автореферата
\input{configurations/disstyles}           % Стили для диссертации
\input{configurations/userstyles}          % Стили для специфических пользовательских задач
\input{../biblio/bibliopreamble}% Настройки библиографии из внешнего файла (там же выбор: встроенная или на основе biblatex)
\SetKwInput{KwData}{Исходные данные}
\SetKwInput{KwResult}{Результат}
\SetAlgorithmName{Алгоритм}{алгоритм}{список алгоритм}
\input{configurations/inclusioncontrol}    % Управление компиляцией отдельных частей диссертации
\newcommand\tab[1][1cm]{\hspace*{#1}}
\begin{document}

\input{configurations/names}             % Переопределение именований

% Структура диссертации (ГОСТ Р 7.0.11-2011, 4)
\include{other_parts/title_phd}           % Титульный лист
\include{other_parts/contents}        % Оглавление
\include{part0_introduction/introduction}    % Введение

%%%%%%%%%%%%%%%%%%%%%%%  ГЛАВА 1 %%%%%%%%%%%%%%%%%%%%%%%%%%%%%%%%%%%%%%%%%
\chapter{Применение компьютерного зрения и машинного обучения в задачах обнаружения и классификация объектов} \label{chapt1}
В этой главе представлен краткий обзор результатов, которые успешно использованы в задачах обнаружения и классификации объектов на изображениях и видеопоследовательностях.  Отдельное внимание уделяется методам обработки изображений, методам машинного обучения. Особое внимание уделяется трем основным этапам: предварительная обработка данных, извлечение признаков, обнаружение и классификация объектов.

%%------------- 1.1------------------%
\section{Обзор разработки и применения машинного обучения} \label{chapter1.1}
%----------1.1.1--------------------%
\subsection{Общее понятие машинного обучения}
\textit{Машинное обучение }- обширный подраздел искусственного интеллекта, изучающий методы построения алгоритмов, способных обучаться \cite{h20, h21, h22}. Машинное обучение рассматривается как метод создания компьютерных программ, которые используют прошлый опыт, наблюдение или данные для улучшения своей будущей работы.

В 1950 году ученый Алан Тьюринг создал тест Тьюринга, чтобы проверить компьютер с истинным интеллектом или нет \cite{h23}. Первая программа машинного обучения была написана Артуром Сэмюэлем и являлась игрой в шашки \cite{h24}. Подход машинного обучения перешел от управляемых знаний \cite{h25} к управляемым данным \cite{h26}. Ученые начали создавать компьютерные программы для анализа больших объемов данных и получения выводов.

В настоящее время алгоритмы машинного обучения исследованы и разработаны для создания полезных приложений во всех сферах жизни.
Например: Google Brain был разработан в 2011 году в глубокой нейронной сети \cite{h27}, которая может изучить и идентифицировать многие объекты. В 2015 году Microsoft создала «Инструмент обучения распределенной машиной - Distributed Machine Learning Toolkit» \cite{h28}, который позволяет эффективно распределять решение задач машинного обучения на нескольких компьютерах.
Внедрение методов машинного обучения позволяет повысить эффективность работы многих приложений и устройств.

Если раньше компьютерное обучение требовало сложного программного обеспечения, передовых компьютерных систем и опытных ученых, чтобы понять механику машинного обучения, то теперь разработаны высокоэффективные алгоритмы машинного обучения, которые проще настраивать и применять.
%----------1.1.2--------------------%
\subsection{Классификация машинного обучения}
Машинное обучение разделено на три основных типа: обучение с учителем, обучение без учителя, обучение с подкреплением. Наиболее популярными являются обучение с учителем, обучение без учителя. Обучение с подкреплением - это улучшенный тип модели обучения с учителем.

\textbf{Обучение с учителем }\cite{h29,h30}: Это метод машинного обучения для изучения данных, которым дана данная метка. Имеется множество объектов и множество возможных меток. Существует некоторая зависимость между метками и объектами, но она неизвестна. Известна только конечная совокупность прецедентов — пар «объект, метка», называемая обучающей выборкой. Задачей обучения с учителем является прогнозирование вывода на основе входного значения. На основе этих данных требуется восстановить зависимость, то есть построить алгоритм, способный для любого объекта выдать достаточно точный ответ.

Обучение с учителем применяется к двум основным задачам классификации для получения множества возможных ответов. Их называют идентификаторами (именами, метками) классов, например задачи — ответы регрессии являются действительными числами или векторами, например: прогноз цен на акции \cite{h31,h32}, прогнозы погоды \cite{h33}; распознавание рукописного текста \cite{h34, h35}.

\textbf{Обучение без учителя} \cite{h36}: это метод машинного обучения, который изучает широкий класс задач обработки данных, в которых известны только описания множества объектов (обучающей выборки), и требуется обнаружить внутренние взаимосвязи, зависимости, закономерности, существующие между объектами. 

Обучение без учителя является наиболее распространенным применением кластеризации \cite{h37, h38, h39, h40}, поиска правил ассоциации, заполнения пропущенных значений и т.д.

%----------1.1.3--------------------%
\subsection{Машинное обучение и практическое применение}
Машинное обучение широко применяется во многих сферах бизнеса и жизни: маркетинге, финансах, банковском деле, страховании, науке, здравоохранении, безопасности, Интернете и т.д. Многие организации и компании в мире применяют методы машинного обучения для своих деловых операций и получили огромные преимущества. Существует много различных практических применений машинного обучения. Машинное обучение используется для двух широко распространенных проблем: интеллектуального анализа данных и распознавания образов.

\textbf{Интеллектуального анализа данных}\cite{h41, h42, h43}: алгоритмы машинного обучения применимы к большой базе данных для обнаружения правил или знаний в этих данных и предсказывают будущую актуальную информацию. Некоторые области применения алгоритмов машинного обучения в области интеллектуального анализа данных: анализ финансовых данных, телекоммуникационной отрасли, анализ биологических данных, некоторые приложения в науке и в сфере безопасности.

\textbf{Распознавание образов} \cite{h44, h45, h46, h47}: применение алгоритмов машинного обучения для обнаружения закономерностей в данных, обычно изображения и звука. Полная система распознавания образов состоит из устройства сбора данных, процесса извлечения данных для вычисления числовой информации из входных данных, процесса классификации данных на основе извлеченных признаков.

Общие применения распознавания образов: автоматическое распознавание речи, различные типы распознавания текста, автоматическая идентификация, идентификация и классификация дефектов дорожного покрытия, автоматическое распознавание номерных знаков и т.д.

%----------1.1.4--------------------%
\subsection{Методы машинного обучения в задчах классификации}
Классификация - это одна из основных задач машинного обучения, который сортирует объекты согласно принадлежности предопределенным классам. Объекты классифицируются по классам на основе значений извлеченных признаков анализируемых объектов. Типичные алгоритмы классификации включают: нейронные сети \cite{h48}, деревья решений \cite{h49, h50}, байесовские сети \cite{h51}, метод опорных векторов \cite{h52, h53}. Все эти подходы создают модели, способные классифицировать неизвестные новые объекты на основе данных аналогичных объектов по которым предварительно проводилось обучение. Изображения и видео содержат большой объем плохо структурированной информации. Поэтому необходим интеллектуальный анализ данных для создания различных приложений.

Решение задачи классификации проводится в два этапа:

-- \textbf{Шаг 1: Обучение.} Цель этого шага - построить модель для определения набора классов данных. Эта модель строится путем анализа наборов обучающих данных, каждый из которых определяется значением признаков. Каждый набор данных относится к одному из предопределенных классов. Набор данных анализируется для построения модели классификации.

-- \textbf{Шаг 2: Тестирования и оценка.} На этом этапе используется модель классификации, полученная на шаге 1. Сначала оценивается точность модели с помощью тестовых наборов. Эти наборы выбираются независимо от образцов, которые были изучены на шаге 1. Количество правильно классифицированных объектов - это точность модели классификатора, основанная на тестовых данных.

\textbf{Методы объединения классификации:} Существует два подхода к решению этой проблемы.

-- Первый подход - заключается в создании каждого классификатора независимо друг от друга, затем используется метод голосования для выбора конечного результата объединения.

-- Второй подход - создание основных классификаторов и присвоение значимости результатам каждого классификатора.

\textit{Метод Бэггинг} \cite{h54, h55}: Бэггинг использует первый подход. Бэггинг создает классификаторы из признаков набора данных объекта и алгоритмов машинного обучения, каждый из которых создает основный классификатор. Классификаторы будут объединены путем массового голосования.

\textit{Метод Бустинг} \cite{h56, h57}: использует второй подход. Этот метод строит классификатор на основе разного весового значения набора данных обучения. Если модель обучения неверно предсказывает увеличение весового значения, и наоборот, уменьшение. Это помогает алгоритму повысить точность.

%----------1.1.5--------------------%
\subsection{Некоторые проблемы в задачах классификации}
Существуют две проблемы с результатами предсказания классификаторов, которые являются предвзятыми предсказаниями - bias \cite{h58, h59}, и предсказанными результатами дисперсии – variance \cite{h60}. Ошибка в машинном обучении состоит из трёх частей: Bias, Variance и Noise. С шумом, как правило, сделать ничего нельзя: он отражает влияние на результат факторов, не учитываемых в модели.  

Ошибка прогнозирования классификатора представляет собой сумму ошибок смещения и дисперсию алгоритма машинного обучения, который используется. Цель создания хорошего классификатора - найти способ минимизировать смещение и дисперсию. Существует общий метод уменьшения дисперсии путем построения набора сингулярного классификатора, а затем голосование на основе результатов классификации с входными данными.

С Bias и Variance ситуация иная. Первое отображает ошибку, связанную с плохо выработанными зависимостями, уменьшается с ростом сложности и при больших значениях говорит о том, что модель недостаточно изучена. Второе отображает чувствительность модели к колебаниям в значениях входных данных, увеличивается с ростом сложности и свидетельствует о переобучении. Отсюда возникает такое понятие как bias/variance tradeoff.

Сложность модели заключается из двух аспектах. Первый аспект общий для всех – это количество используемых признаков или размерность входных данных. К тому же, всегда есть вероятность, что некоторые столбцы могут быть лишними, и их удаление не ухудшит модель. Поскольку при одинаковой точности следует выбирать более простое решение, полезным будет даже такое изменение.

%%------------1.2-----------------------------%
\section{Анализ подходов машинным обучением в обнаружении и классификации дефектов дорожного покрытия}
В этом разделе содержится основная информация: описания дефектов дорожного покрытия, обзор алгоритмов машинного обучения для обнаружения и классификации дефектов дорожного покрытия, анализ программ обнаружения и классификации дефектов дорожного покрытия.
%-------------1.2.1-------------%
\subsection{Описание дефектов дорожного покрытия}
Дефекты дорожного покрытия зависят от причины дефектов, например: климат, транспорт, длительное использование во времени, качество материалов и т.д. Таким образом, природа этих дефектов совершенно иная.

Множеству исследований удалось решить проблему извлечения признаков дефектов. В этом тезисе дефекты анализируются, как в таблице \ref{tab1} придерживается государственный стандарт РФ ГОСТ Р 50597 \cite{h146}, европейский стандарт EN 13108-4 \cite{h148} и Вьетнамский стандарт 1472/QĐ-BGTVT \cite{h147}:
\begin{table}[h!]
  \centering
  \begin{tabular}{ | m{2cm} | m{9cm} | c |  }
    \hline
    Типов дефекта & Описание & Примеры \\ \hline
    %\begin{minipage}[t]{5cm}
   Глубокие трещины
    %\end{minipage}
    & 
    %\begin{minipage}{5cm}
      \begin{itemize}
        \item 	глубокие трещины обычно встречаются в параллельных областях дорожного покрытия. Положение колес автомобиля оказывает прямое воздействие на поверхность. 
 
      \end{itemize}
			&
			 \begin{minipage}{.3\textwidth}
			\centering
      \includegraphics[width=0.5\linewidth]{img1}
    \end{minipage}
    %\end{minipage}
    \\ \hline
%---- dong 2
  Сеть трещин
    %\end{minipage}
    & 
    %\begin{minipage}{5cm}
      \begin{itemize}
        \item 	тип дефекта покрытия, который развивается из горизонтальных трещин и вертикальных трещин. Этот дефект обычно появляется на больших участках бетонной поверхности.
      \end{itemize}
			&
			 \begin{minipage}{.3\textwidth}
			\centering
      \includegraphics[width=0.5\linewidth]{img2}
    \end{minipage}
    %\end{minipage}
    \\ \hline
	%---- dong thu 3
	  Выбоина
    %\end{minipage}
    & 
    %\begin{minipage}{5cm}
      \begin{itemize}
        \item 	это расположение дорожной поверхности, которая подвергается воздействию нескольких слоев абразивного материала.

      \end{itemize}
			&
			 \begin{minipage}{.3\textwidth}
			\centering
      \includegraphics[width=0.5\linewidth]{img3}
    \end{minipage}
    %\end{minipage}
    \\ \hline
			%---- dong thu 4
	  Сдвиги, волны
    %\end{minipage}
    & 
    %\begin{minipage}{5cm}
\begin{itemize}
	\item  неровности в виде чередующихся поперечных выступов и впадин с пологими краями, вызванные смещением верхних слоев дорожных одежд капитального и облегченного типа.
\end{itemize}

			&
			 \begin{minipage}{.3\textwidth}
			\centering
      \includegraphics[width=0.5\linewidth]{pic56}
    \end{minipage}
    %\end{minipage}
    \\ \hline
					%---- dong thu 5
	  Гребенки
    %\end{minipage}
    & 
    %\begin{minipage}{5cm}

\begin{itemize}
	\item  неровности в виде чередующихся правильных и четко выраженных поперечных выступов и впадин на покрытиях переходного типа.
\end{itemize}

			&
			 \begin{minipage}{.3\textwidth}
			\centering
      \includegraphics[width=0.5\linewidth]{pic57}
    \end{minipage}
    %\end{minipage}
    \\ \hline
						%---- dong thu 6
	  Колея
    %\end{minipage}
    & 
    %\begin{minipage}{5cm}

\begin{itemize}
	\item  деформация покрытия с образованием углублений по полосам наката с гребнями или без гребней выпора.
\end{itemize}

			&
			 \begin{minipage}{.3\textwidth}
			\centering
      \includegraphics[width=0.5\linewidth]{pic58}
    \end{minipage}
    %\end{minipage}
    \\ \hline
  \end{tabular}
  \caption{Описание дефектов дорожного покрытия}\label{tab1}
\end{table}

%-------------1.2.2-------------%
\subsection{Методы извлечения признаков дефектов дорожного покрытия}
Изображения и видео - ввод данных системы обнаружения и классификации дефектов дорожного покрытия. Эти данные предварительно обрабатываются, из них извлекаются признаки, затем формируются векторы признаков. Однако в реальности исходные данные известны с точностью до уровня шума. Фактические условия окружающей среды (свет, дорожные характеристики и т.д.) оказывают большое влияние на сбор данных. Подход к решению проблемы заключается в выборе алгоритмов, настройке их параметров, комбинаций методов обработки. Поэтому системы привлекают внимание многих ученых и исследовательских центров по всему миру.

В данной работе мы сосредоточились на изучении, таких признаков извлеченных из изображений, как признаки контуров, признаки формы дефектов дорожного покрытия, а затем применяем соответствующие алгоритмы машинного обучения для их классификации и сокращения времени вычисления и затрат памяти.

Дефекты покрытия являются объектами входных изображений системы выявления и классификации дефектов. Извлечение признаков объекта является важным шагом в анализе данных. Для этого в работе предполагается использование методов обработки изображений для анализа и преобразования информации изображения в векторы признаков для обнаружения и классификации дефектов дорожного покрытия. Метод с использованием вейвлет-преобразования \cite{h61}, метод выборки \cite{h62}, мера анизотропии \cite{h63}, метод на основе алгоритмов нечеткой логики \cite{h64}, метод выборочной экстракции \cite{h65} для решения проблемы извлечения признаков дефектов дорожного покрытия на основе данных избражений и видео.

В \cite{h108} представлен метод извлечения с использованием вейвелет-преобразования. Использование вейвлет-преобразования позволяет проводить анализ дефектов трещин на дорожном покрытии. На рисунке \ref{pic1} описан процесс разложения изображения на основе вейвлет-преобразования, где L и H соответственно фильтры.
\begin{figure}[ht!]
\centering
\includegraphics[width=0.8\linewidth]{pic1}
\caption{Метод извлечения признаков дефектов дорожного покрытия методом Wavelet -Random transform \cite{h108}}
	\label{pic1}
	\end{figure}
	
Анализ формы очень важен для выделения дефектных пикселей с соседними пикселями. В исследовании использовался метод извлечения линейных признаков глубоких трещин на дорожном покрытии - линейных износов \cite{h102}. Термин <<линейный изноз>> впервые был предложен в работе \cite{20wl}. На рис.\ref{pic2} описывается использование двумерного дискретного вейвлет-преобразования и фильтров на основе методов математической морфологии.
\begin{figure}[ht!]
\centering
\includegraphics[width=0.8\linewidth]{pic2}
\caption{Метод извлечения признаков на основе DWT--SMF (discrete wavelet transform -- successive morphologic transform filtering) \cite{h102}}
	\label{pic2}
	\end{figure}

Метод на основе нечеткой логики применяется для извлечения признаков дефектов дорожного покрытия - глубоких трещин. На рисунке (\ref{pic3}a) показано начальное изображение, на рисунке (\ref{pic3}b) показаны результаты сегментации. Рисунки (\ref{pic3}c, d) отображают результаты извлечения конкретных краев.
\begin{figure}[ht!]
\centering
\includegraphics[width=0.6\linewidth]{pic3}
\caption{Результаты извлечения признаков «нечётких» дефектов дорожного покрытия \cite{h102}}
	\label{pic3}
	\end{figure}
	
В работах \cite{h79, h80, h81, h82} применены методы обработки изображений для удаления фоновых изображений, но результаты по-прежнему содержат области без дефектов. Методы машинного обучения используются для отделения дефектов дорожного покрытия от этих областей. В таких исследованиях использовались различные признаки: толщины, площади и пластичности поверхности являются признаками, которые широко используются для выявления интересующих областей.
	
Однако на практике извлечение признаков для обработки является интуитивным и основано на наблюдениях человека. Для создания максимально информативных признаков необходимо уметь оценивать важность их отдельных компонент.
%------------1.2.3-------------%
\subsection{Методы машинного обучения в обнаружении и классификации дефектов дорожного покрытия}
Методы машинного обучения были разработаны и широко применяются для решения задачи обнаружения и классификации дефектов дорожного покрытия. Методы машинного обучения оценивались на основе критериев: времени выполнения, стабильности системы для несбалансированных данных (например, наличие редких типов дефектов), стабильности системы для косвенных факторов (шум, свет и т.д.), способности интерпретировать результаты и ясности процедур (т.е. минимум параметров для настройки).
\begin{figure}[ht!]
\centering
\includegraphics[width=0.8\linewidth]{pic4}
\caption{Общая структура системы обнаружения и классификации дефектов дорожного покрытия.}
	\label{pic4}
	\end{figure}
	
	Методы и алгоритмы машинного обучения были успешно применены и разработаны при автоматическом обнаружении и классификации дефектов дорожного покрытия и имеют общую структуру, как показано на рисунке \ref{pic4}. Эти алгоритмы перечислены в таблице \ref{tab2}. Рассмотрим их более подробно.
	
	 \begin{table}[h!]%
\centering
\caption{Анализ алгоритмов машинного обучения (****: Лучший, *: Худший).}
\label{tab2}
  \begin{tabular}{|c|c|c|c|c|}
    \hline
     \multirow {3}{*}                 &{Деревья} & {Нейронные} & {Наивный}  & {}МОВ\\
		                                  & {решений}& {сети}      & {байесовский} &{}\\
																		 	&&&{классификатор}&\\
    \hline
\multirow {2}{*}{Тип машинного}&{Обучение} 	     &{Обучение} 	&{Обучение}    &{Обучение} \\
                {обучения}    & {без учителя}    &{с учителем} &{с учителем} &{с учителем}\\
\hline 
Точность                    &**   &***  &*  &****\\
\hline 
\multirow {2}{*}{Скорость}  &{***}  &{*}    &{****} &{*}\\
                 {обучения} & {}    &{}     &{}     &{}\\
\hline 
\multirow {2}{*}{Классификация}      &{****}	&{****}	&{****}	&{****} \\
                {скорости }          & {}    &{}     &{}     &{}\\
\hline 
\multirow {2}{*}{Среднеквадратическое}  &{**}   &{**}	  &{***}	&{**} \\
                {отклонение с шумом}    & {}    &{}     &{}     &{}\\
\hline
\multirow {2}{*}{Улучшение процесса}    &{**} &{***} & {****} & {**}\\
\multirow {1}{*}{обучения}            & {}    &{}     &{}     &{}\\
\hline
  \end{tabular}
\end{table}%\vspace{10mm}

\textbf{Дерево решений}. Деревом решений является древовидная структура, в которой каждый узел представляет соответствующий признак. Каждая ветвь представляет результат теста. Узлы обозначают классы или распределения классов \cite{h83}. Чтобы классифицировать неизвестный образец, значения признака этого образца меняются на дерево решений.

Дерево решений можно трансформировать в набор правил классификации \cite{h84, h85}. Математическая основа дерева решений - это <<жадный алгоритм>>, который построил реверсивное дерево решений сверху вниз. Преимущество этого метода заключается в том, что его легко реализовать, легко понять и легко выполнить.

\textbf{Нейронные сети}. Искусственные нейронные сети (ANN) организованы в слои, каждый из которых состоит из взаимосвязанных узлов, содержащих функцию активации\cite{h86}. ANN создают систему обработки информации, которая имитирует нейронную систему мозга человека. Обработка информации в нервной системе состоит из двух частей: обработка входного сигнала и вывод выходного сигнала. Эти два класса взаимодействуют друг с другом через один или несколько скрытых слоев. Элементы в разных классах взвешены. Весовое значение может быть исправлено путем обучения правилам через входные значения, которые  используются \cite{h87, h88}. 

Преимущество этого подхода состоит в том, что он является мощным методом моделирования, который был успешно применен для классификации дефектов дорожного покрытия \cite{h92, h93}.

\textbf{Наивный Байесовский классификатор}. Байесовский метод классификации основан на статистических методах. Этот метод может прогнозировать вероятности классов в наборе данных на основе этой вероятности, которые могут быть помещены в отдельные классы \cite{h89, h90}. Байесовская сеть представляет собой график на графике, который позволяет выражать отношения между признаками.

\textbf{Метод опорных векторов - МОВ} \cite{h91} был впервые предложен Вапником в 1960 году для классификации данных и привлекл большой интерес. МОВ - очень общий метод, который может быть применен к широкому спектру проблем идентификации дефектов дорожного покрытия и классификации \cite{h94}. Целью метода МОВ является создание модели из набора моделей, которые имеют возможность прогнозировать классы для объектов.

Преимущества МОВ: очень эффективен при решении больших размерных данных (анализ экспрессии генов, белков, данных клеток), МОВ позволяет решать проблему переобучения очень хорошо (при наличии шума в данных или когда данных для обучения слишком мало), является быстрым методом классификации, обладает хорошей совокупной производительностью и высокой вычислительной эффективностью.

Существуют различные методы обнаружения дефектов дорожного покрытия на основе характеристик цвета и яркости \cite{h66}, анализа пороговых значений \cite{h67}, признаков структурных характеристик изображений \cite{h68}. Имеется много разных подходов для обнаружения дефектов дорожного покрытия. Один из самых распространенных подходов осуществляется путем анализа гистограмм с использованием искусственных нейронных сетей (ИНС). 

В работе \cite{h69} авторы для классификации сегментов дорожных изображений и трещин предлагают методику, основанную на нейронной сети. Проводится анализ гистограмм изображений. Признаки передаются в нейронную сеть для последующей классификации. После того как изображения классифицируются на наличие трещин, участки, не имеющие трещин, после их сегментации передаются в другую нейронную сеть для классификации типа трещин. Модели на основе теории графов широко используется для сегментации, что отражено, например, в работе \cite{h70}. В работе \cite{h71} авторы предложили объединить методы математической морфологии и преобразования Фурье для создания признаков, которые были классифицированы на основе классификатора AdaBoost \cite{h72}.

В статье \cite{h73} авторы обсудили проблемы автоматического анализа видео, чтобы следить за состоянием дорожного покрытия в лабораторных условиях и показали, что использование компьютерного зрения для решения задач анализа изображений позволяет сократить время внедрения и дает более высокую точность.

Был предложен способ обнаружения поверхностных контуров, который не зависит от теней, освещения и неровностей. Этот метод основан на обработке изображений с использованием локальных особенностей градиента, линейного предсказания \cite{h74} и анализа градиентов изображений \cite{h75}. 

В работе \cite{h76} Лемперт, Сидоров и Жуков представили подход к проблеме определения приоритетов работ по ремонту дорожного покрытия с ограниченными ресурсами, который использует комбинацию методов распознавания и классификации дефектов на основе статистического анализа и машинного обучения с оригинальными методами для решения задачи оптимизации и определения приоритетов работ (оптическая – геометрическая аналогия).

Авторы в работах \cite{h77, h78} предложили два новых метода для регистрации дефектов дорожного покрытия. Подходы основаны на видеосистеме, представляющей собой автомобиль, на котором установлен комплекс для сбора и анализа данных дорожной поверхности. Используется метод сегментации изображений, затем дефекты классифицируются с помощью определенного классификатора. 

В статье \cite{ h103} автор создал систему, состоящую из трех основных частей (рис. \ref{pic5}): предварительная обработка и улучшение качества цифровых изображений, извлечение геометрических объектов в каждой области изображения, обнаружение и идентификация дефектов на основе вейвлет-преобразований для изображений с низким разрешением, использование срединных фильтров для получения пороговых значений, использование морфологических фильтров для признаков формы, использование случайных величин для классификации типов дефектов дорожного покрытия.   
\begin{figure}[ht!]
\centering
\includegraphics[width=0.8\linewidth]{pic5}
\caption{Блок-схема подхода к обнаружению и идентификации линейных дефектов в дорожных покрытиях на цифровых изображениях \cite{h103}.}
	\label{pic5}
	\end{figure}

В \cite{h110} подход машинного обучения использовался для обнаружения и классификации дефектов выборок на видео (рис.\ref{pic6}). Изображения сегментируются в регионы, извлекают формы и текстурные признаки для обнаружения выбоин. Геометрические признаки количественно определяются эллиптической регрессией.
\begin{figure}[ht!]
\centering
\includegraphics[width=0.8\linewidth]{pic6}
\caption{Описание основных частей системы обнаружения дефектов выбоин на видео \cite{h110}.}
	\label{pic6}
	\end{figure}

В \cite{h104} представлена система автоматического обнаружения и классификации дефектов дорожного покрытия на основе комбинации вейвлет-преобразования и нейронной сети (рис.\ref{pic7}). Система состоит из трех основных частей: первая часть - обработка и сохранение входного изображения, вторая часть - извлечение признаков на основе Фурье преобразования и балансе гистограммы, классификация дефектов с использованием нейронных сетей. Третья часть - отображение результатов классификации.
\begin{figure}[ht!]
\centering
\includegraphics[width=0.8\linewidth]{pic7}
\caption{Система автоматически классифицирует и обнаруживает глубокие трещины с использованием динамических нейронных сетей \cite{h104}.}
	\label{pic7}
	\end{figure}
%-----------1.2.4===============%
\subsection{Программы обнаружения и классификации дефектов дорожного покрытия на основе методов машинного обучения}
В течение многих лет были разработаны программы автоматического обнаружения и классификации дефектов дорожного покрытия. Эти данные собираются и сохраняются \cite{h95, h96}. Обзор доступных источников продемонстировал недостатки существующих систем в первую очередь связанные с их высокой стоимостью.

В последние годы получили широкое распространение цифровые системы. Такие системы обрабатывают видеопоток достаточно высокого разрешения. На основании этого можно сделать вывод о необходмости привлечения использования алгоритмов машинного обучения.

Существует множество систем для автоматического обнаружения трещин с использованием линейных сканеров. Предлагаемые системы проводят классификацию согласно ширине дефекта.

В 1999 году Организация научных и промышленных исследований Австралийского содружества (CSIRO) первой разработала систему автоматического обнаружения дефектов трещин - RoadCrack \cite{h18}. 

Платформа ARAN \cite{h97} широко используется в Соединенных Штатах для автоматического анализа дорожных покрытий. Эта система проводит сбор данных, где изображения оцениваются с помощью автоматизированного программного обеспечения для обнаружения трещин - WiseCrax \cite{h19}. Также популярна программа автоматического обнаружения транспортных средств WayLink Digital Highway Data Vehicle \cite{h98}.

В европейских странах система PAVUE \cite{h99} нашла применение в Нидерландах и Финляндии. Система оснащена двумя видеокамерами для сбора данных и настройки компьютера, которые объединяют алгоритмы машинного обучения.

Система использует высокоскоростную технологию обработки, объединяющая устройство для сбора и хранения данных (рис.\ref{pic8}). Это устройство включает в себя цифровые камеры, ультразвуковые и лазерные технологии для сбора структурных дефектов на дороге.
\begin{figure}[ht!]
\centering
\includegraphics[width=0.5\linewidth]{pic8}
\caption{Устройства сбора изображений дорожного покрытия \cite{h109}.}
	\label{pic8}
	\end{figure}
	
Система LRIS состоит из камер высокого разрешения и лазерных источников (рис. \ref{pic9}). Эти камеры установлены на оборудованных для сбора данных автомобилях для и классификации дефектов дорожного покрытия. Данные считываются в автономном режиме с помощью автоматической программы обнаружения трещин \cite{h16}. Преимуществами системы являются быстрота и безопасность.
\begin{figure}[ht!]
\centering
\includegraphics[width=0.5\linewidth]{pic9}
\caption{Система IRIS \cite{h16}.}
	\label{pic9}
	\end{figure}
	
Система GIE LaserVISION \cite{h100} является ярким примером использования лазерной технологии для автоматического обнаружения и классификации дефектов дорожного покрытия. Система использует четыре лазерных датчика и обеспечивает 3D-вычисления для улучшения расчета дефектов в 3D. Однако, разрешение системы низкое, и работает только с горизонтальными трещинами. Основные проблемы обработки видеоданных подобными системами связаны с освещением, углом регистрации изображений дорожного полотна и др. Чтобы уменьшить влияние этих факторов, система StereoVision \cite{h101, h106, h107} использует 3D-технологию для повышения производительности системы.
%%----------1.2.5----------------------%
\subsection{Комбинация метода Марковского случайного поля с методом разреза на графах для сегментации изображений}
Между соседними пикселями в изображениях всегда существует определенная связь. Поэтому статистические модели хорошо подходят для представления этих взаимодействий. В последнее время исследования по решению проблемы сегментации изображений были сфокусированы на модели Марковского случайного поля (MRF) \cite{h111, h112}. В этом разделе анализируется подход к решению проблемы сегментации изображения на основе модели MRF.

Марковское случайное поле можно рассматривать как частный случай случайного поля \cite{h113}, в котором состояние случайной величины зависит только от состояния «смежных» случайных величин. Преимуществами модели MRF являются следующие: В основе MRF лежит хорошо проработанная теория; алгоритм не требует предположения независимости наблюдаемых переменных. Кроме того, использование произвольных факторов позволяет описать различные признаки определяемых объектов, что снижает требования к полноте и объему обучающей выборки.

Алгоритм случайных полей рассматривается как инструмент структурного анализа. Сочетание спектральных и структурных признаков приводит к лучшему анализу изображения. На основе структуры можно определить вероятность появления объекта находящегося в определеной связи с соседними пикселями.

\textbf{a) Марковские случайные поля в сегментации изображения}

Модели MRF часто используются для решения проблем сегментации изображений. В \cite{h114} автор предоставил эффективный метод сегментации изображений без учителя. Использовался иерархический генетический алгоритм для решения сложной вычислительной задачи MRF с использованием моделей. 

В \cite{h115} также был предложен подход к сегментации изображения на основе MRF и разделения исходного изображения на независимые области в виде смежных графов. Для этого были определены надежные признаки и их интеграция в энергетическую функцию, которая управляет процесс. В \cite{h116} предлагается подход, основанный на алгоритме сегментации неэкранированного изображения и модели MRF. Этот метод решения проблем шума и текстуры изображения. Количество классов и параметров модели задано в соответствии с каждым стандартным сегментом изображения \cite{h117}. Результаты демонстрируют улучшение эффективности и надежности.

В \cite{h118} автор представляет два метода сегментации изображений: с учителем и без учителя. Предлагаемый алгоритм может обеспечить максимальный метод с учителем сегментации изображения. В сегментации изображения без учителя была предложена схема параметрической оценки для непосредственного вычисления параметров модели из данного изображения.

Многие другие исследования использовали модель MRF для решения проблем сегментации изображений \cite{h119}. В \cite{h120} предложен алгоритм для сегментирования изображений на основе текстуры изображений с использованием алгоритма вейвлет и MRF. Результаты представлены в \cite{h121, h122} с использованием модели MRF для сегментирования изображений дистанционного зондирования для получения точных и эффективных результатов.

\textbf{b)	Объединение метода разреза на графах с методом MRF}

Метод MRF используется в качестве модели для решения пометок при обработке изображений. Этот метод широко используется для моделирования проблем, таких как восстановление изображений, сегментация изображения, структурный анализ и т. д. Метод разреза на графах считается эффективным способом решения проблемы минимизации энергии в области машинного обучения. Было проведено много исследований с использованием метода разреза на графах для решения вычислительных задач MAP-MRF.

Бойков \cite{h123} предложил новый метод решения проблемы MAP-MRF с использованием алгоритма разреза на графах. На граф был применен метод минимального потока с минимальным разрезом, который продемонстрировал важность метода разрезов графов . Этот алгоритм может решать энергетические функции с элементами MAP-MRF.

Коли и Торр \cite{h124} представили абсолютно новый динамический алгоритм для проблемы <<st-минимального разреза>>, который может быть использован для быстрого поиска решений MAP для некоторых динамически изменяющихся MRF. Этот метод является общим и находит точное решение для всех динамических задач, которые могут быть сформулированы как энергетические функции двоичных переменных. Результаты показали, что их алгоритм существенно быстрее, чем самый известный статический алгоритм.

Алгоритм разреза графов широко используется в области компьютерного зрения. Этот алгоритм используется для сегментации медицинских изображений или сегментации видео \cite{h125}. Авторы Бойков и Джолли \cite{h126} первыми предложили и протестировали алгоритм двоичного разреза графов для сегментации объектов.

В работе \cite{h127} существует несколько алгоритмов для тестирования и применения в сегментации двоичного изображения. Бойков и Функалед предложили методы извлечения признаков объектов с помощью разрезов на графах \cite{h128}. Этот метод может быть применен к интересующей предметной области изображения. Алгоритм <<минимального разреза -- максимального потока>> был протестирован в сегментации 2D и 3D-изображений. Результаты демонстрируют эффективность метода и применения алгоритма разрезов на графах в сегменте изображения.

%%----------1.2.6----------------------%
\section{Алгоритм случайного леса в задаче классификации данных}
Алгоритм <<случайный лес>> - это классификатор объектов. Этот метод разработан Лео Брейманом в Калифорнийском университете в Беркли \cite{h129}. Брейман также является соавтором методологии классификации и регрессии (CART), которая считается одной из 10 лучших классических методов интеллектуального анализа данных. Случайный лес построен на трех основных компонентах: CART, комбинированной модели и синтезе бутстрепа.
Процесс изучения случайный лес предполагает использование случайных входных значений или комбинаций этих значений в каждом узле для построения дерева решений. Алгоритм случайный лес имеет некоторые сильные атрибуты, такие как: 

\begin{itemize}
	\item Его точность похожа на Adaboost, в некоторых случаях превосходит его. 
\item Этот алгоритм правильно решается с данными, которые имеют большой шум. 
\item Время работы алгоритма быстрее, чем Бэггинг или Бустинг. 
\item Легко реализуется параллельно.
\end{itemize}

Однако для достижения вышеуказанных свойств время выполнения алгоритма достаточно длительное и необходимо использовать много ресурсов системы. Таким образом, можно сказать, что алгоритм случайный лес является хорошим методом классификации в виду следующих причин: в алгоритме случайный лес дисперсия минимизируется в результате синтеза случайного леса во многих процессах обучения; выбор случайных значений на каждом шаге в случайном лесе уменьшит корреляцию между изучаемыми в синтезе результатов.

Кроме того, общая ошибка подкласса леса зависит от ошибки отдельных деревьев в лесу, а также от каждой взаимосвязи между деревьями.

\textbf{Оценить ошибку OOB алгоритма случайный лес.} Образец был составлен из учебного набора, было подсчитано, что около $1/3$ элементов не были включены в образец. Это означает, что изучались только около $2/3$ элементов, участвующих в расчете $1/3$ , получены данные об общей сумме. Данные об общей сумме используются для оценки ошибки результатов и оценки важности каждого признака. Реализация вычисления для определения важных атрибутов в алгоритме случайного леса такая же, как и использование OOB для вычисления ошибок в алгоритме случайного леса. 

%--------1.3---------------------%
\section{Анализ подхода к использованию машинного обучения при обнаружении пузырьков}
В этом разделе представлены работы по исследованию и применению методов машинного обучения в проблеме обнаружения пузырьков. Кроме того, были успешно построены оценки программ, они демонстрируют свое приложение на практике. Учитывалась возможность применять Wavelet Transform в пузырьках как функцию.
%%-------1.3.1------------------%
\subsection{Методы обнаружения пузырьков}
Во время анализа изображения пузырьков методы машинного обучения включают в себя два основных процесса, которые представляют собой процесс сегментации изображения для обнаружения местоположений пузырьков и идентификации их формы (одиночный, перекрывающий, частично стирающийся), а затем процесс извлечения признаков, таких как радиус, диаметр, площадь, координаты центра и т.д. от экспериментальных данных.

В работе \cite{h1bb} представлен метод идентификации изображений пузырьков на основе алгоритма сверточных нейронных сетей \cite{h6bb}. Нейронные сети способны определять перекрывающиеся, размытые и несферические пузырьковые изображения. Моделируются реалистичные пузырьковые изображения из экспериментальных данных для создания синтетических изображений, необходимых для обучения нейронных сетей. Это повысило точность распознавания изображений пузырьков, уменьшило количество выбросов, снизило время работы. Предложен метод обнаружения пузырьков прозрачных объектов в жидкости. Авторы сформулировали проблему обнаружения концентрических круговых устройств. Генерация гипотезы основана на выборке из подключенных компонентов ответов подавления ориентированных хребтовых фильтров и оценки параметров концентрических круговых устройств. Предложен способ обнаружения пузырьков, который показал удовлетворительную производительность в промышленном применении, требующий оценки объема газа в суспензии целлюлозы при достижении средней относительной погрешности \cite{h2bb}.

В работе \cite{h3bb} описывается новый признак, который был разработан для автоматического распознавания нефти из других круглых объектов, найденных на изображениях.

В статье \cite{h4bb} предложена система измерения пузырьков, в которой используется метод обнаружения на основе шаблонов. Предлагаемый подход говорит о слабой эффективности традиционных методов для обнаружения пузырьков. В предлагаемом подходе используются шаблоны для повышения надежности и масштабирования изображения для обнаружения пузырьков независимо от их размера.

В статье \cite{h5bb} авторы предоставили метод сегметации пузырьков потока газа $/$ жидкости двухфазного потока на основе оператора Canny \cite{h7bb} и гауссовского гладкого фильтра \cite{h8bb} многопоточного высокоскоростного видеоанализа для сохранения краев и устранения шума. Этот метод очень эффективен для анализа и распознавания многопоточного потока пузырьков.

В статье \cite{h9bb} описана новая полуавтоматическая методика онлайн-оценки диаметров пузырьков нефти и воздушных пузырьков. Изображения данных были предварительно обработаны, чтобы найти края сегментов областей, представляющих интерес. Алгоритм преобразования Хафа \cite{h10bb} используется для восстановления контуров воздушных пузырьков и  пузырьков нефти. Этот метод позволил сократить общее время обработки измерения пузырьков в другой системе.

В работе \cite{h11bb} представлена проблема важности захвата изображений и сегментации, включая гетерогенную прозрачность движущихся объектов интереса и фона, размытие, перекрытие и артефакты на основе метода преобразования Хафа, который был реализован и протестирован. В статье также сделан вывод, что оценка распределения размеров воздушных пузырьков при механическом перемешивании проводилась более эффективным и менее трудоемким способом, чем другие полуавтоматические или ручные методы. 

В \cite{h12bb} предлагается метод перекрывающегося пузырькового расщепления. Во-первых, получается выпуклая оболочка перекрывающегося объекта. Во-вторых, пересечение основано на контуре объекта в каждом пузырьке после повторного поиска, а затем точечные пары перекрывающихся пузырьков получаются путем сопоставления на основе средней оси (МА). Наконец, точность вычисления области пересечения пузырьков обеспечивается путем построения эллипса на основе значений точек пересечения при использовании минимальной среднеквадратической ошибке (MMSE) с ограничениями пересечения.
%---------------1.3.2 ---------------%
\subsection{Программы обнаружения пузырьков}
В настоящее время существует множество проектов, разработавших системы обнаружения пузырьков на основе результатов исследований, например: использование двухфазных пузырьковых потоков во многих технологических и энергетических процессах в качестве технологических, химических и ядерных реакторов. Это объясняет большой интерес к экспериментальным и численным исследованиям таких потоков за последние несколько десятилетий. Использование оптической диагностики для анализа пузырьковых потоков позволяет исследователям получать мгновенные поля скоростей и распределение газовой фазы с высоким пространственным разрешением. Эта работа представляет собой метод идентификации изображений пузырьков, основанный на современной технологии глубокого обучения, называемой сверточными нейронными сетями (CNN). Система обнаружения пузырьков воды \cite{h1bb} основана на новой SIFT \cite{h13bb} и гистограмме метода ориентированных градиентов (HoG) \cite{h14bb}.
\begin{figure}[ht!]
\centering
\includegraphics[width=0.3\linewidth]{pic10}
\caption{Система обнаружения пузырьков воды на основе метода CNN.}
	\label{pic10}
	\end{figure}

Нейронные сети способны определять перекрывающиеся, размытые и несферические изображения пузырьков. Система может повысить точность распознавания изображений пузырьков, уменьшить количество выбросов, сократить время обработки данных и значительно уменьшить количество настроек для идентификации объекта по сравнению со стандартными методами распознавания, разработанными ранее (рис. \ref{pic10}). Кроме того, использование графических процессоров ускоряет процесс обучения CNN, владеющего современными адаптивными методами оптимизации субградиента.

\begin{figure}[ht!]
\centering
\includegraphics[width=0.5\linewidth]{pic11}
\caption{Система обнаружения пузырьков на основе методов анализа изображений.}
	\label{pic11}
	\end{figure}
	
В этой работе \cite{h15bb} было предложено использовать машинное зрение в процессе массового распознавания пузырьков, что подтверждает валидацию моделей кипения, связанных с динамикой пузырьков, а также частоту зарождения, плотность активного слоя и размер пузырьков (рис. \ref{pic11}). Два предложенных алгоритма предназначены для получения стандартных изображений пузырьков, наблюдающихся при барботаже в кипятильных устройствах общего назначения.

\begin{figure}[ht!]
\centering
\includegraphics[width=0.7\linewidth]{pic12}
\caption{Система обнаружения пузырьков на основе технологии 3D-анализа изображений.}
	\label{pic12}
	\end{figure}
	
В этой статье \cite{h16bb} авторы представляют  широкую базовую стереопанельную камеру с широким разрешением, которая преодолевает многие ограничения, поскольку она наблюдает пузыри в двух ортогональных направлениях, используя калибровочные камеры. Помимо описания установки и аппаратного обеспечения системы, в ходе исследования обсуждаются соответствующие калибровки, а также различные автоматические этапы обработки деблокирования, обнаружения, отслеживания и 3D-фитинга, которые имеют решающее значение для получения трехмерной эллипсоидальной формы и повышения скорости каждого пузыря. Полученные значения для одиночных пузырьков могут быть агрегированы в статистические распределения размеров пузырьков или потоки для экстраполяции на основе моделей диффузии и растворения и крупномасштабных акустических съемках. Результаты исследования демонстрируют и дают оценку широкой базовой модели стереоизмерений с использованием контролируемой тестовой установки с информацией о входных данных.

%---------1.3.3 -----------------------------
\subsection{Признак вейвлет-преобразования при обнаружении объектов}
Научное направление сконцентрировано на анализе и применении, связанное с так называемым вейвлетами. Вейвлеты широко применяются  при решении ряда задач, таких как: сжатие и обработка изображений, распознавание образов, обработка и синтез сигналов и т.д. Вейвлеты могут быть ортогональными, полуортогональными и биортогональными. Вейвлет-коэффициенты определяются интегральным преобразованием сигнала \cite{1wl}.

Вейвлет-преобразование похоже на преобразование Фурье, но с совершенно иной оценочной функцией. Основное различие лежит в следующем: вейвлет-преобразование использует функции, локализованные как в реальном, так и в Фурье-пространстве. В общем, вейвлет-преобразование (wavelet transform) является инструментом, разбивающим данные, функции, операторы на составляющие с разными частотами, каждая из которых затем изучается с разрешением, подходящим масштабу \cite{2wl}.

Вейвлет-преобразование разделяется на непрерывное вейвлет-преобразование (НВП) \cite{3wl} и дискретное вейвлет-преобразование (ДВП) \cite{1wl}.

В машинном зрении преобразование Вавелет широко используется в качестве метода обработки изображений для обнаружения и классификации объектов.

В работе \cite{5wl} вейвлеты использовались для анализа изображений и использовались во многих приложениях дистанционного зондирования, таких как слияние изображений с высоким спектральным разрешением с изображениями с высоким пространственным разрешением, анализ текстур и классификация \cite{6wl}, удаление шума из радарных изображений.

Вейвлет-преобразование эффективно используется для удавления шумов с изображений. Изображение претерпевает вейвлет-преобразование, фильтрацию и обратное вейвлет-преобразование (рис.\ref{pic13}) \cite{8wl}.
\begin{figure}[ht!]
\centering
\includegraphics[width=0.8\linewidth]{pic13}
\caption{Использование вейвлет-преобразования при фильтрации шума.}
	\label{pic13}
	\end{figure}
В работе \cite{9wl} вейвлет-преобразование использовалось для классификации сигнала ЭЭГ с интеграцией экспертной модели.

В статье \cite{10wl} вейвлет-преобразование применялось для классификации заземления. Реализация дискретного преобразования вейвлета (ДВП) в качестве метода обработки изображений дает значения преобразования, называемые коэффициентом вейвлет-коэффициента. Задачи представляют собой определение объекта для обнаружения и классификации для  определения коэффициента.

Подход к извлечению признаков, основанный на вейвлет-преобразовании, представляет собой вычисление распределения коэффициентов по выбранному признаку вейвлета. Общей методикой, используемой для выделения признаков ДВП-коэффициента, является использование нейронных сетей \cite{11wl, 12wl, 13wl}.

Алгоритм преобразования вейвлетов представлен в работе \cite{14wl}. Вейвлет-преобразование используется с определенной шкалой разложенного сигнала измерения расхода. Из детального коэффициента вычисляется адаптивный порог и проверяется на подробные коэффициенты. Газовые пузырьки обнаруживаются, если подробные коэффициенты превышают порог в конкретный момент времени.

%%==========1.4----------------------%
\section{Основные результаты и выводы по главе 1}

\begin{enumerate}
	\item В содержании главы 1 было рассмотрено машинное обучение, классификатор и применение машинного обучения на практике, комбинированные методами методы классификации данных. Рассматривались некоторые проблемы в оценке ошибки метода классификации методов машинного обучения.
	\item В первой главе были проанализированы методы обработки изображений в извлечении признаков для двух задач ОКДДП и ОКФП.
\item  В этой главе анализировался подход алгоритмов и методов машинного обучения в классификации данных. Были освещены следующие алгоритмы, методы:

\begin{itemize}
	 \item Дерево решений
   \item Нейронные сети 
   \item Наивный байесовский классификатор
   \item Метод опорных векторов – МОВ
\end{itemize}

\item Рассмотрена возможность использования алгоритма Марковского случайного поля в сочетании с методом разрезов на графах для построения карты дефектов дорожного покрытия и алгоритма случайного леса для классификации данных дефектов дорожного покрытия на изображениях и видео.
\item Рассмотрена возможность использования вейвлет-преобразования в качестве признака для обнаружения пузырьков на фотографиях.
\end{enumerate}
В этом исследовании реализована комбинация методов обработки изображений для данных процесса в условиях нормального освещения и большого шума. Реализованы методы и алгоритмы машинного обучения для решения основной задачи: извлечения признаков, обнаружения и классификации объектов в фотографиях и видео, обеспечена стабильность системы, получена хорошая производительность, обработка данных выполнялась быстро и точно.

Целью этой диссертации является разработка алгоритмов, методов машинного обучения для создания автоматизированной системы для извлечения признаков, обнаружения и классификации объектов в изображениях и видео. Для достижения указанных выше целей необходимо решить следующие задачи:

\begin{itemize}
	\item Создать и разработать методы обработки данных на изображениях и видео.
\item	Проанализировать и улучшить метод обнаружения объектов в соответствии со сменой света для построения карты дефектов дорожного покрытия, чтобы определить границу между объектом для пузырьков воздуха.
\item	Применить алгоритмы, методы обработки изображений для извлечения признаков объектов при наличии шума.
\item	Разработать и внедрить обнаружение и классификацию дефектов дорожного покрытия на основе методов, алгоритмов машинного обучения.
\item	Разработать и внедрить обнаружение и расчет диаметров баллона с использованием функции вейвлет-преобразования .
\item	Создать и оценить точность функций автоматической системы обнаружения, классификации дефектов дорожного покрытия и систем обнаружения воздушных пузырьков.

\end{itemize}

%%%%%%%%%%%%%%%%%%%%%%%  ГЛАВА 2 %%%%%%%%%%%%%%%%%%%%%%%%%%%%%%%%%%%%%%%%%
\chapter{Математические модели и численные методы для решения задачи классификации объектов на основе методов компьютерного зрения и машинного обучения} \label{chapt2}
В этой главе представлены методы математического моделирования и численные методы решения поставленных задач, используя методы машинного обучения. Работа исследований концентрируется на следующих задачах: предварительная обработка изображений, улучшение качества входных данных с помощью алгоритмов обработки изображений, методы машинного обучения для обнаружения и классификации объектов. Предложенная система состоит из четырех основных этапов (рис.\ref{pic14}).
\begin{figure}[ht!]
\centering
\includegraphics[width=0.8\linewidth]{pic14}
\caption{Основные этапы системы обнаружения и классификация на основе методов компьютерного зрения и машинного обучения.}
	\label{pic14}
	\end{figure}
%%--------2.1-------------%
\section{Предварительная обработка изображения и извлечение признаков в системах ОКДДП и ОКФП}
\subsection{Предварительная обработка изображения в системах ОКДДП и ОКФП}
Процесс обработки изображений считается основой системы. Система должна иметь возможность работать с различным разрешением исходных изображений и работать в режиме времени близкого к реальному. Поэтому процесс обработки изображений должен быть изучен и оптимизирован для того, чтобы система производила качественные результаты, но обеспечивала высокую производительность системы.

Основываясь на выборе методов предварительной обработки, результаты обнаружения и классификации системы значительно улучшены, что увеличивает скорость обработки всей системы. Этот процесс включает в себя фильтрацию шума и получение плавного изображения, выявление признаков объектов на изображении, а затем преобразование изображения в полутоновое изображение (рис. \ref{pic15}). В данной работе использован фильтр Гаусса для удаления шума. Сущность этого преобразования состоит в реализации свертки изображения с ядром симметричной формы в виде 2-D функции Гаусса. Эта непрерывная функция определяется следующим образом:
\begin{equation}\label{eq1}
 f\left(x, y\right) = \frac{1}{2\pi\sigma^2} \exp \left(-\frac{x^2+y^2}{2\sigma^2}\right)
\end{equation}

\begin{figure}[ht!]
\centering
\includegraphics[width=0.5\linewidth]{pic15}
\caption{Результат преобразует изображение в полутоновое изображение (2.2.а - изображение дефектов дорожного покрытия; 2.2.б - изображение пузырьков).}
	\label{pic15}
		\end{figure}

Во время сбора объектов изображения повреждаются шумом, который влияет на обнаружение и извлечение признаков объектов.

Кроме того, условия освещения могут изменять различные области изображения, что приводит к их деградации. Эти проблемы решаются с помощью адаптивной балансировки гистограмм (рис. \ref{pic16}).
\begin{figure}[ht!]
\centering
\includegraphics[width=0.5\linewidth]{pic16}
\caption{Результат выравнивания гистограммы.}
	\label{pic16}
		\end{figure}
		
Алгоритм эквализации гистограммы основан на простой идее и часто используется при обработке изображений. Цель эквализации гистограммы, как уже отмечалось выше, заключается в увеличении диапазона яркости изображения.

Эквализация гистограммы изображения g определяется следующим образом:
\begin{equation}\label{eq2}
g_{i,j}=floor \left(\left(L-1\right)\sum^f{i,j}_{n=0} p_n\right)
\end{equation}

\begin{itemize}
	\item $	f$  исходное изображение;
	\item $n=0,1,…,L-1;L [0 ; 255]$;
	\item $p_n$: количество пикселей по интенсивности  n / количество пикселей;
	\item $floor()$ округляется до ближайшего целого числа. Это эквивалентно преобразованию интенсивности пикселей $k \in f$ с помощью функции:
	\begin{equation}\label{eq3}
T\left(K\right)=floor\left(\left(L-1\right)\sum^f{i,j}_{n=0} p_n\right)
\end{equation}
\end{itemize}
Для высокоселективного процесса экстракции в предлагаемой работе используется морфологический метод для выявления дефектов пикселей и удаления небольших участков потенциала шума изображения на изображении (рис. \ref{pic17}). Морфологический метод анализа геометрических структур был разработан для двоичных изображений, а затем расширен до полутоновых изображений. Это один из методов, применяемых на этапе предварительной обработки. Двумя наиболее часто используемыми операциями являются Dilation и Erosion. Из этих двух основных математических операций разрабатывается ряд операций, таких как Close и Open.

\begin{equation}\label{eq4}
I\oplus H = \bigcup_{q \in H} I_q
\end{equation}

\begin{algorithm}[H]
  \KwData{Изображение $I$, структурирующий элемент $H$.}
  \KwResult{Изображение $\acute{I} = I \oplus H$}
 1. Начать с нулевого изображения $\acute{I}$\\
 2. Перебирать все $q \in H$\\
 3. Вычислить сдвинутое изображение $I_q$\\
 4. Обновить $\acute{I} = I \nu I_q $\\
\caption{Алгоритм расширения}\label{alg1}
\end{algorithm}

\begin{algorithm}[H]
  \KwData {Изображение $I$, структурирующий элемент$H$.}
\KwResult: {Изображение $\acute{I} = I \ominus H$}
 1. Начать с инверсии $\acute{I} = \bar{I}$\\
 2. Распространяться $\acute{I}$ with reflected structure element $H^*$\\
 3. Инвертировать $\acute{I}$\\
\caption{Алгоритм эрозии} \label{alg2}
\end{algorithm}

\begin{figure}[ht!]
\centering
\includegraphics[width=0.7\linewidth]{pic17}
\caption{Результаты морфологических операций.}
	\label{pic17}
		\end{figure}
		
\subsection{Извлечение признаков в системах ОКДДП и ОКФП}

До извлечения признаков предлагется предварительная обработка изображений. Сначала  применяется фильтрация шума с использованием фильтра Гаусса и конвертирование в полутоновые изображения. На следующем шаге выполняется сегментация изображения. Предложено разделение изображения на отдельные области. Затем отделяются пиксели объектов изображения для обнаружения связной области. Используется морфологический метод для обнаружения пикселей соответствующих объектов и для удаления небольших областей, которые определены как шум. 

\textbf{а) Извлечение признаков в системе ОКДДП}.

Рассмотрим следующие дефекты:
 
\textit{Сеть трещин} (рис. \ref{pic18}): Взаимосвязанные трещины, образующие серии блоков приблизительно прямоугольной формы, обычно распределены по всей поверхности дорожного покрытия. Атрибутами блока дефекта трещин  являются: ширина преобладающей трещины $\left(mm\right)$, ширина преобладающей ячейки $\left(mm\right)$, пострадавшая площадь $\left(m^2\right)$.
 
\begin{figure}[ht!]
\centering
\includegraphics[width=0.5\linewidth]{pic18}
\caption{Описание сети трещин.}
	\label{pic18}
		\end{figure}
		
\textit{Глубокие трещины} (рис.\ref{pic19}): Неприсоединенная трещина в продольном направлении вдоль дорожноого покрытия. Атрибуты дефекта продольных трещин : ширина трещин $(mm)$, длина трещин $(m)$, интервал трещин $(mm)$, площадь пострадавших трещин $(m^2)$.

\begin{figure}[ht!]
\centering
\includegraphics[width=0.5\linewidth]{pic19}
\caption{Описание глубоких трещин.}
	\label{pic19}
		\end{figure} 
		
\textit{Выбоины } (рис. \ref{pic20}): углубления некруглой формы различных размеров в дорожном покрытии. Атрибутами дефекта выбоин  являются глубина выбоин $(mm)$ и площадь выбоин $(m^2)$.

\begin{figure}[ht!]
\centering
\includegraphics[width=0.5\linewidth]{pic20}
\caption{Описание выбоины.}
	\label{pic20}
		\end{figure} 

Из анализа атрибутов каждого дефекта, мы выбрали следующие признаки:

\textit{Hu-моменты:} Наиболее заметны Hu-моменты, которые могут быть использованы для описания, характеристики, и определения формы объекта в изображении. Hu-моменты, как правило, извлекаются из формы объекта в изображении \cite{h130}. Описывается форма объекта и извлекается вектор признаков формы (т.е. список чисел) для представления формы объекта. Затем проводится сравниние двух векторов признаков с использованием теории подобия или расстояния функции, чтобы определить, насколько формы анологичны.

\textit{Цепной код гистограммы}: Код цепи гистограммы (CCH-chain code histogram) предназначен для группировки объектов, с исользованием методики, анологичной  наблюдением человека\cite{h131}. Он не предназначен для точного обнаружения и классификации задач. Цепной код гистограммы вычисляется из кодовой цепочки представленного контура.

Код цепи Фримэн \cite{h132} представляет собой компактный способ представления контура объекта. Код цепи представляет собой упорядоченную последовательность $n$ связывает $\left\{c_i, i=1,2,.., n\right\}$, где $c_i$ вектор, соединяющий соседние пиксели контура. Направления $c_i$ кодируются с помощью целых значений $k=0,1,...,K-1$ в направлении против часовой стрелки, начиная от направления положительного $x-axis$. Число направлений $k$ принимает целые значения $2^{M+1}$, где m является положительным целым числом. Цепные коды, где $K>8$, называются обобщенными кодами цепей \cite{h133}.

Расчет кода цепи гистограммы производится быстро и просто. Код цепи гистограммы представляет собой дискретную функцию: $p(k)=n_k/n, k=0,1,...,K-1,$ где $n_k$ является количеством значений кода цепи  $k$ в коде цепи, и $n$ это число звеньев в цепи кода. Кроме этого, рассматривается также размер дефекта области (ширина и длина, площадь) и гистограммы изображения. 

Для автоматического обозначения дефектных областей (или отсутствия дефектных областей), предложена  операционная система распознавания образов  с использованием простого пространства признаков . Пространство признаков многомерно, в этой задаче строятся 4 мерные пространства с использованием областей локальной статистики, вычисленные для нормированных и насыщенных изображений. Первыми признаками является среднее значение всех интенсивностей пикселей в области. Второе пространство признака это цепной код гистограммы, третое - Hu-момент, используемый для описания, характеристики, и определения формы объекта в изображении. В-четвертых, размер дефекта области (ширина, длина, площадь) и гистограмма изображения.
		
\textbf{б) Извлечение признаков в системе ОКФП}

В системе обнаружения и классификации признак формы пузырьков  определяется в сегментации, обнаружении и классификации процессов. Признаки описаны следующим образом:

\textit{Преобразование вейвлет} \cite{15wl}, \cite {16wl} применяется для задачи обнаружения объекта, эффективно работает для полей обнаружения мелких объектов, таких как обнаружение диафрагмы \cite{17wl}, обнаружение газового пузыря \cite{18wl} и т.д. Алгоритм вейвлет-преобразования является хорошим вариантом для оценки пузырьков воздуха в сложных полевых условиях сигналов с большим шумом. Общий подход заключается в следующем:
\begin{itemize}
	\item Исходное изображение разлагается и реконструируется в вейвлет. Оно включает в себя низкочастотное изображение и высокочастотные изображения - оно может быть объектом, которое необходимо обнаружить на изображении.
	\item Затем снижается низкочастотное изображение, а высокочастотные изображения группируются в новое изображение.
	\item Новое изображение сегментируется порогом, затем объект распознается и помечается.
\end{itemize}

Хаара-Вейвлет является одним из типов вейвлет-преобразования, который был выбран в качестве признака для извлечения. Он обладает свойствами пространственно-частотной локализации и настраиваемых деталей. 
Для вейвлета Хаара функция масштабирования$\phi\left(x\right)$ определяется как:

\begin{equation} \label{eq5}
\phi\left(x\right)= \left\{\begin{array}{l} 1, if x \in \left[0,1\right],\\
0, if x \notin \left[0,1\right],
\end{array}\right.
\end{equation}

Вейвлет-функция $\varphi\left(x\right)$ поскольку эта функция масштабирования определяется как:

\begin{equation} \label{eq6}
\varphi\left(x\right)= \left\{\begin{array}{l} 1, if x \in \left[0,0.5\right],\\
-1, if x \in \left[0.5,1\right],\\
0, if x \notin \left[0,1\right]
\end{array}\right.
\end{equation}

\textit{Признаки формы, геометрия, текстура}, такие как: яркость, обнуление значения пикселя, среднее различие между соседними пикселями, площадь, периметр, радиус, длина, ширина, длина / ширина.

%-----------------------------------------------
\section{Математическое моделирование и численные методы для сегментации изображений в системах ОКДДП и ОКФП} \label{part2}
\textbf{Математическое моделирование и численные методы для сегментации изображений}

Рассмотрим изображение $Y$ размера $M \times N$. $S = \left\{s\right\}$ - это множество всех расположений пикселя, $s \left(i, j\right)$ - позиция пикселя. Для каждого расположения $s$ существует значение $x_s=\left\{0;1\right\}$, которое определяет состояние пикселя в позиции $s$ какой-то области. Результатом сегментации являются наборы ROI (\ref{pic21}): $\bigcup\ s_{i,j} | x_{s_{i,j}}=1$.

\begin{figure}[ht!]
\centering
\includegraphics[width=0.7\linewidth]{pic21}
\caption{Описание дефектов дорожного покрытия.}
	\label{pic21}
		\end{figure} 
		
\textbf{Численный метод для сегментация изображения на основе разреза на графах и MRF}
\subsection{Численный метод для построения карты дефектов дорожного покрытия}
Обнаружение дефектов дорожного покрытия, где области изображения помечены как содержащие дефектные пиксели или не содержащие и  выявление типа классификации дефектов дорожного покрытия, где каждому выявленному дефекту дорожного покрытия присваиваются метки «блок трещины», «продольные трещины», «выбоины». Для обнаружения дефектов дорожного покрытия,необходима начальная настройка , когда оператор выбирает изображения, используемые для определения оптимального набора параметров обнаружения, приходящихся на пиксель, следующий за пикселем серой шкалы вариации, связанные с контрастом, яркостью и состоянием поверхности дефекта дорожного покрытия. На этом этапе установки, программа обеспечивает визуальную обратную связь с результатами обнаружения в виде дефектных карт основных изображений, прошедших контроль дорожных покрытий.		
	
В настоящем исследовании метод разрезов на графах применен для решения задач сегментации изображений которые имеют сложную структуру на основе анализа соотношения компонентной связности между пикселами для построения карты дефектов \cite{h144,h145}. Построение карты дефектов дорожного покрытия -- суть маркировка пикселей на изображении значением (0 - не дефектом; 1-пиксельный дефект) и выполняется в 3 шага:

\begin{itemize}
	\item \textit{Шаг 1}. Используя метод разреза на графах изображение сегментируется и извлекается область интереса (region of interest -- ROI).
\item \textit{Шаг 2}. Результат сегментации дефектов в ROI улучшается с использованием модели марковских случайных полей.
\item \textit{Шаг 3}. Построение полной карты дефектов путем сравнения и замены пиксельных меток, которые получены из шагов 1 и 2.
\end{itemize}

\textbf{Сегментация изображения алгоритмом минимального разреза -- максимального потока.} Применили метод на основе эффективного графика, чтобы найти оптимальное $D$ (дефект) - $N$ (не дефект). На рис.\ref{pic25} представлен обмен между областями $N$ и $D$ - расширение данной маркировки $f$. Предложено использование алгоритма разреза на графах, чтобы эффективно найти $f$ \cite{h13}. Использовали краткий подход. Пусть $G=\left(V,E\right)$ как взвешенный граф с двумя отмеченными вертикалями называются терминалами. Разрез $C \in E$ является множеством ребер, таких, что терминалы разделены в наведенном графе $G\left(C\right)=\left\langle V,E-C\right\rangle$. Кроме того, ни одно подмножество $С$ не разделяет терминалы в $G\left(C\right)$. Стоимость разреза $C$, обозначается $|C|$, и равна сумме его весовых значимостей.

\begin{figure}[ht!]
\centering
\includegraphics[width=0.5\linewidth]{pic25}
\caption{Использование разреза графа для улучшения сегментации изображения}
	\label{pic25}
		\end{figure} 

\begin{algorithm}[ht!]
  \KwData{ Граф $G\left(V,E\right)$ с емкостью $c$, исток $s$, сток $t$.}
  \KwResult{Максимальный поток от источника $\left(s\right)$ до стока $\left(t\right)$}
		1. \For{\texttt{$i := 1$ для всех ребер $\left(u,v\right)$ }}
     {
		$f\left(u,v\right) \leftarrow 0$
		}
	
		2. \eIf{$c_f\left(u,v\right) > 0$ для всех ребер $\left(u,v\right) \in p$}{
      $c_f\left(p\right) = min \left\{c_f\left(u,v\right)| \left(u,v\right) \in p\right\}$;
    }{
       todo 1;
    }
		3. \For{\texttt{$i := 1$ для всех ребер $\left(u,v\right)$ }}
     {
		$f\left(u,v\right) \leftarrow f\left(u,v\right) +c_f\left(p\right) $;
		}		
  \caption{Описание алгоритма разрезов на графах} \label{alg3}
\end{algorithm}

Подход на основе граф дает возможность использования эффективных решений задачи минимального разреза -- максимального потока между источником и приемником узлов ориентированных граф. Проблема минимального разреза в том, чтобы найти самый оптимальный срез среди всех разрезов, разделяющих сегменты. Минимальные разрезы могут быть эффективно найдены с помощью стандартных комбинаторных алгоритмов с различными полиномиальными сложностями низкого порядка \cite{h137}. Экспериментальные результаты были получены с помощью использования нового алгоритма минимального разреза -- максимального потока, имеющего лучшую скорость на графах для решения многих современных алгоритмов \cite{h138}. Цель заключается в сегментации изображений путем построения графа таким образом, что минимальный разрез этого графа мог вырезать все ребра, соединяющие пиксели различных объектов друг с другом (рис.\ref{pic22}).
\begin{figure}[ht!]
\centering
\includegraphics[width=0.5\linewidth]{pic22}
\caption{Сегментация изображения дефектов дорожного покрытия.}
	\label{pic22}
		\end{figure} 
		
\textbf{Оптимизация сегментации изображений на основе алгоритма Марковских случайных полей}. Для повышения качества сегментации изображений используется алгоритм Марковского случайного поля \cite{h134}. Использование SA \cite{h139, h140} для создания Марковского процесса для модели MRF, которая применяется к изображению.


\begin{itemize}
	\item \textit{Шаг 1}. Выполнить случайно M-решения для сегментов из исходного изображения.
	\item \textit{Шаг 2}. Выбрать начальное решение, называемое $X_0$. Вычислить стоимость $U \left(X_0\right)$.
	\item \textit{Шаг 3}. Повторить с оставшимся раствором $M-1$. С решением $X_i$:
	
	\begin{itemize}
		\item Вычислить стоимость $U\left(X_0\right)$.
		\item Сравнить $U\left(X_i\right)$ и $U\left(X_0\right)$. Увеличить $m_1$ если $U\left(X_i\right) < U\left(X_0\right)$, иначе увеличить $m_2$.
	\end{itemize}
Где  $m_1$: количество рекомендуемых переходов $x \rightarrow y$ так что $U\left(y\right) \leq U\left(x\right)$ (принять); 
	   $m_2$: количество рекомендуемых переходов $x \rightarrow y$ так что $U\left(y\right) \leq U\left(x\right)$ (понесено случайно) 
	\item Вычислить $^{\Delta_U}$ на $U\left(X_i\right) > U\left(X_0\right) $. 
	Где $^{\Delta_U}$ средняя стоимость минус $m_2$
	\item Вычислить $\chi \left(T\right) = \frac{m_1}{m_1+m_2}$
\end{itemize}

После выполнения алгоритма SA, полученного в модели MRF, эта модель представляется как взаимодействие между соседними пикселями. Этот сегмент изображения демонстрирует различие между большими дефектными областями и дефектами мелкой области на основе взаимодействия пикселей для группировки пикселей вместе (рис.\ref{pic23}). Преимущество использования этой модели заключается в том, что она удаляет мелкие детали или объединяет меньшие части в большую часть в процессе сегментации, чтобы идентифицировать наиболее дефектные области.

\begin{figure}[ht!]
\centering
\includegraphics[width=0.7\linewidth]{pic23}
\caption{Сегментация изображения дефектов дорожного покрытия на основе взаимодействия пикселей.}
	\label{pic23}
		\end{figure} 
		
\textbf{Алгоритм изменения перемещения и расширения для получения полной карты дефектов дорожного покрытия}. Эти сегменты называются «участками» и имеют предварительно определенную ориентацию 0, 45, 90 или 135 градусов. Разделение между обоими случаями осуществляется с помощью параметра $k \in \left(0,1\right)$. 

Эти дефектные карты обеспечивают мгновенную обратную связь с эффективностью параметров. Посредством итеративного процесса, оптимальные параметры обнаружения выбираются для каждого управления дорожного покрытия. После  выбора настройки, программа будет автоматически обрабатывать изображение дорожного покрытия для обнаружения дефектов. Для каждого дефекта, длина, ширина и ориентация вычисляются и сохраняются. Примером является цифровая карта дефектов, которая демонстрирует карту дефектов, соответствующую изображению.

\begin{algorithm}[H]
 1.Начать с произвольной маркировки. $f$\\
 2.Установить переменную success:= 0\;
 3.\For{\texttt{каждая пара метки ${D, N} \subset L$}}
     {
		Найти $\hat{f} = argminE(\acute{f})$ среди $\acute{f}$ в течение одного $D-N$ перекачка $f$\;
		\If{$E( \hat{f}) < E(f)$}{
   задавать $f := \hat{f}$\;
	 success := 1\;
   }
		}
4.\If{success = 1}{
   идти к 2\;
   }
5.Вернуть $f$
\caption{Шаги изменения алгоритма  перемещения и расширения.} \label{alg4}
\end{algorithm}

Время работы близко к линейному на практике. Некоторые результаты сегментации классов дефектов дорожного покрытия показаны на рис.\ref{pic24}.

\begin{figure}[ht!]
\centering
\includegraphics[width=0.7\linewidth]{pic24}
\caption{Результаты сегментации классов дефектов дорожного покрытия.}
	\label{pic24}
		\end{figure}
%--------------------------------------------------------
\subsection{Численный метод для обнаружения пузырьков}
Пусть графа $ G = \left\{V, E \right\} $ строится узлами $ V $ и ребрами $ E $. Рассмотрены пиксели изображения как узлы, два дополнительных узла $ S $ (источник) для объектов-пучков и $ T $ (терминал) для фона. Для построения взвешенного графа из изображения (рассматривается система $ d = \left(4,8 \right) $ - окрестности). Для каждой пары узлов t-link - это ребра. Функция веса присваивается каждой t-ссылке $ \left (B_ {p, q} \right) $  и приведится ниже.

\begin{equation} \label{eq9}
B_{p,q}= \exp - \frac{\left(I_p - I_q\right)}{2\sigma^2} \frac{1}{dist\left(p,q\right)}
\end{equation}

Где $ B_ {p, q} $ - измерение сходства интенсивностей изображения в пикселях $ p $ и $ q $. Для параметров инициализации алгоритма разреза на графах выбирается набор пикселей на переднем плане (объекты-пузырьки) и фон. Интерактивная сегментация учитывает мягкие ограничения (граничные и региональные свойства сегментов). Вес определяется отрицанием сходства. Область сегментации определяет инициализированные регионы на основе выбранных гистограмм.


\begin{equation} \label{eq10}
\left\{\begin{array}{l} W_p\left('FG'\right)=-\ln His\left(I_p|O\right),\\
W_p\left('BG'\right)=-\ln His\left(I_p|B\right),
\end{array}\right.
\end{equation}

Где $ His \ left (I_p | O \ right) $, $ His \ left (I_p | B \ right) $ - гистограммы интенсивности объекта и фона соответственно. Весы по краям графа назначаются в соответствии с алгоритмом разреза на графах (Алгоритм \ref{alg3}). Графический подход использует эффективные решения проблемы минимального разреза -- максимального потока между узлами источника и приемника в ориентированных графах.

В работе используется алгоритм минимального разреза -- максимального потока для реализации сегментации изображения пузырьков. В данном исследовании реализован этот алгоритм, который получен из реализации алгоритм минимального разреза -- максимального потока. На рис.\ref{pic30} показан результат построения карты с помощью признаков текстуры.

\begin{figure}[ht!]
\centering
\includegraphics[width=0.8\linewidth]{pic30.png}
\caption{Признаки текстуры, используемые при сегментации изображения.}
	\label{pic30}
	\end{figure}
	
	Результат сегментации изображения пузырьков с использованием базы данных текстуры на основе алгоритма разреза на графах (рис.\ref{pic31}). Изображение сегментируется в область для извлечения признаков пузырьков.
\begin{figure}[ht!]
\centering
\includegraphics[width=0.8\linewidth]{pic31.png}
\caption{Результат сегментации изображений пузырьков алгоритмом разреза на графах.}
	\label{pic31}
	\end{figure}

В предлагаемом методе используется алгоритм обнаружения пузырьков, основанный на вейвлет-преобразовании Хаара. Этот метод применяется к каждой области интереса - ROI, полученной из сегментации изображения с помощью разрезов на графах. Характерный признак вейвлет-преобразования Хаара является разделимым и легко вычисляется. В этом случае пузырьки воздуха в процессе являются ключевым изменением параметра. Следовательно, вейвлет-коэффициенты могут быть лучшим представлением сигнала, чем выборки во временной области.

Процесс извлечения объектов в изображениях на основе преобразований Хаара-Вавелета (рис.\ref{pic32}) выполняется алгоритмом \ref{alg5}.

\begin{algorithm}[ht!]
  \KwData{N изображений.}
  \KwResult{Значения вейвлет-преобразования Хаара (коэффициенты)}
		1. Преобразование изображения RGB в полутоновое изображение;
		
		2. Изменить размер изображения на $128 \times 128$;
		
		3. Предварительная обработка (фильтрация шума, морфологическая обработка)
		
		4. Вейвлет-преобразование для создания вектора $\vec{I_i}, i=1..N$;
		
		5. Среднее значение: $I_N=\frac{1}{N}\sum_{i=1}^N I_i$
		
		6. Расчет для каждого объекта: $\vec{u_i}=\sum_{k=1}^N V_{ik} \phi_k, i=1..N$\\
		$\phi_k=\vec{I_i}-\vec{I_N}$: вычитание среднего изображения из каждого изображения.
			$V_{ik}$: вектор каждой матрицы $W^tW, W=\left\{\vec{\phi_1},...,\vec{\phi_N}\right\} $
  \caption{Извлечение признаков пузырей вейвлет-преобразованием Хаара}\label{alg5}
\end{algorithm}

Для сравнения полученных значений вейвлет-преобразования Хаара используется универсальный порог $ T = \sigma \sqrt {2 \ln n} $, где $ \sigma $ - среднее абсолютное отклонение, а $ n $ - количество выборок конкретных коэффициентов вейвлет-преобразования. Частота дискретизации и масштаб вейвлет-разложения фиксированы $ T = k \sigma $, где $ k = \sqrt {2 \ln n} $ является параметром.
\begin{figure}[ht!]
\centering
\includegraphics[width=0.7\linewidth]{pic32}
\caption{Анализ с использованием вейвлет-преобразования Хаара для потока сигнала с пузырьками}
	\label{pic32}
	\end{figure}

Вейвлет-преобразование Хаара извлекается и создается вектор из подробного коэффициента, который вычисляется путем адаптивного порога. Порог проверяется на подробные коэффициенты. Если подробные коэффициенты превышают порог в конкретном случае, можно сделать вывод, что обнаружены пузырьки воздуха (рис.\ref{pic33}).
\begin{figure}[ht!]
\centering
\includegraphics[width=0.8\linewidth]{pic33}
\caption{Результат обнаружения пузырьков с использованием вейвлет-преобразований Хаара с измерением радиуса объекта.}
	\label{pic33}
	\end{figure}

\textbf{Классификация формы пузырьков на основе радиуса признак}

Пузырьки всегда существуют во множестве (fig.\ref{fig11})  разновидностей:одиночные, перекрывающиеся, погруженные в жидкость или растворенные в окружающей среде.
\begin{figure}[ht!]
\centering
\includegraphics[width=1\linewidth]{p11.png}
\caption{Типы существующих пузырьков.}
	\label{fig11}
	\end{figure}
	
Классификацию пузырей проводим на основе особенностей радиуса. Проводим сравнение расстояния между центром и радиусом пузырей, рассматривая окружающие пузыри. Если центр тяжести больше или равен радиусу, то пузыри существуют в единственной форме, иначе перекрываются.

%---------------------------------------------------------
\section{Математическое моделирование и численные методы для классификации объектов}
\subsection{Проблема несбалансированности данных в классификации объектов}

Данные несбалансированности исходят из идеи чувствительной стоимости изучения, в таком случае случайный лес больше подходит для изучения. Класс весового значения является важным параметром настройки для достижения желаемой производительности. В процедуре построенния дерева индукции, критерий Джини используется при  определении весового значения, чтобы находить расколы. В концевых узлах каждого дерева, вновь принимается во внимание весовой класс . Введем понятие \cite{h143}: Правильный позитив (True Positive - TP) - правильно классифицирован как положительный, правильный негатив (True Negative - TN) - правильно классифицирован как негативный, неправильный позитив (False Positive – FP) - классифицирован неправильно как положительный, неправильный негатив (False Negative - FN) - ошибочно классифицируется как негативный. Для алгоритма случайного леса, всегда существует компромисс между правильным позитивом  и правильным негативом и то же самое применяется для отзыва и точности.

Правильный негатив ставки $=\frac{TN}{TN+FP}$, Правильный позитив ставки $=\frac{TP}{TP+FN}$, Точность $=\frac{TP}{TP+FP}$

\textbf{Несбалансированность данных в классификации дефектов дорожного покрытия}. Обучающий набор состоит из 500 изображений (200 классов «дефектов» и 300 классов «без дефектов» (таб. \ref{tab3}). В 200 изображений классов дефекта включены 150 изображений (50 сеть трещин, 50 глубокие трещины, 50 выбоины изображений) для тренировочного процесса и 50 изображений для процесса тестирования. В 300 изображениях класса без дефектов включены 200 для процесса обучения и 100 изображений для процесса тестирования. Данные изображения были построены с помощью «центра телекоммуникаций и мультимедиа», INESC TEC, Португалия и наш собственный набор данных, который собирается с помощью камеры (Canon D100 16 Мп). Изображения захватываются в обычном состоянии дневного света, расстояние от камеры до поверхности дороги 1м-1.2м. Затем изображения обрабатываются с разрешением (256 * 256) пикселей и (500 * 500) пикселей.

	 \begin{table}[h!]%
\centering
\caption{Лучшая модель для классификации зависит от точности, истинной положительной ставки, ложной положительной ставки в системе ОКДДП.}
\label{tab3}
  \begin{tabular}{|c|c|c|c|}
    \hline														
     \multirow {2}{*}    {Класс}      & {Истинно}            & {Ложно}          & {Точность}\\
		                                  & {положительный}      & {положительный}  &{}\\
    \hline
Выбоины                   &0.903 	&0.556 	&0.843\\
\hline 
Сеть трещин 	&0.80 	&0.726 	&0.880\\
\hline 
Глубокие трещины	&0.947 	&0.230 	&0.926\\
\hline 
  \end{tabular}
\end{table}%\vspace{10mm}

%\textbf{Несбалансированность данных в классификации формы пузырьков}. Данные включают в себя 2 изображений База данных <<ShellScratch>> и <<TrueTear>> (таб. \ref{tab31}) , которые полностью описывают информацию в формате * .png и разрешение 1300 * 1030 пикселей, предоставленное лабораторией в Сингапуре. Данные разделены на две части: 2/3 данных для обучения и 1/3 данных для тестировани данных. Затем изображения будут уменьшены в размере до того же разрешения (128 * 128) или (256 * 256). Диапазон размеров для пузырьков составляет $>$ 1 пиксель, а форма пузырька $>$ 10 пикселей.
%
	 %\begin{table}[h!]%
%\centering
%\caption{Лучшая модель для классификации зависит от точности, истинной положительной ставки, ложной положительной ставки в системе ОКФП.}
%\label{tab31}
  %\begin{tabular}{|c|c|c|c|}
    %\hline														
     %\multirow {2}{*}    {База данных}      & {Истинно}            & {Ложно}          & {Точность}\\
		                                  %& {положительный}      & {положительный}  &{}\\
    %\hline
%ShellScratch                &0.910 	&0.603 	&0.925\\
%\hline 
%TrueTear	&0.928 	&0.176 	&0.944\\
%\hline 
%
  %\end{tabular}
%\end{table}%\vspace{10mm}
%------------------------------------------------------------
\subsection{Математическое моделирование и численные методы для классификации объектов в системе ОКДДП}
\textbf{Математическое моделирование для классификации объектов в системе ОКДДП}

Пусть $S=\left\{X, Y\right\}$ - множество N обучающих выборок, набор дефектов дорожного покрытия признаков $X=\left\{x_i | i = 1, ..., 5\right\}$ и набор меток классов $Y = \left\{y_i |i = 1, ..., 3\right\}$. Классификатор - это функция $H: X \rightarrow Y$, которая отображает $x$ в элемент $y$ из $Y$. Это значит, построение модели обучения и тестирования для классификации объектов по меткам $Y$.

Используется алгоритм случайного леса для классификации дефектов: сети трещин, глубоких трещин и выбоин. Таким образом, каждое дерево строится на множестве $N_t$, это множество случайным образом выводится из $N$. Данные обучения $N_p$ узла $p$ делятся на два набора: $N_l$ - слева и $N_r$ – справа должны следовать пороговое значение $c$ характеристического вектора $v$ в функции $f$:
\begin{equation}\label{eq11}
N_l = \left\{n \in N_p | f\left(v_n\right) > c\right\};  \\
N_r=N_p / N_l.
\end{equation}
На каждом узле создается группа $ m $ -функции, которая случайным образом предлагается для функции $ f $ и находит все возможные значения $ c $ и выбирает из них значение. Чтобы получить самый высокий индекс Джини \cite{19wl}:
\begin{equation}\label{eq12}
\Delta I_G\left(N_p\right)=I_G\left(N_p\right) - \frac{|N_l|}{|N_p|}I_G\left(N_l\right) - \frac{|N_r|}{|N_p|}I_G\left(N_r\right), 
\end{equation} где $I_G\left(N\right)$ - индекс Джини для данных изучения $N$.

\textbf{Численный метод для классификации на основе алгоритма случайного леса}.

Случайный лес - это метод машинного обучения, основанный на комбинированном обучении, который является методом создания множественных классификаторов и синтезирует их результаты для получения конечного результата.

Случайный лес генерирует множество деревьев решений, а именно алгоритм классификации и регрессии (CART) и использует метод Bagging. Каждое дерево обучается с помощью бутстрапа, который получен из исходного набора данных. Для создания методов обучения и тестирования в случайном лесу использовалась техника «Out-Of-Bag».

Численный метод основан на алгоритме случайного леса (алг.\ref{alg4}) для решения проблемы классификации дефектов дорожных покрытий в системе ОКДДП.
\begin{algorithm}[ht!]
  \KwData{$T$ - количество деревьев; $N$ - количество выборок из набора данных.}
  \KwResult{$y\left(x_i\right)$ - метка для объекта}
   1. \For{\texttt{$t := 1$ to $T$ }}
     {
		1.1 Создавать $M$ - новый шаблон данных с $n$ - размер $M$ из исходного набора данных $N$;
		
		1.2 Создавать дерево классификаторов для каждого$M$;
		
		1.3 \For{\texttt{$i := 1$ ко всем узлам}}
     {
		  1.3.1 Получить случайные  $mtry$ признаки из оригинальной функции;
			
			1.3.2. Выбор лучшего раскола в $mtry$ признаке;
		}
		1.4 $y\left(x_t\right)$ = метка $t$ дерева; 
		}
		
		2. Вернуть $y\left(x_i\right)$ = $\left\{y\left(x_t\right)\right\}^T$ - большинство голосов;
		
  \caption{Классификация объектов на основе алгоритма случайного леса.}\label{alg4}
	
\end{algorithm}

В данном разделе описывается классификация, основанная на методе обучения с учителем (рис. \ref{pic26}) -- алгоритм случайного леса в дальнейшем принимает концепцию дерева решений, производя большое количество деревьев решений. Сначала извлекается случайная выборка данных и определяется ключевой набор признаков, чтобы вырастить каждое дерево решений. Эти деревья решений, то есть их «Out-Of-Bag» определяют ошибку (частота ошибок модели), а затем ряд деревьев решений сравнивается, чтобы найти совместный набор переменных, которые производят самую точную модель классификации.

\begin{figure}[ht!]
\centering
\includegraphics[width=0.9\linewidth]{pic26}
\caption{Блок схема классификации объектов в системе ОКДДП.}
	\label{pic26}
		\end{figure}	
		
%Классификатор Random Forest был построен с использованием пакета Random Forest 4.5-16 для статистической среды \cite{h142}, чтобы классифицировать векторы признаков по наличию дефекта. Классификатор прошел обучение по набору данных для дорожного покрытия  с помощью цепного кода гистограммы - CCH, Hu-момент, размер дефекта для каждого случая в системе ОКДДП.
%
%В системе ОКФП использовался алгоритм случайного леса (алг.\ref{alg4}), чтобы классифицировать форму пузырьков с целью обнаружения метки между двумя пересекающимися пузырьками. Вектор признаков создается извлеченными значениями  признаков, которые являются: признаками текстуры, признаками геометрии, спектральными признаками. Объект на изображении будет помечен двумя метками $ \left\{- 1; +1\right\}$.
%------------------------------------------------
\section{Применение методов и алгоритмов компьютерного зрения и машинного обучения для обнаружения и классификации объектов на изображениях}
В этом разделе описывается процесс применения алгоритмов и методов машинного обучения \ref{pic27} для извлечения дефектов дорожного покрытия (гистограмма цепного кода, локальные шаблоны, размер дефекта области – ширина, длина, площадь), построение карты дефектов дорожного покрытия и классификация данных представлена в разделе \ref{partc} ; обнаруживается радиус пузырька, классифицируется форма пузырька в системе, как показано в разделе \ref{partd}.
\begin{figure}[ht!]
\centering
\includegraphics[width=0.8\linewidth]{pic27}
\caption{Блок-схема обнаружения и классификации объектов в системе ОКДДП и ОКФП.}
	\label{pic27}
		\end{figure}
%----------------------------------
\subsection{Применение методов и алгоритмов компьютерного зрения и машинного обучения для обнаружения и классификации дефектов дорожного покрытия  на изображениях в системе ОКДДП} \label{partc}
В результате проведенного анализа в разделе \ref{part2} установлено, что основы метода машинного обучения включают в себя следующие этапы (рис \ref{pic28}) для обнаружения и классификации дефектов дорожного покрытия:

\begin{itemize}
	\item \textbf{Шаг 1 предварительная обработка изображения}
	
	\begin{itemize}
	\item преобразование в полутоновое изображение;
	\item изменение размера изображение до $512*512$ пкс;
	\item Подавление шума линейным фильтром Гаусса; 
	\item баланс гистограммы;
	\item использование морфологических операторов -- дилатаций, чтобы улучшить качество точек изображения со структурами элементов.
	\end{itemize}

\item \textbf{Шаг 2 сегментация изображения и постройние карты дефектов дорожного покрытия}

\begin{itemize}
	\item сегментация изображения алгоритмом минимального разреза -- максимального потока;
	\item оптимизация сегментированного изображения на основе алгоритма случайных полей Маркова;
	\item алгоритмы изменения перемещения и расширения для получения полной карты дефектов дорожного покрытия.
\end{itemize}

\item \textbf{Шаг 3 извлечение признаков}

\begin{itemize}
	\item гистограмма ценпного кода;
	\item ланкальные бинарные шаблоны;
	\item размер дефекта области (ширина, длина, площадь);
	\item создание антропометрического вектора признаков.
\end{itemize}
\item \textbf{Шаг 4 классификация типов дефектов дорожного покрытия по новым данным}
 
 \begin{itemize}
	 \item процесса обучения;
	 \item процесса тестирования;
 \end{itemize}

\end{itemize}
\begin{figure}[ht!]
\centering
\includegraphics[width=0.8\linewidth]{pic28}
\caption{Блок-схема обнаружения и классификация дефектов дорожного покрытия в системе ОКДДП.}
	\label{pic28}
		\end{figure}

%------------------------------------
\subsection{Применение методов и алгоритмов компьютерного зрения для обнаружения и классификации формы пузырьков на изображении в системе ОКФП} \label{partd}
В данной работе рассматривается задача извлечения  признаков пузырьков на изображениях. Для решения этой задачи предложен алгоритм, основанный на совместном применении предложенного алгоритма машинного обучения компьютерного зрения и техника анализа изображения (рис. \ref{pic29} описывает основные этапы этого процесса. 

\begin{itemize}
	\item \textbf{Шаг 1 предварительная обработка изображения}
	
	\begin{itemize}
	\item преобразование в полутоновое изображение;
	\item изменение размера изображения на $128 \times 128$;
	\item подавление шума линейным фильтром Гаусса; 
	\item баланс гистограммы;
	\item обнаружение кромок - алгоритм Canny;
	\item использование морфологических операторов.

	\end{itemize}

\item \textbf{Шаг 2 сегментация изображения и обнаружение пузырьков}

\begin{itemize}
	\item сегментация изображения алгоритмом минимального разреза -- максимального потока;
	\item извлечение признаков пузырей вейвлет-преобразованием Хаара;
	\item сравнение со значением адаптивного порога.

\end{itemize}

\item \textbf{Шаг 3 извлечение признаков}

\begin{itemize}
	\item спектральный признак;
	\item геометрический признак;
	\item структурный признак;
	\item создание антропометрического вектора признаков.
\end{itemize}
\item \textbf{Шаг 4 классификация формы пузырьков с новыми данными}

\end{itemize}
\begin{figure}[ht!]
\centering
\includegraphics[width=0.8\linewidth]{pic29}
\caption{Блок-схема обнаружения и классификация формы пузырьков в системе ОКФП.}
	\label{pic29}
		\end{figure}
%%---------2.5------------------%
\section{Основные результаты и выводы по главе 2}

\begin{enumerate}
	\item В главе 2 представлено математическое моделирование, численные методы, основанные на методах компьютерного зрения и машинного обучения, комбинированные методы обработки изображений для решения проблемы обнаружения и классификации объектов в системах ОКДДП и ОКФП.
\item  В главе 2 основное внимание уделялось задачам обработки данных, предварительной обработки изображений, улучшению качества входных данных с помощью алгоритмов обработки изображений. Оценивалась база данных, собранную для системы обработки.
\item В этой главе были представлены математические модели и численные методы, основанные на алгоритме разреза на графах и алгоритме случайного Марковского поля для сегментирования изображений и получения интересующих областей - областей изображения, которые могут содержать объекты, подлежащие рассмотрению: дефекты дорожного покрытия, пузырьки воздуха.
\item В исследовании представлены признаки объектов двух систем (дефекты дорожного покрытия, пузырьки воздуха) и метод извлечения признаков.
\item В этой главе построена математическая модель и используется численный метод, основанный на алгоритме случайного леса, для классификации объектов.
\item Подробное описание шагов процесса обнаружения и классификации объектов в ОКДДП и ОКФП представлено на изображении.
На основе результатов сбора и предварительной обработки изображений, улучшения изображения, извлечения признаков, математического моделирования и численных методов подробно описаны реальные этапы работы. Целью исследования является обнаружение и классификация объектов на изображении в главе \ref{chapt2}. Исследовательская работа по созданию программы, объединяющей автоматическое обнаружение и классификацию объектов на изображении, представлена в главе \ref{chapt3}.

\end{enumerate}


%%%%%%%%%%%%%%%%%%%%%%%  ГЛАВА 3 %%%%%%%%%%%%%%%%%%%%%%%%%%%%%%%%%%%%%%%%%
\chapter{Реализация методов компьютерного зрения и машинного обучения для классификации объектов в системах ОКДДП и ОКФП} \label{chapt3}
В этой главе дается описание реализации алгоритмов и методов машинного обучения для построения систем ОКДДП и ОКФП для которых в главе 2 были разработаны соответствующие математические модели. Описана структура программ ОКДДП и ОКФП и приведены результаты экспериментов для оценки разработанного метода обнаружения и классификации данных на изображении. Проанализирована эффективность предлагаемого метода в сравнении с другими передовыми методами. 
%%---------3.1-----------------%
\section{Структура программы обнаружения и классификация объектов в системах ОКДДП и ОКФП}
Программная система состоит из 5 модульных структур, которые предствляются структура программ ОКДДП (рис. \ref{pic47}) и ОКФП (рис. \ref{pic59}). Структура системы предназначена для обеспечения того, чтобы каждый модуль имел работающие функции для выполнения одной и той же задачи и мог расширять сферу разработки системы научно обоснованным образом.

\begin{figure}[ht!]
\centering
\includegraphics[width=1\linewidth]{pic47}
\caption{Структура ПО системы ОКДДП}
	\label{pic47}
		\end{figure} 
		
\begin{figure}[ht!]
\centering
\vspace{-0.8em}
\includegraphics [width=0.8\linewidth]{images/pic59.png}
%\captionsetup{justification=justified, labelsep=period}
\caption{Структура ПО системы ОКФП} \label{pic59}
\end{figure}
В качестве входных данных для обнаружения и классификации объектов используются изображение. Процесс улучшения качества изображения и повышения эффективности последующих процессов. Сегментация изображений определяет области изображения, называемые областями интереса (ROI) - местами с высокой вероятностью содержащихся объектов. Эти регионы будут обрабатываться на этапе извлечения признаков, чтобы создать вектор признаков, который содержит описательные значения для каждого типа объектов. Наконец,  признаки были извлечены из объектов, эти объекты проходили процесс классификации, чтобы увидеть, какому классу принадлежит объект, какой тип дефекта, какая форма пузыря.
%-------------3.1.1
\subsection{Модуль ввода/вывода данных}
Блок ввода данных в системе выполняет задачи: хранения изображений, корректировки формата и размера базы данных в стандарте, ввода параметров для алгоритма, отображения результатов системы.

Класс «InOutData» используется для хранения изображений и результатов системы. Структура класса приводится «InOutData» на рис.\ref{pic34}. Класс «InOutData» приведен в таб.\ref{tab5}.

\begin{figure}[ht!]
\centering
\includegraphics[width=0.5\linewidth]{pic34}
\caption{Структура класса InOutData.}
	\label{pic34}
		\end{figure}
		
\begin{table}[h!]%
\centering
\caption{Основные переменные и функции класса InOutData.}
\label{tab5}
  \begin{tabular}{|c|c|}
    \hline
           Название переменной и функции  & Описание \\
\hline 
 \multirow {2}{*}    {inImg: Image  }      & {Переменное хранение}\\
		                                       & {входных изображений.}\\

\hline 
\multirow{2}{*} {outImg: Image} 	         &{Переменное хранение}\\
																					&{выходных изображений.}\\
\hline 
\multirow{2}{*} {label}	                  &{Метка обрабатываемого}\\ 
																					&{объекта.}\\
\hline 
\multirow{2}{*} {LoadData()}	             &{Функция ввода данных}\\
																					&{в систему.}\\
\hline 
\multirow{2}{*} {ReturnData()}	           &{Функция вывода данных}\\
																						&{в систему.}\\
\hline 
  \end{tabular}
\end{table}%\vspace{10mm}
%-------------3.1.2
\subsection{Модуль предварительной обработки изображения}
Блок предварительной обработки изображения можно рассматривать как первый блок в системе для обнаружения и классификации объектов на изображении. Из-за этого точность данного блока будет предоставлять данные для сегментации изображения и обнаружения объектов, извлеченных признаков блоков. Это существенно влияет на эффективность и надежность всей системы обработки.

Технология обработки изображений включает в себя: фильтрацию шума, балансировку гистограмм, обнаружение границ и морфологические операции  для получения изображений хорошего качества, совершенного периметра и полного формата.

Класс «PreProcessImg» предназначен для обработки изображений. Этот класс включает такие процессы, как коррекцуию размера изображения, фильтрацию шума, морфологию, балансировку гистограмм, обнаружение границ. Класс «PreProcessImg» представлен на рис.\ref{pic35}. Основные функции класса «PreProcessImg» приведены в таб.\ref{tab6}.

\begin{figure}[ht!]
\centering
\includegraphics[width=0.5\linewidth]{pic35}
\caption{Структура класса PreProcessImg.}
	\label{pic35}
		\end{figure}
		
\begin{table}[h!]%
\centering
\caption{Основные переменные и функции класса PreprocessImg.}
\label{tab6}
  \begin{tabular}{|c|c|}
    \hline
           Название переменной и функции  & Описание \\
\hline 
\multirow{2}{*} {Img: Image}           &{Переменные изображения} \\
																			&{необходимо обработать.}\\
\hline 
\multirow{2}{*} {ResizeImg()}	         &{Функция регулировки}\\
																		&{размера изображения.}\\
\hline 
\multirow{3}{*} {FilterGauss()}	            &{Функция  фильтрующий шум и}\\
																						&{гладкое изображение} \\
																						&{по гауссовскому методу.}\\
\hline 
\multirow{2}{*} {Morphology()}	             &{Функция использует} \\
																						&{морфологические операции.}\\
\hline 
\multirow{2}{*} {EqHist()}	                &{Функция балансировка}\\
																							&{гистограмм.}\\
\hline 
\multirow{2}{*} {EdgeCanny()}                &{Функция обнаружения края} \\
																							&{методом Canny.}\\
\hline 
  \end{tabular}
\end{table}%\vspace{10mm}
%--------------3.1.3
\subsection{Модуль сегментации изображения}
Сегментация изображений играет важную роль в общей системе. Этот процесс определяет, какие области изображения будут обработаны и какие объекты будут идентифицированы. Точность процесса - это фактор, который сокращает время обработки, повышает надежность и стабильность системы. Этот процесс состоит из двух этапов: сегментация изображений и построение карты дефектов дорожного покрытия для систем ОКДДП и обнаружение формы пузырьков для систем ОКФП.

Класс «SegmentImg» предназначен для сегментации изображений. Этот класс включает в себя такие процессы, как сегментация изображения, построение карты дефектов дорожного покрытия для систем ОКДДП и обнаружение пузырьков для систем ОКФП. Класса «SegmentImg» представлена на рис.\ref{pic36} Основные функции класса ImageProcessing приведены в таб. \ref{tab7}.
\begin{figure}[ht!]
\centering
\includegraphics[width=0.5\linewidth]{pic36}
\caption{Структура класса SegmentImg.}
	\label{pic36}
		\end{figure}
		
\begin{table}[h!]%
\centering
\caption{Основные переменные и функции класса SegmentImg.}
\label{tab7}
  \begin{tabular}{|c|c|}
    \hline
           Название переменной и функции  & Описание \\
\hline 
\multirow{2}{*} {Img: Image}         &{Переменная содержит изображение}\\
																			&{для процесса сегментации.}\\
\hline 
\multirow{2}{*} {ncut}                &{Количество срезов алгоритма}\\
																			&{разреза на графах.}\\
\hline 
\multirow{2}{*} {Maxthresh}	           &{Максимальное значение}\\
																			&{порога.}\\
\hline 
\multirow{2}{*} {Minthresh}             &{Минимальное значение}\\
																				&{порога.}\\
\hline
 \multirow{2}{*} {SegImage()}             &{Функция сегментации}\\
																				&{изображения.}\\
\hline
\multirow{3}{*} {Buildmap()}             &{Функция построение карту}\\
																				&{дефектов дорожного покрытия}\\
																				&{для систем ОКДДП.}\\
\hline
\multirow{2}{*} {DetectBubbles()}       &{Функция обнаружение}\\
																				&{пузырьков для систем ОКФП.}\\
\hline
  \end{tabular}
\end{table}%\vspace{10mm}
%-------------3.1.4
\subsection{Модуль извлечения признаков}
Блок извлечение признаков незаменим в системе. Задачей блока является анализ, использование информации об объекте на изображении для создания вектора признаков. Результатом этого блока является вход в блок классификации объектов. Выбор признаков является определяющим фактором эффективности процесса классификации. Для разных типов объектов выбранные функции будут разными, в зависимости от наиболее соответствующего описания объекта.

Класс «FeaExtrac» предназначен для извлечения признаков. Этот класс состоит из процессов: извлечения признака текстуры, признака контура, геометрических признаков и вейвлет-функций. Класса «FeaExtract» представлена на рис.\ref{pic37} Основные функции класса ImageProcessing приведены в таб.\ref{tab8}.
\begin{figure}[ht!]
\centering
\includegraphics[width=0.5\linewidth]{pic37}
\caption{Структура класса FeaExtract.}
	\label{pic37}
		\end{figure}
	
	\begin{table}[h!]%
\centering
\caption{Основные переменные и функции класса FeaExtract.}
\label{tab8}
  \begin{tabular}{|c|c|}
    \hline
           Название переменной и функции  & Описание \\
\hline 
\multirow{2}{*} {ROI: Image} &{Переменная содержит}\\
															&{интересующую область.} \\
\hline
\multirow{3}{*} {hightCoeff}  &{Массив вейвлет-коэффициентов}\\
															&{для вычисления значения}\\
															&{аппроксимации.} \\
\hline 
\multirow{3}{*} {lowCoeff}    &{Массив вейвлет-коэффициентов}\\
															&{для вычисления значения}\\
															&{различия.} \\
\hline
\multirow{2}{*} {length}      &{Количество коэффициентов}\\
															&{вейвлет-преобразования} \\
\hline
\multirow{2}{*} {ExtractContour()} &{Функция извлечения}\\
																	&{признака контура.}\\
\hline
\multirow{2}{*} {ExtractTexture()} &{Функция извлечения}\\
																		&{признака текстуры.}\\
\hline
\multirow{2}{*} {ExtractGeometry()}  &{Фукция извлечения}\\
																			&{геометрических признаков.}\\
\hline
\multirow{2}{*} {ExtractWavelet()}  &{Функция извлечения признака}\\
																		&{вейвлет-функций.}\\
\hline
\multirow{3}{*} {getHighCoeff()}    &{Доступ к массиву коэффициентов} \\
																		&{для вычисления значения}\\
																		&{аппроксимации.}\\
\hline
\multirow{3}{*} {getLowCoeff()}     &{Доступ к массиву коэффициентов}\\
																		&{для вычисления значения}\\
																		&{различия.}\\
\hline 
\multirow{2}{*} {getLength()}       &{Доступ к количеству коэффициентов}\\
																		&{вейвлет-преобразования.} \\
\hline
  \end{tabular}
\end{table}%\vspace{10mm}
%-------------3.1.5
\subsection{Модуль классификации объектов}
Классификация объектов критична ко всей системе, поскольку эффективность этого блока напрямую влияет на выход всей системы. Следовательно, хорошее разрешение в обучении и тестировании, создание хорошей модели классификации приведет к точности и надежности системы для обнаружения и классификации объектов. Чтобы создать эффективную и точную систему для идентификации и классификации рабочих объектов, важно объединить блоки анализа и обработки данных, построить математическую модель и использовать хорошие численные методы для решения проблемы.

Класс «ClassifierObject» предназначен для классификации объектов. Этот класс состоит из процессов: обучения данных и тестирования с новыми данными. Класс «ClassifierObject» на рис.\ref{pic38} Основные функции класса ImageProcessing приведен в таб.\ref{tab9}.
\begin{figure}[ht!]
\centering
\includegraphics[width=0.5\linewidth]{pic38}
\caption{Структура класса ClassifierObject.}
	\label{pic38}
		\end{figure}
		
\begin{table}[h!]%
\centering
\caption{Основные переменные и функции класса ClassifierObject.}
\label{tab9}
  \begin{tabular}{|c|c|}
    \hline
           Название переменной и функции  & Описание \\
\hline 
mtry : int  &Параметры обучения.\\
\hline
deep: int   &Глубина дерева.\\
\hline
\multirow{2}{*} {ntree: int}   &{Количество классифицированных}\\
																&{деревьев.}\\
\hline
feaVector: Vector  &Векторная признаки объекта.\\
\hline 
TrainProcess()    &Функция выполняет обучение.\\
\hline
TestProcess()     &Функция выполняет тестирование.\\
\hline
  \end{tabular}
\end{table}%\vspace{10mm}
%%=========3.2-----------------%
\section{Постановка экспериментов}
Для оценки эффективности алгоритмов и методов компьютерного зрения и машинного обучения, которые предлагаются для разработки систем ОКДДП и ОКФП, необходимо провести эксперименты с фактическими данными изображения.

Эти эксперименты были направлены на оценку факторов, влияющих на результат программы, таких как шум, условия окружающей среды и оборудование приемника. Оценивалось рабочее время системы, чтобы обеспечить работу в реальном времени.

Эксперимент рассмотрел возможность и эффективность алгоритмов и методов машинного обучения на каждом этапе обработки программы шаг за шагом: предварительная обработка изображения, обнаружение области изображения, извлечение признаков, классификация объектов в системах ОКДДП и ОКФП.

В этом исследовании проведена оценка пригодности структуры программы для каждого типа данных. Эксперименты проводились с двумя наборами данных для каждой системы. В частности данные для системы ОКДДП состоят из двух наборов данных, собранных в городе Иркутск(Россия) и в городе Тхаи Нгуен(Вьетнам). Данные для системы ОКФП состоят из двух наборов данных, представленных в лаборатории в Сингапуре.

В этом разделе рассмотрены следующие факторы:

\begin{itemize}
\item \textbf{СКО- среднеквадратическая ошибка};
\begin{equation}\label{eq13}
СКО=\frac{1}{n} \sum_{i=1}^n (f(x_i)-y_i)^2,
\end{equation}

Где $n$ - количество тестовых примеров, $f\left(x_i\right)$ - вероятностный выход  классификатора на $x_i$ и $y_i$ является фактическими этикетками. Алгоритм случайного леса быстро обучается, но они часто требуют громоздких деревьев. Случайный лес не достаточно подходит , но исползуется алгоритм проверки ошибок.

 \item \textbf{Время исполнения};
\begin{equation}
cT\sqrt{MN}\log N,
\end{equation}

Где $c$ постоянно зависит от сложности данных, $T$ количество дерева, $M$ количество переменных and $N$ количество экземпляров. 
\end{itemize}

%%----------3.2.1----------------%
\subsection{Экспериментальная оценка системы обнаружения и классификации дефектов дорожного покрытия на изображениях}
Классификатор был построен с использованием параметров $ntree=\left(50, 100\right)$, $mtry=2$ и $depth=\left(2, 5\right)$. В таблице \ref{tab4}, \ref{tab10} показывается эффект увеличения количества деревьев в ансамбле. Для обоих, увеличение деревьев требует больше времени для обучения, но и обеспечить лучшие результаты в терминах среднеквадратичной ошибки (СКО)  и рассчитывается следующим образом:

\textbf{а) База данных дефектов дорожного покрытия в Иркутске (России)}

\begin{figure}[ht!]
\centering
\includegraphics[width=0.8\linewidth]{pic39}
\caption{База данных дефектов дорожного покрытия в Иркутске.}
	\label{pic39}
		\end{figure}
	
	\begin{table}[h]
			\centering
			\begin{tabular}{|c|c|c|c|c|}
\hline
 &\multicolumn{2}{|c|}{Случайный лес}&\multicolumn{2}{|c|}{ Boosting(GBTs)}\\
\cline{2-5}
 \multirow{3}{*}&Количество &Количество &Количество &Количество \\
 &деревьев:50 &деревьев:100 &деревьев:50 &деревьев:100\\
 &Глубина:2 &Глубина:5  &Глубина:2 &Глубина5 \\ \hline 
Время обучения(sec)&150&257&140&278\\
СКО             &0.393&0.366&0.3&0.516 \\ \hline 
\end{tabular}
\caption{Время обучения, правильная скорость и проверка ошибок алгоритмов классификации: случайный лес и активизация (boosting-GBTs)} \label{tab4}
		\end{table}
	
\begin{figure}[ht!]
\centering
\includegraphics[width=0.5\linewidth]{pic41}
\caption{Результаты классификации базы данных дефектов дорожного покрытия в Иркутске.}
	\label{pic41}
		\end{figure}
		
\textbf{б) База данных дефектов дорожного покрытия в Тхай Нгуене(Вьетнам)}

\begin{figure}[ht!]
\centering
\includegraphics[width=0.8\linewidth]{pic40}
\caption{База данных дефектов дорожного покрытия в Тхай Нгуене.}
	\label{pic40}
		\end{figure}
	
	\begin{table}[h]
			\centering
			\begin{tabular}{|c|c|c|c|c|}
\hline
 &\multicolumn{2}{|c|}{Случайный лес}&\multicolumn{2}{|c|}{ Boosting(GBTs)}\\
\cline{2-5}
 \multirow{3}{*} &Количество &Количество &Количество &Количество \\
                 &деревьев:50 &деревьев:100 &деревьев:50 &деревьев:100\\
                 &Глубина:2 &Глубина:5  &Глубина:2 &Глубина: 5 \\ 
\hline 
Время обучения (sec)&775&1132&830&1270\\
СКО             &0.230&0.296&0.280&0.311 \\ \hline 
\end{tabular}
\caption{Время обучения, правильная скорость и проверка ошибок алгоритмов классификации: случайный лес и активизации (boosting-GBTs)} \label{tab10}
		\end{table}
		
\begin{figure}[ht!]
\centering
\includegraphics[width=0.5\linewidth]{pic42}
\caption{Результаты классификации базы данных дефектов дорожного покрытия в Тхай Нгуене.}
	\label{pic42}
		\end{figure}
		
Эксперименты показывают, что больше деревьев всегда лучше с убывающим возвращением. Более разветвленные деревья почти всегда лучше, при условии аналогичной производительности. Два вышеуказанных положения являются  непосредственно результатом  сочетания диагональной дисперсии. Более разветвленные деревья уменьшает отклонение; большее количество деревьев  уменьшает дисперсию. Есть несколько способов, чтобы контролировать насколько разветвлены наши деревья (ограничение на максимальную глубину, ограничение количества узлов, ограничение количества объектов, необходимых для разделения, остановка расщепления, если разделение идет не в достаточной степени, улучшение подгонки, и т.д.). В большинстве случаев  рекомендуется подрезать деревья (ограничивать глубину), если мы имеем дело с зашумленных данных. Наконец, можно использовать наши полностью развитые деревья, чтобы вычислить производительность коротких деревьев, поскольку они представляют собой «подмножество»  полностью развитых.
%%-----------3.2.2---------------%
\subsection{Экспериментальная оценка системы обнаружения и классификации формы пузырьков на изображениях}
Точность системы обнаружения пузырьков во многом зависит от выбора комбинаций алгоритмов и методов обработки изображений (вкладка \ ref {tab1}). В этой работе были проведены эксперименты с 30 изображениями по объединению алгоритмов сегментации изображений с вейвлет-преобразованием Хаара (O-HWT), Wavelet Transform (W-HWT), преобразованием графа-Haar Wavelet Transform (GC-HWT), чтобы продемонстрировать эффективность алгоритма разрезания Графа, сокращения Графа - преобразование Hough (GC - HT), чтобы правильно продемонстрировать Haar-Wavelet Transforms для решения задачи.
\begin{figure}[ht!]
\centering
\includegraphics[width=0.8\linewidth]{pic45}
\caption{База данных <<ShellScratch>>.}
	\label{pic45}
		\end{figure}
 \begin{table}[h!]%
\centering
\caption{Отображены результаты обнаружения пузырьков с использованием метода компьютерного зрения.}
\label{tab1}
  \begin{tabular}{|c|c|c|c|c|}
    \hline
                            &O-HWT & W-HWT&GC-HWT&GC-HT  \\
    \hline
Правильный образец                      &21	&23	&27	&25\\
\hline 
Неправильный образец                       &9   &7  &3  &5\\
\hline 
Правильный образец (\%)                 &70  &76.66 &90 &83.3\\
\hline 
Время работы (128 *128) (s) &0.25	&0.82	&1.13	&1.3 \\
\hline 
Время работы (256 *256) (s) &0.67	&1.08	&1.79	&1.62 \\
\hline
  \end{tabular}
\end{table}%\vspace{10mm}
%Классификатор был построен с использованием параметров $mtry=3$ и $depth=\left(2, 5\right)$. В таблице \ref{tab11}, \ref{tab12} показано время обучения, истинная скорость и проверка ошибок алгоритмов классификации: случайный лес и решение дерева.
%
%\textbf{а) База данных <<ShellScratch>>}
%
%\begin{figure}[ht!]
%\centering
%\includegraphics[width=0.8\linewidth]{pic45}
%\caption{База данных <<ShellScratch>>.}
	%\label{pic45}
		%\end{figure}
	%
	%\begin{table}[h]
			%\centering
			%\begin{tabular}{|c|c|c|c|c|}
%\hline
 %&\multicolumn{2}{|c|}{Случайный лес}&\multicolumn{2}{|c|}{Решение дерево}\\
%\cline{2-5}
 %&Глубина:2 &Глубина:5  &Глубина:2 &Глубина: 5 \\ \hline 
%Время обучения (sec)&171&203&82&110\\
%MSE             &0.201&0.414&0.274&0.507 \\ \hline 
%\end{tabular}
%\caption{Время обучения, истинная скорость и проверка ошибок алгоритмов классификации: случайный лес и решение дерева} \label{tab11}
		%\end{table}
	%
%\begin{figure}[ht!]
%\centering
%\includegraphics[width=0.5\linewidth]{pic43}
%\caption{Результаты классификации базы данных <<ShellScratch>>.}
	%\label{pic43}
		%\end{figure}
%\textbf{б) База данных <<TrueTear>>}
%
%\begin{figure}[ht!]
%\centering
%\includegraphics[width=0.5\linewidth]{pic46}
%\caption{База данных <<TrueTear>>.}
	%\label{pic46}
		%\end{figure}
	%
	%\begin{table}[h]
			%\centering
			%\begin{tabular}{|c|c|c|c|c|}
%\hline
 %&\multicolumn{2}{|c|}{Случайный лес}&\multicolumn{2}{|c|}{Решение дерево}\\
%\cline{2-5}
 %&Глубина:2 &Глубина:5  &Глубина:2 &Глубина: 5 \\ \hline 
%Время обучения (sec)&248&371&301&657\\
%MSE             &0.206&0.370&0.388&0.469 \\ \hline 
%\end{tabular}
%\caption{Время обучения, истинная скорость и проверка ошибок алгоритмов классификации: случайный лес и решение дерева} \label{tab12}
		%\end{table}
	%
%\begin{figure}[ht!]
%\centering
%\includegraphics[width=0.8\linewidth]{pic44}
%\caption{Результаты классификации базы данных <<TrueTear>>.}
	%\label{pic44}
		%\end{figure}
		%
%Результаты опыта показали, что оба алгоритма основаны на идее алгоритма CART, метода Bagging и  условия одинаковой глубины. Алгоритм случайного леса классификатора лучше, чем решение дерево, потому что алгоритм случайного леса построен из множества решений деревьев. Он может обрабатывать много вероятных случаев. Таким образом, он может эффективно работать с данными, число образцов больше, чем количество функций.
%%------------3.5---------------%
\section{Основные результаты и выводы по главе 3}

\begin{enumerate}
	\item В этой главе описывается структура системы для обнаружения и классификации объектов и классов данных, соответствующих блокам: ввода и вывода данных, предварительная обработка изображений, сегментация изображений, извлечение признаков, классификация объектов.
\item Проведены эксперименты по классификации дефектов дорожного покрытия и классификация пузырьков. Эти эксперименты показали, что предлагаемые алгоритмы обладают, с высокой эффективностью в том числе на изображениях с шумом и ограниченным освещением, что особенно важно для их практического использования.
\item Проведен анализ и сравнение с результатами анализа изображений и алгоритмов сегментации изображений другими методами, сделано заключение, что успех комбинации методов сегментации изображений основан на методе разреза на графах и алгоритме Марковского случайного поля.
\item Результаты экспериментов продемонстировали эффективность классификации различных типов дефектов. Сравнение полученных результатов с результатами приложения с других алгоритмов машинного обучения подтвердило стабильность системы, точность и приемлемое время выполнения.

\end{enumerate}

%%%%%%%%%%%%%%%%%%%%%%%  ГЛАВА 4 %%%%%%%%%%%%%%%%%%%%%%%%%%%%%%%%%%%%%%%%%
\chapter{Программное обеспечение для обнаружения и классификации объектов на изображениях} \label{chapt4}
В этой главе представлена среда и инструменты для разработки программного обеспечения для обнаружения и классификации объектов систем ОКДДП и ОКФП. Описаны интерфейс и результаты каждого программного обеспечения.
%%---------4.1-------------%
\section{Среда разработки программного обеспечения}
Программное обеспечение написано на языке программирования Matlab в программном пакете Matlab 7.0, а также при поддержке библиотеки обработки изображений Intel OpenCV.
%-------------4.1.1
\subsection{Matlab и панель инструментов для обработки изображений}
Matlab - это высокоуровневый язык программирования, используемый для решения технических проблем. 

Matlab интегрирует вычисления, показывает результаты, позволяет программным интерфейсам очень легко  работать при использовании. Данные предварительно запрограммированной библиотеки позволяют пользователям получать следующие применения:

\begin{itemize}
	\item Позволяет программистам создавать новые приложения.
 \item Позволяет моделировать фактическую модель.
\item Анализ, просмотр и отображение данных.
\item Расширенная поддержка графического программного обеспечения.
\item Позвольте разрабатывать, общаться с каким-либо другим программным обеспечением, таким как C++, Fortran.

\end{itemize}

%%---------4.1.2-------------%
\subsection{Библиотека обработки изображений OpenCV}
OpenCV \cite{h149} - библиотека компьютерного  зрения с открытым исходным кодом. Эта библиотека написана на C и C ++ и может работать на платформах Linux, Windows и Mac OS X. Кроме того, она разработана в Python, Ruby, Matlab и некоторых других языках

Одной из целей OpenCV является упрощение того, что возможно в области компьютерного зрения, чтобы помочь пользователям быстро создавать надежные и сложные приложения. Библиотека OpenCV имеет более 500 функций и делится на множество визуальных полей, таких как безопасность, здравоохранение, робототехника.

OpenCV структурирован из пяти основных компонентов, четыре из которых представлены на рисунке \ref{pic50}. Компоненты CV содержат основные алгоритмы обработки изображений и усовершенствованные алгоритмы обработки компьютерного зрения; ML - это библиотека, работающая в области машинного обучения, включающая в себя множество статистических классов и инструментов кластеризации. HighGUI содержит компоненты и функции импорта и хранения изображений и видео, а CXCore содержит контент и основные структуры данных.
\begin{figure}[ht!]
\centering
\includegraphics[width=0.7\linewidth]{pic50}
\caption{Струтура библиотеки OpenCV.}
	\label{pic50}
		\end{figure}

\textit{Open CV имеет следующие основные модули:}

-- Opencv core: обеспечивает основную функциональность. Включает в себя базовые структуры, вычисления, ввод и вывод для XML и т.д;

-- Opencv imgproc: обработка изображений;

-- Opencv highgui: модуль для создания пользовательского интерфейса;

-- Opencv feature2d: распознавание и описание плоских примитивов (SURF, FASR и др.);

-- Opencv video:анализ движения и отслеживание объектов;

-- Opencv objdetect: обнаружение объектов на изображении;

-- Opencv ml:модели машинного обучения.
%-----------4.2
\section{Программа обнаружения и классификации объектов на изображениях}
Программа реализована на языке Matlab на основе методов обработки изображений и алгоритмов машинного обучения.  
%-------4.2.1
\subsection{Применение для обнаружения и классификации дефектов дорожного покрытия}
Входные данные программы являются изображениями  дорожного покрытия. Выходными данными являются результаты анализа изображений, обнаружения и классификации дефектов. Программа позволяет пользователям использовать автоматическое обнаружение местоположения дефектов на дороге, извлечение признаков для сортировки и показа результатов. Интерфейс пользователя представлен на рис. \ref{pic48}, имеет следующие функции: 

\begin{itemize}
	\item <<Load image>>: загрузка данных (изображения дорожного покрытия); 
	\item <<Select>>: выбор областей для анализа; 
	\item <<Pre-processing>>: предварительная обработка изображения (преобразование на полутоновое изображение, фильтрация шумов,...) для улучшения качества изображений; 
		\item <<Segmentation>>: сегментация изображение; 
	\item <<Feature extraction>>: извлечение признаков в изображениях; 
	\item <<Classification>>: классификация дефектов дорожного покрытия.
\end{itemize}


\begin{figure}[ht!]
\centering
\includegraphics[width=0.7\linewidth]{pic48}
\caption{Интерфейс пользователи программы ОКДДП.}
	\label{pic48}
		\end{figure}
		
Классификация дефектов: существует 3 типа основных дефектов, таких как выбоины, блок трещины, продольные трещины, они соответственно помечены 1 (рис. \ref{pic49}), 2 (рис. \ref{pic54}), 3 (рис. \ref{pic55}). Например, если результат классификации с вектором признаков обозначен как 1, значит дефект дорожного покрытия – выбоины (рис.\ref{pic49}).  

\begin{figure}[ht!]
\centering
\includegraphics[width=0.7\linewidth]{pic49}
\caption{Результат «выбоина» при обнаружении и классификации дефетов дорожного покрытия.}
	\label{pic49}
		\end{figure}
		
\begin{figure}[ht!]
\centering
\includegraphics[width=0.7\linewidth]{pic54}
\caption{Результат «блок трещина» при обнаружении и классификации дефетов дорожного покрытия.}
	\label{pic54}
		\end{figure}
		
\begin{figure}[ht!]
\centering
\includegraphics[width=0.7\linewidth]{pic55}
\caption{Результат «продольные трещины» при обнаружении и классификации дефектов дорожного покрытия.}
	\label{pic55}
		\end{figure}

%-------4.2.2
\subsection{Применение для обнаружения и классификации формы пузырьков}
Программное обеспечение для обнаружения и классификации пузырьков построено с использованием языка Matlab и библиотеки OpenCV.
Основные функции программы (рис. \ref{pic51}) включают:

\begin{figure}[ht!]
\centering
\includegraphics[width=0.7\linewidth]{pic51}
\caption{Интерфейс пользователи программы ОКФП.}
	\label{pic51}
		\end{figure}
		
\begin{itemize}
	\item <<Image Source>> Загрузить данные (рис. \ref{pic52}):

\begin{itemize}
	\item  <<Demo image>> Изображение доступно в системе.
  \item <<Browser image>> изображения браузера.
\end{itemize}

\item <<Process>> Процессы программы:

\begin{itemize}
	\item <<Pre-processing>> выполняет предварительную обработку.
  \item <<Segmentation and detection of bubbles>> выполняет сегменты, обнаруживает пузырьки.
  \item <<Classification of shape bubbles>> извлекает признаки пузырьков, классифицирует форму пузырьков.
\end{itemize}

\item <<Start>> для отображения результатов обработки.
\item <<Exit>> для вывода программы.

\end{itemize}

\begin{figure}[ht!]
\centering
\includegraphics[width=0.7\linewidth]{pic52}
\caption{Функция <<browser image>> изображения браузера.}
	\label{pic52}
		\end{figure}
Результаты системы классификации формы пузырьков с целью классификации помеченных только двумя пересекающимися пузырьками. Результат программы проиллюстрирован на рисунке \ref{pic53}.
\begin{figure}[ht!]
\centering
\includegraphics[width=0.7\linewidth]{pic53}
\caption{Результат при обнаружении и классификации формы пузырьков.}
	\label{pic53}
		\end{figure}
%----------4.3-------------%
\section{Основные результаты и выводы по главе 4}

\begin{enumerate}
\item В этой работе представлены инструменты для разработки систем распознавания и классификации объектов: язык программирования Matlab и библиотека OPENCV с открытым исходным кодом.
\item Предоставляются инструкции для пользователей программ.
\item Представление интерфейса и основных функций системы обнаружения и классификации дефектов дорожного покрытия включают в себя следующие функции: ввод данных, предварительная обработка изображений, сегментация изображения и построение карты дефектов дорожного покрытия, извлечение признаков, классификация типов дефекты.
\item Дано описание интерфейса и основных функций системы для обнаружения и классификации форм пузырьков: ввод данных, предварительная обработка изображений, сегментация изображения и обнаружение пузырьков, извлечение признаков, классификация формы пузырьков.
\item Программное обеспечение было разработано в Matlab для персональных компьютеров. Программный модуль имеет простой, удобный и интуитивно понятный интерфейс.
\end{enumerate}


%%%%%%%%%%%%%%%%%%%%%% CONCLUSION %%%%%%%%%%%%%%%
\chapter*{Заключение}
\addcontentsline{toc}{chapter}{Заключение}
%\Conclusion
В диссертационной работе решена актуальная научно-техническая задача исследования и разработки алгоритмов обнаружения и классификации объектов на статических изображениях.

В результате выполнения диссертационной работы были получены следующие основные научные и практические результаты и сделаны следующие выводы:

1.	Предложены и разработаны математическая модель и численный метод на основе комбинации алгоритмов разреза на графах и Марковского случайного поля для сегментации изображений для построения карты дефектов дорожного покрытия в системе ОКДДП; использована комбинация метода разреза на графах с вейвлет-преобразованием Хаара для обнаружения пузырьков в системе ОКФП.

2.	Проведен анализ и извлечение соответствующих признаков для разных объектов в каждой системе, особое внимание уделено признаку текстуры, геометрическому признаку и вейвлет-преобразованию Хаара.

3.	Разработан метод машинного обучения на основе алгоритма случайного леса для разработки модели классификации объектов в системах ОКДДП и ОКФП.

4.	На основе экспериментов проведен анализ и сравнение фактических результатов обнаружения и классификации дефектов дорожного покрытия в системе ОКДДП; обнаружения и классификации формы пузырьков в системе ОКФП. Результаты эксперимента показывают, что алгоритмы работают на высокой скорости, точности и стабильности системы, обеспечивая работу с шумом и в режиме реального времени.

5.	Созданы программные модули ОКДДП и ОКФП, которые используются для обнаружения и классификации объектов на изображениях в режиме реального времени.


%%%%%%%%%%%%%
\include{part5_references/references_phd}      % Список литературы
\include{other_parts/lists}           % Списки таблиц и изображений (иллюстративный материал)
\chapter*{Приложения}
\addcontentsline{toc}{chapter}{Приложения}
%-------------------------------------------------------
\begin{center}
Приложение А  \\
Акт о внедрении программного продукта
\end{center}
\begin{figure}[ht!]
\centering
\includegraphics [width=0.8\linewidth] {images/h58.png}\label{img58}
\end{figure}
%------------------------------------------------------
\begin{center}
Приложение Б\\
Свидетельства о государственной регистрации программы для ЭВМ
\end{center}
\begin{figure}[ht!]
\centering
\includegraphics [width=0.9\linewidth] {images/p31.png}\label{imgp31}
\end{figure}
%--------
\begin{figure}[ht!]
\centering
\includegraphics [width=1\linewidth] {images/p25.png}\label{imgp25}
\end{figure}

%--------------------------------------------------------
\newpage
\begin{center}
Приложение В\\
Дипломы
\end{center}
\begin{figure}[ht!]
\centering
\includegraphics [width=0.8\linewidth] {images/p29.png}\label{imgp29}
\end{figure}
%----------
\begin{figure}[ht!]
\centering
\includegraphics [width=0.8\linewidth] {images/p27.png}\label{imgp27}
\end{figure}
%----------
\begin{figure}[ht!]
\centering
\includegraphics [width=0.8\linewidth] {images/p28.png}\label{imgp28}
\end{figure}
%-------
\begin{figure}[ht!]
\centering
\includegraphics [width=0.8\linewidth] {images/p30.png}\label{imgp30}
\end{figure}
%--------
\begin{figure}[ht!]
\centering
\includegraphics [width=0.8\linewidth] {images/p26.png}\label{imgp26}
\end{figure}

%\include{other_parts/appendix}        % Приложения
%\include{other_parts/symb_index}		% Список обозначений

\end{document}
