\documentclass[14pt]{extarticle}
\usepackage{lipsum}
\usepackage[a4paper]{geometry}

\usepackage{fontspec}
\setmainfont{Times New Roman}
\usepackage{indentfirst}

\usepackage[babelshorthands]{polyglossia}
\setdefaultlanguage{russian}
\setotherlanguage{english}
\geometry{a4paper,margin={2cm,2cm}}
\def\baselinestretch{1.1}

\begin{document}
\thispagestyle{empty}

\vspace{1em}
\begin{center}
  \large\textbf{ОТЗЫВ}
\end{center}
\begin{quote}
  \noindent рецензента канд.техн.наук Черкашина Евгения Александровича.

  \noindent на диссертацию \textbf{Нгуен Тху Хыонг} на тему \textbf{<<Применение технического зрения и машинного обучения для выявления и классификации дефектов>>}, представленную на соискание ученой степени кандидата технических наук по специальности \textbf{05.13.18 — <<Математическое моделирование, численные методы и комплексы программ>>}
\end{quote}


На отзыв представлена диссертация, выполненная в Федеральном государственном бюджетном образовательном учреждении высшего образования «Национальный исследовательский иркутский технический университет».  Работа изложена на 113 страницах, состоит из из введения, четырех глав, заключения, списка литературы из 114 наименований, списков сокращений, иллюстраций, таблиц, и одного приложения.  Иллюстративный материал состоит из 54 рисунков и 11 таблиц.

Автореферат диссертационной работы представлен на 16 страницах.

\section{Актуальность темы диссертационной работы}

Автоматизация процесса проверки качества дорожного покрытия -- важная задача, связанная с повышением производительности труда дорожных служб.  Для построения сметы расходов на ремонт/реконструкцию дорожного покрытия важно не только распознать наличие нарушения целостности покрытия, но и правильно его классифицировать, а также оценить физические параметры (величину, площадь и т.~п.).  Большинство современных программных систем анализа поверхности не обеспечивает полный цикл обнаружения и оценки параметров нарушений, поэтому исследования и разработка программных технологий, направленных на решение данной задачи, повышающих качество результата являются актуальными исследовательскими проблемами.  Аналогично, задача обнаружения инородных тел в жидкостях в автоматическом режиме также направлена на решение технических проблем эксплуатации оборудования, повышение качества результата.

В работе рассматривается проблема создания математического и программного обеспечения для автоматизации решения задач распознавания объектов на динамических изображениях в режиме, близком к реальному времени.  Результаты исследований в рамках этой проблемы преследуют целью расширить класс применимости новых методов компьютерного обучения задачами распознавания, классификации объектов, а также оценкой их физических характеристик.  \textbf{(практическая значимость исследования)}.

Результаты рецензируемой работы представляют собой программную технологию анализа изображений, позволяющую разрабатывать системы распознавания объектов и измерения их параметров, реализующие полный цикл обработки информации по следующей схеме: предобработка, сегментация, анализ и извлечение признаков, формирование хранилища изображений.  Для каждого из этапов проведены оценки производительности.  Результаты апробированы на решении практических задач.

\section{Научная новизна диссертационной работы}
\label{sec:sci-new}

В процессе выполнения диссертационной работы автором получены следующие научные результаты:
\begin{enumerate}
\item

\end{enumerate}

\section{Практическое значение предложений и выводов диссертационной работы}
\label{sec:prec-val}

В диссертации Нгуен Тху Хыонг решается задача создания инструментария для разработки программных систем компьютерного зрения и машинного обучения, предназначенного для распознавания, классификации и оценки физических параметров на \textbf{динамическом изображении}.



Разработанная база знаний, концептуальная модель предметной области электротехнических схем, представляет собой программный каркас (framework) для проектирования ПО численного моделирования, позволяющий решать как прямые, так и обратные задачи в этой предметной области.

Результаты диссертации применимы в научных исследованиях институтов Иркутского научного центра СО РАН, а именно, в СИФИБР СО РАН, ИСЗФ СО РАН, ИЗК СО РАН, ЛИН СО РАН для автоматизации обнаружения объектов на изображениях, полученных с микроскопов и спутников, в медицинских учреждениях (например, ИООД) на этапах контроля качества анализов.


\section{Замечания}
\begin{enumerate}
\item В тексте обнаружены ... и ошибки пунктуации.
\item \textbf{В работе не уделено внимание задачам реализации вариантов вычислительного процесса на специализированных аппаратных вычислительных устройствах, например, CUDA.}

\end{enumerate}



\end{document}


%%% Local Variables:
%%% TeX-engine: luatex
%%% TeX-master: t
%%% End:
