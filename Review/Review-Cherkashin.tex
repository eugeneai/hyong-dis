\documentclass[12pt]{extarticle}
\usepackage{lipsum}
\usepackage[a4paper]{geometry}

\usepackage{fontspec}
\setmainfont{Times New Roman}
\newfontfamily{\cyrillicfonttt}{Linux Biolinum O}
\setsansfont{Linux Biolinum O}
\usepackage{indentfirst}
\usepackage{hyperref}

\usepackage[babelshorthands]{polyglossia}
\setdefaultlanguage{russian}
\setotherlanguage{english}
\geometry{a4paper,margin={2cm,2cm}}
\def\baselinestretch{1.1}
\hypersetup{
  pdfkeywords={},
  pdfsubject={},
  pdfcreator={Emacs 25.3.1 (Org mode 8.2.10)}}
\tolerance=1000

\begin{document}
\thispagestyle{empty}

\vspace{1em}
\begin{center}
  \large\textbf{ОТЗЫВ}
\end{center}
\begin{quote}
  \noindent рецензента канд.техн.наук Черкашина Евгения Александровича.

  \noindent на диссертацию \textbf{Нгуен Тху Хыонг} на тему \textbf{<<Применение технического зрения и машинного обучения для выявления и классификации дефектов>>}, представленную на соискание ученой степени кандидата технических наук по специальности \textbf{05.13.18 — <<Математическое моделирование, численные методы и комплексы программ>>}
\end{quote}


На отзыв представлена диссертация, выполненная в Федеральном государственном бюджетном образовательном учреждении высшего образования «Национальный исследовательский иркутский технический университет».  Работа изложена на 113 страницах основного текста, состоит из из введения, четырех глав, заключения, списка литературы из 114 наименований, списков сокращений, иллюстраций, таблиц, и одного приложения.  Иллюстративный материал состоит из 54 рисунков и 11 таблиц.

Автореферат диссертационной работы представлен на 16 страницах.

\section{Актуальность темы диссертационной работы}

Автоматизация процесса проверки качества дорожного покрытия -- важная задача, связанная с повышением производительности труда дорожных служб.  Для построения сметы расходов на ремонт/реконструкцию дорожного покрытия важно не только распознать наличие нарушения целостности покрытия, но и правильно его классифицировать, а также оценить физические параметры (величину, площадь и т.~п.).  Большинство современных программных систем анализа поверхности не обеспечивает полный цикл обнаружения и оценки параметров нарушений, поэтому исследования и разработка программных технологий, направленных на решение данной задачи, повышающих качество результата являются актуальными исследовательскими проблемами.  Аналогично, задача обнаружения инородных тел в жидкостях в автоматическом режиме также направлена на решение технических проблем эксплуатации оборудования, повышение качества результата.

В работе рассматривается проблема создания математического и программного обеспечения для автоматизации решения задач распознавания объектов на динамических изображениях в режиме, близком к реальному времени.  Результаты исследований в рамках этой проблемы преследуют целью расширить класс применимости новых методов компьютерного обучения задачами распознавания, классификации объектов, а также оценкой их физических характеристик.  % FIXME: \textbf{(практическая значимость исследования)}.

Результаты рецензируемой работы представляют собой программную технологию анализа изображений, позволяющую разрабатывать системы распознавания объектов и измерения их параметров, реализующие полный цикл обработки информации по следующей схеме: предобработка, сегментация, анализ и извлечение признаков, формирование хранилища изображений.  Для каждого из этапов проведены оценки производительности.  Результаты апробированы на решении практических задач.

\section{Научная новизна диссертационной работы}
\label{sec:sci-new}

В процессе выполнения диссертационной работы автором получены следующие научные результаты:
\begin{enumerate}
\item Преложены новые варианты алгоритмов, моделирующих процессы обнаружения объектов на динамических изображениях, классификации этих объектов и оценки их физических параметров.  Алгоритмы настроены на решение двух вышеупомянутых прикладных задач.
\item Разработаны алгоритмы улучшения изображения, учитывающие наличие в исходных данных определенного уровня шума.
\item Построены методы машинного обучения, основанные на численных алгоритмах, а именно, комбинации марковских случайных полей и разрезов на графах, а также алгоритма случайного леса.
\item Методы вейвлетных преобразований применены в решении задачи обнаружения пузырьков. Предложены соответствующие алгоритмы и технологии.
\item Разработанные методы и алгоритмы реализованы в виде программной библиотеки.
\end{enumerate}

\section{Практическое значение предложений и выводов диссертационной работы}
\label{sec:prec-val}

В диссертации Нгуен Тху Хыонг решается задача создания инструментария для разработки программных систем компьютерного зрения и машинного обучения, предназначенного для распознавания, классификации и оценки физических параметров на динамическом изображении.  Инструментарий представляет собой %\textbf{настраиваемый}
программный модуль системы \texttt{MATLAB}.  Программное обеспечение успешно протестировано на тестовых примерах, представляющих собой динамические изображения физических объектов.  Результаты тестов показали работоспособность предложенных технологий.   Результаты исследований применены в ИСЭМ СО РАН, получен акт о внедрении, а также три охранных свидетельства о Государственной регистрации программ для ЭВМ.

Результаты диссертации применимы в научных исследованиях институтов Иркутского научного центра СО РАН, а именно, в СИФИБР СО РАН, ИСЗФ СО РАН, ИЗК СО РАН, ЛИН СО РАН для автоматизации обнаружения объектов на изображениях, полученных с микроскопов и спутников, в медицинских учреждениях (например, ИООД) на этапах контроля качества анализов.

\section{Обоснованность и достоверность основных положений и выводов}
\label{sec:verification}

Результаты и выводы, полученные в ходе выполнения диссертационной работы, основаны на применении современных алгоритмов машинного обучения и компьютерного зрения, сравнительным анализом результатов моделирования с существующими подходами.  Достоверность подтверждается публикациями в журналах рекомендованных ВАК и рецензируемых изданиях, а также успешным применением разработанных технологий для решения прикладных задач.

Результаты работы докладывались на \textbf{девяти} конференциях различного уровня, а также обсуждалась на семинарах ИСЭМ СО РАН и Института высоких технологий ИРНИТУ.

\section{Публикация результатов диссертационной работы}
\label{sec:publ}

Результаты диссертационной работы представлены в 16 научных работах, включающих 8 статей в изданиях из перечня ВАК и трех свидетельств (ФИПС) о государственной регистрации программы для ЭВМ, 5 статей, опубликованных в других изданиях.  Публикации в полной мере отражают основное содержание диссертационной работы.

\section{Структура и содержание диссертации}

Во \textbf{введении} обосновывается актуальность темы, формулируется цель, ставятся задачи и кратко описывается содержание работы. \textbf{Первая глава} рассматривает общие вопросы применения компьютерного зрения и машинного обучения в решении задач обнаружения, распознавания и оценки физических параметров объектов на динамических изображениях.  Поставлены цели и задачи диссертации. Преложен общий поход к решению поставленных задач.

Во \textbf{второй главе} рассмотрены математические модели и численные методы решения задач диссертации на основе машинного зрения.  Предложена схема обработки информации в виде последовательности, включающей предварительную обработку, сегментацию, обнаружение и классификация объектов, а также оценка физических параметров этих объектов.  Далее, каждый этап рассматривается как отдельная задача в контексте каждой из двух проблем.

В \textbf{третьей главе} приводится описание реализации алгоритмов и оценке их производительности.

В \textbf{четвертой главе} представлены требования к аппаратному обеспечению рабочей станции, где запускается разработанное программное обеспечение, а также оценки производительности программ и качества результатов.

В \textbf{Заключении} приводится перечень полученных результатов в виде пяти защищаемых положений.

\section{Замечания}
В тексте автореферата выявлены следующие замечания:
\begin{enumerate}
\item В разделе <<Методы исследования>> автореферата в теоретической части необходимо обратить внимание на необходимость настройки существующих методов на специфику поставленных задач.
\item В тексте автореферата надо явно связать упоминание о наличии <<авторского метода>> с его описанием (см. абзац для п. 2.1), иначе трудно ориентироваться в представленной информации.
\item В тексте обнаружены орфографические ошибки и ошибки пунктуации.
\item Необходимо исправить (минимизировать) отклонения от ГОСТ оформления диссертаций, например, в заголовках алгоритмов вместо <<Алгоритм 1:>> необходимо использовать <<Алгоритм 1 -->>.
\item Рисунки 3 и 4 во многом схожи.  Имеет смысл заменить их одной общей схемой, а выделенное место использовать для более подробного изложения материала.
\item Представление материал третьей главы в автореферате необходимо сделать более детальным.
\item Защищаемые положения 2 и 3 надо об объединить в одно, причем суть (результат) пункта 2 до конца не ясна.
\end{enumerate}

Замечания по тексту диссертации во многом аналогичны вышеуказанным замечаниям.
\begin{enumerate}
\item Текст диссертации требует значительной доработки в части грамматики русского языка.
\item Таблица 1.1 подписана не по ГОСТ.
\item
\end{enumerate}

В работе не уделено внимание задачам реализации вариантов вычислительного процесса на специализированных аппаратных вычислительных устройствах, например, CUDA.  Вычислительные устройства типа CUDA обладают хорошим соотношением производительности и стоимости, и исследование реализуемости представленных в диссертации алгоритмов позволило бы определить степень применимости таких устройств в решении поставленных задач.  В случае успеха возможно получение новых оценок производительности и качества решения алгоритмов машинного обучения.

\section{Общая характеристика работы}

В целом, несмотря на отмеченные недостатки, диссертационная работа содержит достаточно нового материала, чтобы квалифицировать ее как завершенное научное исследование по актуальной теме.  Результаты диссертации обладают научной новизной и практической значимостью.

% Представить, что замечательного.

Предметная область исследования -- популярное направление в области исследований искусственного интеллекта. Необходимо отметить, что в таких условиях в работе представлен качественный литературный обзор, причем много внимания уделяется российским исследованиям.

Основные результаты диссертации опубликованы в открытой печати: с статьях и изданиях, включенных в список ВАК, в трудах ряда всероссийских и международных конференций. %Автореферат диссертации в полной мере раскрывает содержание представленной работы.

Диссертация соответствует паспорту специальности 05.13.18 -- <<Математическое моделирование, численные методы и комплексы программ>>, т.~к. в ней обоснованы оригинальные результаты одновременно из трех областей:
\begin{itemize}
\item математическое моделирование -- представлены новые варианты алгоритмов машинного зрения и компьютерного обучения, направленные на решение определенных классов задач.

\item численные методы -- разработаны новые варианты реализаций алгоритмов машинного обучения в виде специальных вариантов и комбинаций существующих алгоритмов, а также их идентификации.

\item комплексы программ -- разработано новое программное обеспечение (в виде приложений на платформе Windows 10 / MATLAB), реализующее предложенные методы и алгоритмы.
\end{itemize}

\section*{Заключение}
\label{sec-conc}


Таким образом, диссертация Нгуен Тху Хыонг является
% качественной
научно"=квалификационной работой, выполнененной на актуальною тему, носит вполне законченный характер, содержит новые научные результаты, обладающие практической полезностью, т.~е. удовлетворяет требованиям ВАК, предъявляемым к кандидатским диссертациям.  Тема диссертации соответствует требованиям, предъявляемым к диссертациям на соискание степени кандидата технических наук по специализированных 05.13.18 -- <<Математическое моделирование, численные методы и комплексы программ>>.


% Таким образом, диссертация Нгуен Тху Хыонг является завершенной научно-квалификационной работой, выполнена на актуальною тему, носит законченный характер, содержит новые научные результаты, обладающие практической полезностью, т.е. удовлетворяет требованиям ВАК, предъявляемым к кандидатским диссертациям.  Диссертант соответствует требованиям, предъявляемым к научным работникам, и заслуживает присуждения ему ученой степени кандидата технических наук по специальности 05.13.18 -- <<Математическое моделирование, численные методы и комплексы программ>>.

\vspace{2em}
\noindent{}Рецензент диссертационной работы --\\
старший научный сотрудник\\
Лаборатории комплексных информационных систем\\
ФГБУН Институт динамики систем и теории\\
управления им. В.М. Матросова СО РАН\\
кандидат технических наук,\\
доцент \hfil Черкашин Евгений Александрович\linebreak{}{}\\[0.5em]
30 марта 2018 года\\[0.5em]

\noindent{}Сведения о рецензенте: Черкашин Евгений Александрович\\[0.5em]
Почтовый адрес:\\
664033, г. Иркутск, ул. Лермонтова, 134\\[0.2em]
тел:. +7 (914) 870 67 54\\[0.2em]
e-mail: \href{mailto:eugeneai@irnok.net}{\nolinkurl{eugeneai@irnok.net}}


\end{document}


%%% Local Variables:
%%% TeX-engine: luatex
%%% TeX-master: t
%%% End:
