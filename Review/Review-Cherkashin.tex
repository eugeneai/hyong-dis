\documentclass[14pt]{extarticle}
\usepackage{lipsum}
\usepackage[a4paper]{geometry}

\usepackage{fontspec}
\setmainfont{Times New Roman}
\newfontfamily{\cyrillicfonttt}{Linux Biolinum O}
\setsansfont{Linux Biolinum O}
\usepackage{indentfirst}
\usepackage{hyperref}

\usepackage[babelshorthands]{polyglossia}
\setdefaultlanguage{russian}
\setotherlanguage{english}
\geometry{a4paper,margin={2cm,2cm}}
\def\baselinestretch{1.1}
\hypersetup{
  pdfkeywords={},
  pdfsubject={},
  pdfcreator={Emacs 25.3.1 (Org mode 8.2.10)}}
\tolerance=1000

\begin{document}
\thispagestyle{empty}

\vspace{1em}
\begin{center}
  \large\textbf{ОТЗЫВ}
\end{center}
\begin{quote}
  \noindent рецензента канд.техн.наук Черкашина Евгения Александровича.

  \noindent на диссертацию \textbf{Нгуен Тху Хыонг} на тему \textbf{<<Применение технического зрения и машинного обучения для выявления и классификации дефектов>>}, представленную на соискание ученой степени кандидата технических наук по специальности \textbf{05.13.18 — <<Математическое моделирование, численные методы и комплексы программ>>}
\end{quote}


На отзыв представлена диссертация, выполненная в Федеральном государственном бюджетном образовательном учреждении высшего образования «Национальный исследовательский иркутский технический университет».  Работа изложена на 113 страницах, состоит из из введения, четырех глав, заключения, списка литературы из 114 наименований, списков сокращений, иллюстраций, таблиц, и одного приложения.  Иллюстративный материал состоит из 54 рисунков и 11 таблиц.

Автореферат диссертационной работы представлен на 16 страницах.

\section{Актуальность темы диссертационной работы}

Автоматизация процесса проверки качества дорожного покрытия -- важная задача, связанная с повышением производительности труда дорожных служб.  Для построения сметы расходов на ремонт/реконструкцию дорожного покрытия важно не только распознать наличие нарушения целостности покрытия, но и правильно его классифицировать, а также оценить физические параметры (величину, площадь и т.~п.).  Большинство современных программных систем анализа поверхности не обеспечивает полный цикл обнаружения и оценки параметров нарушений, поэтому исследования и разработка программных технологий, направленных на решение данной задачи, повышающих качество результата являются актуальными исследовательскими проблемами.  Аналогично, задача обнаружения инородных тел в жидкостях в автоматическом режиме также направлена на решение технических проблем эксплуатации оборудования, повышение качества результата.

В работе рассматривается проблема создания математического и программного обеспечения для автоматизации решения задач распознавания объектов на динамических изображениях в режиме, близком к реальному времени.  Результаты исследований в рамках этой проблемы преследуют целью расширить класс применимости новых методов компьютерного обучения задачами распознавания, классификации объектов, а также оценкой их физических характеристик.  \textbf{(практическая значимость исследования)}.

Результаты рецензируемой работы представляют собой программную технологию анализа изображений, позволяющую разрабатывать системы распознавания объектов и измерения их параметров, реализующие полный цикл обработки информации по следующей схеме: предобработка, сегментация, анализ и извлечение признаков, формирование хранилища изображений.  Для каждого из этапов проведены оценки производительности.  Результаты апробированы на решении практических задач.

\section{Научная новизна диссертационной работы}
\label{sec:sci-new}

В процессе выполнения диссертационной работы автором получены следующие научные результаты:
\begin{enumerate}
\item

\end{enumerate}

\section{Практическое значение предложений и выводов диссертационной работы}
\label{sec:prec-val}

В диссертации Нгуен Тху Хыонг решается задача создания инструментария для разработки программных систем компьютерного зрения и машинного обучения, предназначенного для распознавания, классификации и оценки физических параметров на \textbf{динамическом изображении}.

Разработанная база знаний, концептуальная модель предметной области электротехнических схем, представляет собой программный каркас (framework) для проектирования ПО численного моделирования, позволяющий решать как прямые, так и обратные задачи в этой предметной области.

Результаты диссертации применимы в научных исследованиях институтов Иркутского научного центра СО РАН, а именно, в СИФИБР СО РАН, ИСЗФ СО РАН, ИЗК СО РАН, ЛИН СО РАН для автоматизации обнаружения объектов на изображениях, полученных с микроскопов и спутников, в медицинских учреждениях (например, ИООД) на этапах контроля качества анализов.

\section{Обоснованность и достоверность основных положений и выводов}
\label{sec:verification}

Результаты и выводы, полученные в ходе выполнения диссертационной работы, основаны на ..., сравнительным анализом результатов моделирования с \textbf{существующими подходами}.  Достоверность подтверждается публикациями в журналах рекомендованных ВАК и рецензируемых изданиях, а также успешным применением разработанных технологий для решения прикладных задач.

Результаты работы докладывались на ... конференциях различного уровня, а также на ежегодных научно"=практических конференциях \textbf{Восточно"=Сибирского государственного университета технологий и управления}.

\section{Публикация результатов диссертационной работы}
\label{sec:publ}

Результаты диссертационной работы представлены в 16 научных работах, включающих 8 статей в изданиях из перечня ВАК и трех свидетельств (ФИПС) о государственной регистрации программы для ЭВМ, 5 статей, опубликованных в других изданиях.  Публикации в полной мере отражают основное содержание диссертационной работы.

\section{Структура и содержание диссертации}

Во \textbf{введении} обосновывается актуальность темы, формулируется цель, ставятся задачи и кратко описывается содержание работы. \textbf{Первая глава} рассматривает общие вопросы применения функциональной парадигмы для



\section{Замечания}
\begin{enumerate}
\item В тексте обнаружены ... и ошибки пунктуации.
\item \textbf{В работе не уделено внимание задачам реализации вариантов вычислительного процесса на специализированных аппаратных вычислительных устройствах, например, CUDA.}
\end{enumerate}

\section{Общая характеристика работы}

В целом, несмотря на отмеченные недостатки, диссертационная работа содержит достаточно нового материала, чтобы квалифицировать ее как завершенное научное исследование по актуальной теме.  Результаты диссертации обладают научной новизной и практической значимостью.

Использование одного формализма суперпозиции конченого набора функций для представления всех необходимых структур, связанных с описанием исходной модели, процессом трансляции и порождения целевого исходного кода является оригинальным результатом диссертационной работы.  Порождаемые суперпозиции базовых функций -- программы, соответствующие алгоритмам в смысле интуитивного определения Фон Неймана, причем программ содержащих формальные параметры.  Одной из направлений дальнейшего развития результатов, полученных в диссертации, является задание интерпретации (В терминах вычисления функции) существующим элементам онтологий, используемым в Семантическом вебе.

Основные результаты диссертации опубликованы в открытой печати: с статьях и изданиях, включенных в список ВАК, в трудах ряда всероссийских и международных конференций. Автореферат диссертации в полной мере раскрывает содержание представленной работы.

Диссертация соответствует паспорту специальности 05.13.18 -- <<Математическое моделирование, численные методы и комплексы программ>>, т.~к. в ней обоснованы оригинальные результаты одновременно из трех областей:
\begin{itemize}
\item математическое моделирование -- .

\item численные методы -- .

\item комплексы программ -- .
\end{itemize}

\section{Заключение}
\label{sec-conc}

% Таким образом, диссертация Нгуен Тху Хыонг является завершенной научно-квалификационной работой, выполнена на актуальною тему, носит законченный характер, содержит новые научные результаты, обладающие практической полезностью, т.е. удовлетворяет требованиям ВАК, предъявляемым к кандидатским диссертациям.  Диссертант соответствует требованиям, предъявляемым к научным работникам, и заслуживает присуждения ему ученой степени кандидата технических наук по специальности 05.13.18 -- <<Математическое моделирование, численные методы и комплексы программ>>.

\vspace{2em}
\noindent{}Рецензент диссертационной работы --\\
старший научный сотрудник\\
Лаборатории комплексных информационных систем\\
ФГБУН Институт динамики систем и теории\\
управления им. В.М. Матросова СО РАН\\
кандидат технических наук,\\
доцент \hfil Черкашин Евгений Александрович\linebreak{}{}\\[0.5em]
30 марта 2018 года\\[0.5em]

\noindent{}Сведения о рецензенте: Черкашин Евгений Александрович\\[0.5em]
Почтовый адрес:\\
664033, г. Иркутск, ул. Лермонтова, 134\\[0.2em]
тел:. +7 (914) 870 67 54\\[0.2em]
e-mail: \href{mailto:eugeneai@irnok.net}{\nolinkurl{eugeneai@irnok.net}}


\end{document}


%%% Local Variables:
%%% TeX-engine: luatex
%%% TeX-master: t
%%% End:
