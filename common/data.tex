%%% Основные сведения %%%
\newcommand{\thesisAuthor}             % Диссертация, ФИО автора
{\texorpdfstring{\todo{\textbf{НГУЕН ТХУ ХЫОНГ}}}{Фамилия Имя Отчество автора}}  % \texorpdfstring takes two arguments and uses the first for (La)TeX and the second for pdf
\newcommand{\thesisUdk}                % Диссертация, УДК
{\todo{xxx.xxx}}
\newcommand{\thesisTitle}              % Диссертация, название
{\texorpdfstring{\todo{\MakeUppercase{%Применение интегральных моделей в задачах распознавания объектов в системах машинного зрения
%Применение интегральных моделей Вольтерра в прикладных задачах моделирования развития динамических систем}}}{Название диссертационной работы}}
%Интегральные модели Вольтерра: \newline теория и приложения в моделировании развития динамических систем}}}{Название диссертационной работы}}
%Интегральные уравнения Вольтерра: теория и приложения в моделировании развития динамических систем}}}{Название диссертационной работы}}
%Интегральные уравнения Вольтерра: численные методы и приложения в моделировании развития динамических систем}}}{Название диссертационной работы}}
%Определение стратегии сохранения и генерации электроэнергии во время пиковых нагрузок и спадов на основе интегральных моделей Вольтерра}}}{Название диссертационной работы}}
%Определение стратегии сохранения и генерации электроэнергии на основе интегральных моделей Вольтерра}}}{Название диссертационной работы}}
%Определение стратегии аккумулирования и генерации электроэнергии на основе интегральных моделей Вольтерра}}}{Название диссертационной работы}}
%Определение стратегии аккумулирования электроэнергии на основе интегральных моделей Вольтерра}}}{Название диссертационной работы}}
Применение методов компьютерного зрения и машинного обучения для выявления и классификации дефектов}}}{Название диссертационной работы}}
%Применение интегральных моделей Вольтерра в определении стратегии сохранения и генерации электроэнергии во время пиковых нагрузок}}}{Название диссертационной работы}}
\newcommand{\thesisSpecialtyNumber}    % Диссертация, специальность, номер
{\texorpdfstring{\todo{05.13.18}}{XX.XX.XX}}
\newcommand{\thesisSpecialtyTitle}     % Диссертация, специальность, название
{\texorpdfstring{\todo{Математическое моделирование,\\ численные методы и комплексы программ}}{Название специальности}}
\newcommand{\thesisDegree}             % Диссертация, научная степень
{\todo{
%кандидата физико-математических наук
кандидата технических наук
}}
\newcommand{\thesisCity}               % Диссертация, город защиты
{\todo{Иркутск}}
\newcommand{\thesisYear}               % Диссертация, год защиты
{\todo{2018}}
\newcommand{\thesisOrganization}       % Диссертация, организация
{\todo{МИНИСТЕРСТВО ОБРАЗОВАНИЯ И НАУКИ РОССИЙСКОЙ ФЕДЕРАЦИИ Федеральное государственное бюджетное образовательное учреждение высшего образования «Иркутский Национальный Исследовательский Технический Университет»}}

\newcommand{\thesisInOrganization}       % Диссертация, организация в предложном падеже: Работа выполнена в ...
{\todo{Иркутском Национальном Исследовательском Техническом Университете}}

\newcommand{\supervisorFio}            % Научный руководитель, ФИО
{\todo{Сидоров Денис Николаевич}}
\newcommand{\supervisorRegalia}        % Научный руководитель, регалии
{\todo{доктор физико-математических наук, доцент}}

\newcommand{\opponentOneFio}           % Оппонент 1, ФИО
{\todo{Фамилия Имя Отчество}}
%{\todo{Воскобойников Юрий Евгеньевич}}
\newcommand{\opponentOneRegalia}       % Оппонент 1, регалии
{\todo{доктор физико-математических наук, профессор}}
\newcommand{\opponentOneJobPlace}      % Оппонент 1, место работы
%{\todo{Сибирский государственный архитектурно-строительный университет}}
{\todo{Не очень длинное название для места работы}}
\newcommand{\opponentOneJobPost}       % Оппонент 1, должность
%{\todo{зав. каф. прикладной математики}}
{\todo{старший научный сотрудник}}

\newcommand{\opponentTwoFio}           % Оппонент 2, ФИО
{\todo{Фамилия Имя Отчество}}
%{\todo{Чистякова Елена Викторовна}}
\newcommand{\opponentTwoRegalia}       % Оппонент 2, регалии
{\todo{кандидат физико-математических наук}}
\newcommand{\opponentTwoJobPlace}      % Оппонент 2, место работы
%{\todo{Институт динамики систем и теории управления СО РАН}}
{\todo{Основное место работы c длинным длинным длинным длинным названием}}
\newcommand{\opponentTwoJobPost}       % Оппонент 2, должность
{\todo{старший научный сотрудник}}

\newcommand{\leadingOrganizationTitle} % Ведущая организация, дополнительные строки
%{\todo{Федеральное государственное бюджетное образовательное учреждение высшего образования Санкт-Петербургский государственный университет информационных технологий, механики и оптики}}
{\todo{Федеральное государственное бюджетное образовательное учреждение высшего профессионального образования с длинным длинным длинным длинным названием}}

\newcommand{\defenseDate}              % Защита, дата
{\todo{DD mmmmmmmm YYYY~г.~в~XX часов}}
\newcommand{\defenseCouncilNumber}     % Защита, номер диссертационного совета
{\todo{NN}}
\newcommand{\defenseCouncilTitle}      % Защита, учреждение диссертационного совета
{\todo{Название учреждения}}
\newcommand{\defenseCouncilAddress}    % Защита, адрес учреждение диссертационного совета
{\todo{Адрес}}

\newcommand{\defenseSecretaryFio}      % Секретарь диссертационного совета, ФИО
{\todo{Фамилия Имя Отчество}}
\newcommand{\defenseSecretaryRegalia}  % Секретарь диссертационного совета, регалии
{\todo{д-р~физ.-мат. наук}}            % Для сокращений есть ГОСТы, например: ГОСТ Р 7.0.12-2011 + http://base.garant.ru/179724/#block_30000

\newcommand{\synopsisLibrary}          % Автореферат, название библиотеки
{\todo{Название библиотеки}}
\newcommand{\synopsisDate}             % Автореферат, дата рассылки
{\todo{DD mmmmmmmm YYYY года}}

\newcommand{\keywords}%                 % Ключевые слова для метаданных PDF диссертации и автореферата
{}