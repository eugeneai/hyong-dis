%\newcommand{\actuality}{\underline{\textbf{Актуальность темы.}}}
%\newcommand{\development}{\underline{\textbf{Степень разработанности.}}}
%\newcommand{\aim}{\underline{\textbf{Целью}}}
%\newcommand{\tasks}{\underline{\textbf{задачи}}}
%\newcommand{\methods}{\underline{\textbf{Методы исследования. }}}
%\newcommand{\novelty}{\underline{\textbf{Научная новизна:}}}
%\newcommand{\theorInfluence}{\underline{\textbf{Теоретическая значимость.}}}
%\newcommand{\influence}{\underline{\textbf{Практическая значимость.}}}
%\newcommand{\defpositions}{\underline{\textbf{Основные положения, выносимые на~защиту:}}}
%\newcommand{\reliability}{\underline{\textbf{Достоверность}}}
%\newcommand{\probation}{\underline{\textbf{Апробация работы.}}}
%\newcommand{\contribution}{\underline{\textbf{Личный вклад.}}}
%\newcommand{\publications}{\underline{\textbf{Публикации.}}}

\textbf{Актуальность исследования.}
Быстрое развитие информационных технологий (ИТ) сделало возможным сбор и хранение многомерных данных для создания эффективных информационных систем. Для преобразования данных в полезные знания требуется развитие новых алгоритмов и программных средств. Таким образом, алгоритмы машинного обучения (МО) стали областью популярных исследований и приложений. МО\cite{h1, h2} -- важнейшая область искусственного интеллекта, которая имеет множество реальных приложений в различных областях. Например, поисковые инструменты Google \cite{h3}, Яндекс \cite{h4} анализируют семантику ключевых слов поиска и, используя методы МО, выдают лучшие результаты поиска или лучшие предложения, товары и услуги для конкретного пользователя. Другие приложения МО включают социальные сети VK \cite{h5} и Facebook \cite{h6}, анализирующие интересы и персональные данные клиентов и продающие их данные рекламодателям и прочим заинтересованным лицам. Анализ медицинских изображений для автоматического выявления или прогнозирования вероятности развития патологии пациента  также активно использует МО \cite{h7, h8}. Автоматическое распознавание голоса в системах безопасности \cite{h9, h10}, автоматическая классификация (изображений, видео, текста), обнаружение вирусов или мошенничества в системах информационной безопасности и безопасности в банках, акционерных обществах \cite{h11, h12} и т.д. В основе МО лежит процесс обучения на специально размеченных данных либо такая разметка происходит автоматически. Это позволяет решить конкретную проблему, построить автоматизированные системы  и даже оказывать экспертную помощь в различных областях естествознания. По сути методы МО автоматизируют построение аналитических моделей основываясь на данных. Оно использует итеративные алгоритмы для изучения данных и поиска в них ценной информации для получения новых знаний. МО не является чем-то абстрактным, а скорее конкретным вычислительным инструментом, который играет важную роль в услугах, которые люди используют каждый день. Используя преимущества методов и  алгоритмов машинного обучения и компьютерного зрения, в диссертации рассмотрены две задачи. Первая задача состояла в автоматизации обнаружения и классификации дефектов дорожного покрытия на изображениях. Вторая задача состояла в автоматизации средств изучения динамики фазовых переходов путем обнаружения и классификации пузырей в жидкости.

Для содержания и планирования ремонтов дорог, дорожные компании нуждаются в точной и своевременной информации о дефектах дорог. В мире миллионы дорог, которые необходимо проверять каждый год. Ранее периодические проверки проводились вручную инженерами. Этот метод требует много времени и небезопасен для людей. За последние несколько лет было проведено множество успешных исследований в области автоматического обнаружения и классификации дефектов дорожного покрытия \cite{h13, h14, h15, h16, h17}. При этом решалась задача оптимизации планирования проведения дорожных ремонтных работ с использованием алгоритмов построения оптимальных покрытий \cite{h150, h151, h152}. Имеются программы, которые автоматически обнаруживают и классифицируют дефекты дорожного покрытия, такие как RoadCrack vehicle \cite{h18}, система дорожной разметки ARAN \cite{h19}. Но в этих исследованиях основное внимание уделяется только дефектам типа <<трещина>>, не было проведено исследований, в которых обнаруживали различные типы дефектов, как для отдельных изображений, так и для видеопоследовательностей. Поэтому представляется необходимым повысить точность обнаружения и классификации различных типов дефектов дорожных покрытий. 

Задача обнаружения дефектов в виде пузырей возникает в таких важных областях, как изучение фазовых переходов \cite{4wl}, медицинские технологии \cite{h17bb}, фармацевтический контроль процесса \cite{h18bb}, нефтяная промышленность \cite{h19bb}. Применение методов МО позволяет разрабатывать эффективные алгоритмы обнаружения пузырьков. Особенно в двуфазных средах входные изображения весьма разнообразны и часто имеют сложный фон изображения. С другой стороны, такой объект, как пузырь, редко существуют в одиночку, они всегда взаимосвязаны или частично скрыты. Поэтому обнаружение и вычисление их характеристик - непростая задача.

Таким образом, выбор и применение алгоритмов компьютерного зрения в комбинации с методами МО для решения вышеуказанных проблем очень важны.

\textbf{\textit{Задача 1.}} Обнаружение и классификация дефектов дорожного покрытия (ОКДДП). Задача состоит в том, чтобы автоматически обнаруживать и классифицировать различные дефекты дорожного покрытия, возникающие при их эксплуатации. Входные данные это изображения дорожного полотна, результатом является расположение и тип дефекта дорожного покрытия (3 класса: глубокие трещины, сеть трещин, выбоины).

\textbf{\textit{Задача 2.}} Обнаружение и классификация формы пузырьков на изображениях (ОКФП) для автоматизации средств изучения динамики фазовых переходов. Цель задачи состоит в том, чтобы определить в заданной области количество пузырьков, их диаметр и выделить два пересекающихся пузырька. Входные данные представляют собой статическое изображение жидкости, выходное требование - это количество пузырьков и распределение их размеров.

\textbf{Цель и задачи исследования.} Целью и задачами исследования являются разработка математических моделей, численных методов с использованием методов МО для разработки программного обеспечения для обнаружения и классификации дефектов дорожного покрытия и пузырьков. 

	Для достижения целей работы необходимо выполнить \textbf{основные задачи} диссертации, а именно:
 
1) разработать основные этапы обработки и анализа изображений: предобработка, сегментация, анализ и извлечение признаков, формирование базы данных знаний. Реализовать эти этапы в виде программного модуля. Провести анализ эффективности работы разработанных модулей;
	
2) выбрать и совершенствовать методы обнаружения объектов (дефекты дорожного покрытия, пузырьки воздуха) в соответствии с различными условиями их регистрации;
	
3) разработать алгоритмы обработки изображений для извлечения признаков объектов на изображениях при наличии шума;
	
4) разработать методы и алгоритмы машинного обучения для классификации данных;
	
5) оценить качество и эффективность работы программного обеспечения на тестовых изображениях. Предложить средства для повышения эффективности работы.
 
Результаты экспериментов, изучающих разработку и применение методов, алгоритмы машинного обучения и сравнение с другими соответствующими исследованиями подтвердили эффективность исследовательской работы.


\textbf{Методы исследования.} Методы теоретических исследований: алгоритмы обнаружения и классификации объектов на изображениях. Методы прикладных исследований: исследование и применение алгоритмов и методов машинного обучения для  задачи автоматического обнаружения и классификации дефектов дорожного покрытия и задачи обнаружения и классификации пузырьков; разработка программных обеспечений; тестирование программ, оценка и анализ результатов.

\textbf{Научная новизна} результатов диссертационной работы заключается в следующем:

1) предложены методы математического моделирования для построения системы обнаружения и классификации объектов (дефектов дорожного покрытия и пузырьков) на основе технологий компьютерного зрения и методов машинного обучения. Суть этого процесса: анализ изображений, обнаружение и классификация объектов на основе их признаков;

2) разработаны алгоритмы улучшения изображений, комбинированы алгоритмы извлечения признаков с учетом таких факторов, как условия освещения, возникновение шума;

3) предложены численные методы на основе: комбинации марковских случайных полей и разрезов на графах для оптимизации сегментации изображений; машинного обучения случайного леса для классификации объектов;

4) использован признак на основе вейвлет-преобразования Хаара для обнаружения пузырьков;

5) разработанные алгоритмы и методы реализованы в виде програмного обеспечения решения следующих задач: обнаружение и классификация дефектов дорожного покрытия; обнаружение и классификация пузырьков.

\textbf{Теоретическая значимость.} Результатом исследования является комбинация алгоритмов обработки изображений и методов машинного обучения для анализа и обработки данных на изображениях в реальном времени, внешней среды и шума. 
 
%Внедрение работы. Результаты исследований были применены на практике в области управления дорожного движения в « ». Программа была отмечена в этой области и были получены сертификаты, подтверждающие точность и работоспособность приложения на практике.

\textbf{Научная ценность исследования} заключается в совершенствовании методов обнаружения и классификации объектов на изображениях. Исследование способствует развитию и внедрению методов машинного обучения и компьютерного зрения в автоматизированных системах неразрушающего контроля качества. Проведен эффективный синтез методов извлечения признаков дефектов дорожного покрытия и пузырьков.

\textbf{Практическая ценность} работы состоит в разработке эффективных методов выявления и классификации дефектов дорожного покрытия, а также в разработке средств технического зрения для мониторинга фазовых переходов.

\textbf{Апробация работы.} Работа выполнена на кафедре вычислительной техники института высоких технологий ИРНИТУ. Результаты диссертационной работы обсуждались и докладывались на следующих симпозиумах, семинарах и конференциях: Всероссийские молодежные научно-практические конференции «Винеровские чтения» (ИРНИТУ, г. Иркутск. 2014, 2015); XIX Байкальская всероссийская конференция «Информационные и математические технологии в науке и управлении» (г. Улан-Удэ. 2014); XVI Байкальская международная школа-семинар «Методы оптимизации и их приложения» (о. Ольхон, г. Иркутск. 2014); The 4th, 5th International Conference on Analysis of Images, Social Networks, and Texts (г. Екатеринбург. 2015, 2016); V Научно-практическая Internet-конференция «Междисциплинарные исследования в области математического моделирования и информатики» (г. Тольятти. 2015); XIII Всероссийская конференция молодых ученых «Моделирование, оптимизация и информационные технологии» (г. Иркутск – Старая Ангасолка. 2017); XVII Байкальская международная школа-семинар «Методы Оптимизации и их Приложения» (с.Максимиха, Бурятия. 2017). Работа выполнена при поддержке Министерства образования и подготовки кадров Социалистической Республики Вьетнам и программы развития ФГБОУ ВО ИРНИТУ.

Результаты диссертации неоднократно докладывались на научных семинарах кафедры вычислительной техники Иркутского национального исследовательского технического университета и Института систем энергетики им. Л.А. Мелентьева СО РАН.

\textbf{Личный вклад автора.} Основные результаты выносимые на защиту получены автором лично. Постановки задач и анализ результатов осуществлены совместно с Д. Н. Сидоровым. Автор благодарен А. В. Жукову, Т. Л. Нгуен и А. И. Дрегля за поддержку и ценные советы. Конфликта интересов с соавторами нет.

\textbf{Тематика работы} соответствует следующим пунктам паспорта специальности 05.13.18:

-- п.3 <<Разработка, обоснование и тестирование эффективных вычислительных методов с применением современных компьютерных технологий>>;

-- п.4. <<Реализация эффективных численных методов и алгоритмов в виде комплексов проблемно-ориентированных программ для проведения вычислительного эксперимента>>;

-- п.5 <<Комплексные исследования научных и технических проблем с применением современной технологии математического моделирования и вычислительного эксперимента>>.

\textbf{Публикации.} По теме диссертации опубликовано 16 научных работ, 8 из которых – в рецензируемых научных журналах и изданиях, рекомендованных ВАК РФ, 3 свидетельства регистрации программы на ЭВМ, 2 статьи опубликованы в журналах, индексируемых Web of Science и 3 статьи опубликованы в журналах, индексируемых Scopus.

Основные результаты исследования опубликованы в следующих работах.

\textbf{Издания, входящие в Перечень ВАК РФ}

1. Нгуен Т. Х. О распознавании и классификации дефектов дорожного покрытия на основе изображений / Т. Х. Нгуен, Т. Л. Нгуен // Вестник Иркутского гос. технического ун-та. -- 2016. -- №~10 (117). -- С. 111--118.

2. Nguyen T. H. A Robust Approach for Defects Road Pavement Detection and Classification / D. N. Sidorov, T. H. Nguyen, T. L. Nguyen // Journal of Computational and Engineering Mathematics. -- 2016. -- V. 3. -- No. 3. -- P. 40--52.

3. Nguyen T. H. On Road Defects Detection and Classification / T. H. Nguyen, T. L. Nguyen, A. Zhukov // CEUR Workshop Proceedings. -- 2016. -- V. 1710, -- P. 266--278.

4. Nguyen T. H. Machine learning algorithms application to road defects classification / T. H. Nguyen, T. L. Nguyen, D. N. Sidorov, A. I. Dreglea // Intelligent Decision Technologies. -- 2017. -- V. Preprint. -- No. Preprint. -- P. 1--8.

5. \textbf{Nguyen T.H.} Robust Approach to Detection of Bubbles Based on Images Analysis / T. H. Nguyen, T. L. Nguyen, A. I. Dreglea // International Journal of Artificial Intelligence. -- 2018. -- V. 16. -- No. 1. -- P. 167--177.

6. Нгуен Т. Х. Об автоматизации извлечения и классификации антропометрических признаков / Т. Л. Нгуен, Т. Х. Нгуен // Вестник Иркутского гос. технического ун-та. -- 2015. -- №~4 (99). -- С. 17--23.

7. Nguyen T. H. Studies of Anthropometrical Features using Machine Learning Approach / T. L. Nguyen, T. H. Nguyen, A. Zhukov // CEUR Workshop Proceedings. -- 2015. -- V. 1452. -- P. 96--105.

8. Nguyen T. H. Automatic Anthropometric System Development Using Machine Learning / T. L. Nguyen, T. H. Nguyen // BRAIN. Broad Research in Artificial Intelligence and Neuroscience. -- 2016. -- V. 7. -- P. 5--15.

 \textbf{Свидетельства о государственной регистрации программы для ЭВМ}

9. Нгуен Т. Х. Обнаружение и классификация пузырьков на цифровых изображениях / Д. Н. Сидоров, Т. Х. Нгуен, Т. Л. Нгуен // Свидетельство о гос. регистрации программы для ЭВМ. № 2018612096 от 12 февраля 2018 г. М.: Федеральная служба по интеллектуальной собственности. 2018.
 
10. Нгуен Т. Х. Программа автоматического обнаружения и классификации дефектов дорожного покрытия / Д. Н. Сидоров, Т. Х. Нгуен, Т. Л. Нгуен // Свидетельство о гос. регистрации программы для ЭВМ. № 2016619386 от 18 августа 2016 г. М.: Федеральная служба по интеллектуальной собственности. 2016.

11. Нгуен Т. Х. Программа бесконтактной антропометрии для смартфонов на операционной системе Андроид / Д. Н. Сидоров, Т. Л. Нгуен, Т. Х. Нгуен // Свидетельство о гос. регистрации программы для ЭВМ. № 2016611475 от 03 февраля 2016 г. М.: Федеральная служба по интеллектуальной собственности. 2016.
	
\textbf{Прочие издания}
	
12. Нгуен Т. Х. Автоматизация антропометрических измерений и извлечение признаков из 2D-изображений / Т. Л. Нгуен, Т. Х. Нгуен // XVI Байкальская международная школа-семинар <<методы оптимизации и их приложения>>. О. Ольхон, Иркутск 2014г. -- С. 153.

13. Нгуен Т. Х. Построение программы для обнаружения контуров человека в изображении с помощью методов математической морфологии / Т. Л. Нгуен, Т. Х. Нгуен // Материалы всероссийской молодежной научно-практической конференции <<Винеровские чтения 2014>>. Иркутск: Изд-во Иркутск, 2014. -- С. 10.

14. Нгуен Т. Х. Классификация и кластерный анализ антропометрических признаков / Т. Л. Нгуен// Материалы всероссийской молодежной научно-практической конференции <<Винеровские чтения 2015>>. Иркутск: Изд-во Иркутск, 2015. -- С. 8.

15. Нгуен Т. Х. Методы математической морфологии в цифровой обработке изображений / Т. Л. Нгуен, Т. Х. Нгуен // Труды XIX Байкальской Всероссийской конференции <<информационные и математические технологии в науке и управлении>>. Иркутск: ИСЭМ СО РАН, 2014. -- С. 75--81.

16. Нгуен Т. Х. Анализ антропометрических признаков с использованием методов машинного обучения / Т. Л. Нгуен, Т. Х. Нгуен // Междисцплинарные исследования в области математического моделирования и информатики . Ульяновск: Изд-во SIMJET, 2015. -- С. 204--210.

\textbf{Структура и объем работы.} Диссертация содержит введение, четыре главы, заключение и список использованной литературы, содержащий 185 наименований. Общий объем диссертации составляет 124 страниц машинописного текста, иллюстрированного 55 рисунками и 13 таблицами.

Кратко изложим содержание \textbf{основных разделов работы}.

\textbf{Глава 1.} Дан обзор приложения машинного обучения для решения указанных задач. Проведен анализ алгоритмов, методов машинного обучения и представлены результаты исследований в этих областях. На основании этого сформулирована постановка задачи диссертационного исследования. Описан предлагаемый подход для решения задачи автоматического обнаружения и классификации дефектов дорожного покрытия и задачи обнаружения и измерения пузырьков.

\textbf{Глава 2.} В этой главе представлены методы математического моделирования и численные методы решения поставленных задач, используя методы машинного обучения. Работа сфокусирована на решении следующих задачах: предварительная обработка изображений, улучшение качества входных данных с помощью алгоритмов обработки изображений, методы машинного обучения для обнаружения и классификации объектов. 

\textbf{Глава 3.} В этой главе дается описание реализации алгоритмов и методов машинного обучения для построения систем ОКДДП и ОКФП для которых в главе \ref{chapt2} были разработаны соответствующие математические модели. Описана структура программ ОКДДП и ОКФП и приведены результаты экспериментов для оценки разработанного метода обнаружения и классификации данных на изображении. Проанализированы результаты предлагаемого метода в сравнении с другими передовыми методами. 

\textbf{Глава 4.} В этой главе представлена среда и инструменты для разработки программного обеспечения для обнаружения и классификации объектов систем ОКДДП и ОКФП. Описаны интерфейс и результаты каждого программного обеспечения.

